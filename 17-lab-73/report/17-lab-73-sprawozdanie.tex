\documentclass[a4paper,12pt]{article}
\usepackage[left=2cm,right=2cm,top=2cm,bottom=2cm]{geometry} % Do ustawień marginesów
\usepackage{multicol} % Dla podziału na kolumny
\usepackage{ragged2e} % Dla justowania tekstu
\usepackage{graphicx} % Required for inserting images
\usepackage{float}
\usepackage{caption}
\usepackage{amsmath} % Math formulas
\usepackage{amssymb} % Symbols
\usepackage[svgnames]{xcolor}
\usepackage[colorlinks=true, urlcolor=blue, linkcolor=black, citecolor=orange]{hyperref} % Hyperlinks
\usepackage{polski} % Polish language
\usepackage[utf8]{inputenc} % Text encoding
\usepackage{enumitem} % Pakiet do elastycznego sterowania listami
\usepackage{indentfirst}
\usepackage{array}
\usepackage{longtable}
\usepackage{pdflscape}
\usepackage[round]{natbib}
\setlist[itemize]{itemsep=0pt, topsep=0pt}
\usepackage{siunitx}
\sisetup{output-decimal-marker={,}}
\sisetup{exponent-product = \cdot}
% LTeX: language=pl-PL

\begin{document}

% Górna część strony
\noindent
\begin{minipage}{0.5\textwidth}
	\raggedright
	\textbf{} \\
	II rok, Fizyka \\
	Wtorek, 8:00-10:15
	\vspace{0.5cm}
\end{minipage}
\begin{minipage}{0.5\textwidth}
	\raggedleft
	18.11.2025 \\
	\vspace{0.5cm}
	Prowadząca: \\
	dr Sylwia Owczarek
\end{minipage}

% Tytuł ćwiczenia
\vspace{2cm}
\begin{center}
	\LARGE \textbf{Ćwiczenie nr 73} \\[0.5cm]
	\Large \textbf{Wyznaczanie prędkości fali dźwiękowej w powietrzu metodą rury rezonansowej}
\end{center}

% Reszta treści
\vspace{1cm} % Kolejny odstęp
\noindent

% \tableofcontents
% \newpage

% ---------- WSTĘP TEORETYCZNY ----------
\section{Wstęp teoretyczny}

\subsection{Rodzaje fal, równanie falowe, prędkość fazowa fali}
Fale można podzielić na podłużne i poprzeczne, w zależności od kierunku drgań ośrodka względem kierunku rozchodzenia się fali. Fala dźwiękowa w powietrzu jest falą mechaniczną podłużną, polegającą na rozchodzeniu się zaburzeń gęstości i ciśnienia ośrodka. Ogólne równanie różniczkowe fali (równanie falowe) dla jednowymiarowego przypadku ma postać:
\begin{equation}
	\frac{\partial^2 \Psi}{\partial x^2} = \frac{1}{v^2} \frac{\partial^2 \Psi}{\partial t^2}
\end{equation}
gdzie $\Psi(x,t)$ reprezentuje zaburzenie (np. ciśnienie akustyczne), a $v$ jest prędkością fazową fali. Prędkość fazowa opisuje szybkość przemieszczania się punktów o stałej fazie i dla fali harmonicznej wyraża się wzorem $v = \frac{\omega}{k}$, gdzie $\omega$ to częstość kołowa, a $k$ to liczba falowa \citep{fizyka_dla_szkół_wyższych_tom_3}.

\subsection{Równanie opisujące falę harmoniczną}
Najprostszym rozwiązaniem równania falowego jest fala harmoniczna (sinusoidalna). Równanie opisujące taką falę biegnącą w kierunku dodatnim osi $x$ ma postać:
\begin{equation}
	\Psi(x,t) = A \sin(kx - \omega t + \phi)
\end{equation}
gdzie $A$ to amplituda fali, $k = \frac{2\pi}{\lambda}$ jest liczbą falową, $\omega = 2\pi f$ jest częstością kołową, a $\phi$ fazą początkową. Wielkości te wiążą się z prędkością propagacji zależnością $v = \lambda f$ \citep{fizyka_dla_szkół_wyższych_tom_3}.

\subsection{Rezonans akustyczny - fala stojąca}
Fala stojąca powstaje w wyniku interferencji (superpozycji) dwóch fal o tej samej częstotliwości i amplitudzie, biegnących w przeciwne strony. Zjawisko to zachodzi w ograniczonych ośrodkach, np. w rurze, gdzie fala padająca nakłada się z falą odbitą od końca rury. Równanie fali stojącej można zapisać jako:
\begin{equation}
	\Psi_{st}(x,t) = 2A \sin(kx) \cos(\omega t)
\end{equation}
Charakterystyczną cechą fali stojącej jest występowanie węzłów (punktów o zerowej amplitudzie) oraz strzałek (punktów o maksymalnej amplitudzie). Odległość między sąsiednimi węzłami (lub strzałkami) wynosi $\frac{\lambda}{2}$. W rurze jednostronnie zamkniętej (tzw. rezonator ćwierćfalowy) węzeł przemieszczenia cząsteczek powstaje na końcu zamkniętym (strzałka ciśnienia), a strzałka przemieszczenia na końcu otwartym (węzeł ciśnienia). Warunek rezonansu dla takiej rury o długości $L$ jest spełniony, gdy:
\begin{equation}
	L = (2n - 1) \frac{\lambda}{4}, \quad \text{gdzie } n = 1, 2, 3, \dots
\end{equation}

\subsection{Prędkość rozchodzenia się fali dźwiękowej w powietrzu}
Prędkość dźwięku w gazach zależy od ich właściwości termodynamicznych. Zgodnie z teorią kinetyczną gazów, prędkość ta wyraża się wzorem:
\begin{equation}
	v = \sqrt{\frac{\kappa R T}{M}}
\end{equation}
gdzie $\kappa$ to wykładnik adiabaty (dla powietrza ok. 1,4), $R$ to uniwersalna stała gazowa, $T$ to temperatura bezwzględna, a $M$ to masa molowa gazu. Widać stąd, że prędkość dźwięku rośnie wraz z pierwiastkiem temperatury \citep{fizyka_dla_szkół_wyższych_tom_2}.
Pomiar prędkości dźwięku można zrealizować metodą rury rezonansowej (jak w niniejszym ćwiczeniu) lub np. rurą Kundta, gdzie wizualizuje się węzły i strzałki za pomocą pyłu korkowego. W obecnym ćwiczeniu wykorzystuje się zmianę długości słupa powietrza ("Puzon") przy stałej częstotliwości źródła, co pozwala wyznaczyć długość fali $\lambda$ z odległości między maksimami natężenia dźwięku.

\subsection{Generacja i detekcja fal dźwiękowych}
W układzie pomiarowym źródłem fali dźwiękowej jest głośnik zasilany z generatora akustycznego, który przetwarza sygnał elektryczny na drgania mechaniczne membrany. Detekcja odbywa się za pomocą mikrofonu, który zamienia zmiany ciśnienia akustycznego na sygnał napięciowy. Sygnał ten jest następnie wzmacniany i mierzony za pomocą woltomierza. Maksymalne wskazania woltomierza odpowiadają występowaniu rezonansu (strzałki fali stojącej) przy mikrofonie.

% ---------- OPIS DOŚWIADCZENIA ----------
% \section{Opis doświadczenia}

% ---------- OPRACOWANIE WYNIKÓW POMIARÓW ----------
% \section{Opracowanie wyników pomiarów}

% ---------- TABELE ----------
% \subsection{Tabele pomiarowe}

% ---------- OBLICZENIA ----------
% \subsection{...}

% ---------- NIEPEWNOŚCI ----------
% \section{Ocena niepewności pomiaru}

% ---------- WNIOSKI ----------
% \section{Wnioski}

% ---------- WYKRESY ----------
% \section{Wykresy}

\bibliographystyle{apalike}
\bibliography{bibliography}

\end{document}
