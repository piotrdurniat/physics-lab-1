\documentclass[a4paper,12pt]{article}
\usepackage[left=2cm,right=2cm,top=2cm,bottom=2cm]{geometry}
\usepackage{multicol}
\usepackage{ragged2e}
\usepackage{graphicx}
\usepackage{float}
\usepackage{caption}
\usepackage{amsmath}
\usepackage{amssymb}
\usepackage[svgnames]{xcolor}
\usepackage[colorlinks=true, urlcolor=blue, linkcolor=black, citecolor=orange]{hyperref}
\usepackage{polski}
\usepackage[utf8]{inputenc}
\usepackage{enumitem}
\usepackage{indentfirst}
\usepackage{array}
\usepackage{longtable}
\usepackage{pdflscape}
\usepackage[round]{natbib}
\setlist[itemize]{itemsep=0pt, topsep=0pt}
\usepackage{siunitx}
\sisetup{output-decimal-marker={,}}
\sisetup{exponent-product = \cdot}

\begin{document}

% Górna część strony
\noindent
\begin{minipage}{0.5\textwidth}
	\raggedright
	\textbf{Piotr Durniat, 347264} \\
	II rok, Fizyka \\
	Wtorek, 8:00-10:15
	\vspace{0.5cm}
\end{minipage}
\begin{minipage}{0.5\textwidth}
	\raggedleft
	18.11.2025 \\
	\vspace{0.5cm}
	Prowadząca: \\
	dr Sylwia Owczarek
\end{minipage}

% Tytuł ćwiczenia
\vspace{2cm}
\begin{center}
	\LARGE \textbf{Ćwiczenie nr 73} \\[0.5cm]
	\Large \textbf{Wyznaczanie prędkości fali dźwiękowej w powietrzu metodą rury rezonansowej}
\end{center}

\vspace{1cm}
\noindent

% ---------- WSTĘP TEORETYCZNY ----------
\section{Wstęp teoretyczny}

\subsection{Rodzaje fal i równanie falowe}
Fale można podzielić na podłużne i poprzeczne, w zależności od kierunku drgań ośrodka względem kierunku rozchodzenia się fali. Fala dźwiękowa w powietrzu jest falą mechaniczną podłużną, polegającą na rozchodzeniu się zaburzeń gęstości i ciśnienia ośrodka \citep{szczeniowski1972}. Ogólne równanie różniczkowe fali (równanie falowe) dla jednowymiarowego przypadku ma postać:
\begin{equation}
	\frac{\partial^2 \Psi}{\partial x^2} = \frac{1}{v^2} \frac{\partial^2 \Psi}{\partial t^2}
\end{equation}
gdzie $\Psi(x,t)$ reprezentuje zaburzenie (np. ciśnienie akustyczne), a $v$ jest prędkością fazową fali.

\subsection{Rezonans akustyczny i fala stojąca}
Fala stojąca powstaje w wyniku interferencji dwóch fal o tej samej częstotliwości i amplitudzie, biegnących w przeciwne strony \citep{szydlowski1999}. W rurze jednostronnie zamkniętej (rezonator ćwierćfalowy) na końcu zamkniętym powstaje węzeł przemieszczenia (strzałka ciśnienia), a na końcu otwartym strzałka przemieszczenia (węzeł ciśnienia). Warunek rezonansu dla takiej rury o długości $L$ jest spełniony, gdy:
\begin{equation}
	L = (2n - 1) \frac{\lambda}{4}, \quad \text{gdzie } n = 1, 2, 3, \dots
\end{equation}

\subsection{Prędkość dźwięku w powietrzu}
Prędkość dźwięku w gazach zależy od ich właściwości termodynamicznych. Zgodnie z teorią kinetyczną gazów:
\begin{equation}
	v = \sqrt{\frac{\kappa R T}{M}}
\end{equation}
gdzie $\kappa$ to wykładnik adiabaty (dla powietrza $\approx 1,4$), $R$ to uniwersalna stała gazowa, $T$ to temperatura bezwzględna, a $M$ to masa molowa gazu \citep{fizyka_dla_szkół_wyższych_tom_2}.

% ---------- OPIS DOŚWIADCZENIA ----------
\section{Opis doświadczenia}

Układ pomiarowy zestawiono zgodnie ze schematem w instrukcji \citep{instrukcja73}, łącząc generator akustyczny z głośnikiem oraz układ detekcyjny składający się z mikrofonu, wzmacniacza i woltomierza.
Pomiary przeprowadzono w dwóch etapach:
\begin{enumerate}
	\item Dla częstotliwości \SI{800}{\hertz} zmierzono rozkład napięcia wzdłuż rury, przesuwając mikrofon co \SI{1}{\centi\meter}, aby wyznaczyć kształt fali stojącej i zlokalizować węzły.
	\item Dla częstotliwości \SI{1}{\kilo\hertz}, \SI{1.5}{\kilo\hertz}, \SI{2}{\kilo\hertz} oraz \SI{2.5}{\kilo\hertz} wyznaczono położenia strzałek i węzłów fali stojącej, szukając maksimów i minimów napięcia na woltomierzu.
\end{enumerate}

\section{Tabele pomiarowe}

\begin{table}[H]
	\centering
	\begin{tabular}{|c|c|}
		\hline
		\textbf{Odległość $x$ [cm]} & \textbf{Napięcie $U$ [V]} \\ \hline
		90                          & 0,071                     \\ \hline
		89                          & 0,061                     \\ \hline
		88                          & 0,046                     \\ \hline
		87                          & 0,029                     \\ \hline
		86                          & 0,016                     \\ \hline
		85                          & 0,025                     \\ \hline
		84                          & 0,029                     \\ \hline
		83                          & 0,045                     \\ \hline
		82                          & 0,058                     \\ \hline
		81                          & 0,074                     \\ \hline
		80                          & 0,084                     \\ \hline
	\end{tabular}
	\caption{Rozkład napięcia wzdłuż rury dla częstotliwości $f = 800$ Hz.}
	\label{tab:pomiar_800}
\end{table}

\begin{table}[H]
	\centering
	\begin{tabular}{|c|c|c|c|}
		\hline
		\multicolumn{2}{|c|}{\textbf{Węzły (min)}} & \multicolumn{2}{c|}{\textbf{Strzałki (max)}}                      \\ \hline
		$x$ [cm]                                   & $U$ [V]                                      & $x$ [cm] & $U$ [V] \\ \hline
		87,5                                       & 0,014                                        & 80,0     & 0,215   \\ \hline
		70,5                                       & 0,018                                        & 62,5     & 0,226   \\ \hline
		53,0                                       & 0,021                                        & 46,0     & 0,193   \\ \hline
		37,0                                       & 0,032                                        & 29,0     & 0,222   \\ \hline
	\end{tabular}
	\caption{Położenia węzłów i strzałek dla $f_{nom}=1000$ Hz ($f_{m}=1000,596$ Hz).}
	\label{tab:pomiar_1000}
\end{table}

\begin{table}[H]
	\centering
	\begin{tabular}{|c|c|c|c|}
		\hline
		\multicolumn{2}{|c|}{\textbf{Węzły (min)}} & \multicolumn{2}{c|}{\textbf{Strzałki (max)}}                      \\ \hline
		$x$ [cm]                                   & $U$ [V]                                      & $x$ [cm] & $U$ [V] \\ \hline
		90,0                                       & 0,016                                        & 84,0     & 0,176   \\ \hline
		79,0                                       & 0,017                                        & 73,0     & 0,176   \\ \hline
		67,5                                       & 0,018                                        & 61,0     & 0,176   \\ \hline
		56,0                                       & 0,021                                        & 49,0     & 0,178   \\ \hline
		44,0                                       & 0,050                                        & 38,0     & 0,180   \\ \hline
		33,0                                       & 0,052                                        & 27,0     & 0,173   \\ \hline
	\end{tabular}
	\caption{Położenia węzłów i strzałek dla $f_{nom}=1500$ Hz ($f_{m}=1500,795$ Hz).}
	\label{tab:pomiar_1500}
\end{table}

\begin{table}[H]
	\centering
	\begin{tabular}{|c|c|c|c|}
		\hline
		\multicolumn{2}{|c|}{\textbf{Węzły (min)}} & \multicolumn{2}{c|}{\textbf{Strzałki (max)}}                      \\ \hline
		$x$ [cm]                                   & $U$ [V]                                      & $x$ [cm] & $U$ [V] \\ \hline
		92,0                                       & 0,025                                        & 87,5     & 0,265   \\ \hline
		83,0                                       & 0,027                                        & 79,5     & 0,263   \\ \hline
		75,0                                       & 0,030                                        & 70,5     & 0,263   \\ \hline
		66,0                                       & 0,034                                        & 63,0     & 0,264   \\ \hline
		57,5                                       & 0,037                                        & 53,0     & 0,247   \\ \hline
		49,0                                       & 0,045                                        & 44,5     & 0,247   \\ \hline
		40,0                                       & 0,048                                        & 36,0     & 0,247   \\ \hline
		31,5                                       & 0,049                                        & --       & --      \\ \hline
	\end{tabular}
	\caption{Położenia węzłów i strzałek dla $f_{nom}=2000$ Hz ($f_{m}=2025,44$ Hz).}
	\label{tab:pomiar_2000}
\end{table}

\begin{table}[H]
	\centering
	\begin{tabular}{|c|c|c|c|}
		\hline
		\multicolumn{2}{|c|}{\textbf{Węzły (min)}} & \multicolumn{2}{c|}{\textbf{Strzałki (max)}}                      \\ \hline
		$x$ [cm]                                   & $U$ [V]                                      & $x$ [cm] & $U$ [V] \\ \hline
		85,0                                       & 0,031                                        & 89,0     & 0,212   \\ \hline
		78,5                                       & 0,034                                        & 82,0     & 0,213   \\ \hline
		72,0                                       & 0,036                                        & 75,0     & 0,213   \\ \hline
		65,0                                       & 0,036                                        & 68,5     & 0,213   \\ \hline
		58,0                                       & 0,038                                        & 61,5     & 0,214   \\ \hline
		51,0                                       & 0,040                                        & 54,5     & 0,215   \\ \hline
		44,0                                       & 0,041                                        & 48,0     & 0,182   \\ \hline
		37,0                                       & 0,043                                        & 41,0     & 0,184   \\ \hline
		30,0                                       & 0,044                                        & 34,0     & 0,183   \\ \hline
		--                                         & --                                           & 26,5     & 0,181   \\ \hline
	\end{tabular}
	\caption{Położenia węzłów i strzałek dla $f_{nom}=2500$ Hz ($f_{m}=2501,82$ Hz).}
	\label{tab:pomiar_2500}
\end{table}


% ---------- OPRACOWANIE WYNIKÓW POMIARÓW ----------
\section{Opracowanie wyników pomiarów}

\subsection{Badanie fali stojącej dla f = 800 Hz}

Sporządzono wykres zależności napięcia $U$ od położenia mikrofonu $x$ dla częstotliwości $f = \SI{800}{\hertz}$, który przedstawiono w sekcji \ref{sec:wykresy} (Rys. \ref{fig:wykres_800}). Na wykresie zaznaczono teoretyczną skalę długości fali, ponieważ zakres ruchu mikrofonu był krótszy od pełnej długości fali $\lambda$.

\subsection{Wyznaczenie prędkości dźwięku}

Dla częstotliwości z zakresu \SI{1}{\kilo\hertz} -- \SI{2.5}{\kilo\hertz} wyznaczono średnie odległości między węzłami $\Delta x_{avg}$. Długość fali obliczono jako $\lambda = 2 \cdot \Delta x_{avg}$. Prędkość fali obliczono ze wzoru $v = \lambda f$.

\subsubsection*{Przykładowe obliczenia (dla $f=\SI{1000}{\hertz}$)}
Dla częstotliwości \SI{1000}{\hertz} odczytano następujące położenia węzłów (Tabela \ref{tab:pomiar_1000}): $87,5\,\text{cm}$, $70,5\,\text{cm}$, $53,0\,\text{cm}$ oraz $37,0\,\text{cm}$. Obliczono odległości między sąsiednimi węzłami:
\begin{align*}
	\Delta x_1 & = |87,5 - 70,5| = 17,0\,\text{cm} \\
	\Delta x_2 & = |70,5 - 53,0| = 17,5\,\text{cm} \\
	\Delta x_3 & = |53,0 - 37,0| = 16,0\,\text{cm}
\end{align*}
Średnia odległość między węzłami wynosi:
$$ \Delta x_{avg} = \frac{17,0 + 17,5 + 16,0}{3}\,\text{cm} \approx 16,833\,\text{cm} = 0,1683\,\text{m} $$

Następnie obliczono długość fali oraz prędkość dźwięku:
$$ \lambda = 2 \cdot \Delta x_{avg} = 2 \cdot 0,1683\,\si{\meter} = 0,3366\,\si{\meter} \approx 0,3367\,\si{\meter} $$
$$ v \approx 0,3367\,\si{\meter} \cdot 1000\,\si{\hertz} = 336,7\,\si{\meter\per\second} $$

Wyniki dla wszystkich częstotliwości przedstawiono w Tabeli \ref{tab:wyniki}.

\begin{table}[H]
	\centering
	\begin{tabular}{|S[table-format=4.0]|S[table-format=1.4]|S[table-format=1.4]|S[table-format=3.1]|}
		\hline
		{\textbf{Częstotliwość $f$ [\si{\hertz}]}} & {\textbf{$\Delta x_{avg}$ [\si{\meter}]}} & {\textbf{$\lambda$ [\si{\meter}]}} & {\textbf{$v$ [\si{\meter\per\second}]}} \\ \hline
		1000                                       & 0.1683                                    & 0.3367                             & 336.7                                   \\ \hline
		1500                                       & 0.1140                                    & 0.2280                             & 342.0                                   \\ \hline
		2000                                       & 0.0864                                    & 0.1729                             & 345.7                                   \\ \hline
		2500                                       & 0.0688                                    & 0.1375                             & 343.8                                   \\ \hline
	\end{tabular}
	\caption{Wyniki pomiarów długości fali i obliczone prędkości dźwięku.}
	\label{tab:wyniki}
\end{table}

Wartość średnia prędkości dźwięku obliczona dla $n=4$ pomiarów:
\begin{equation}
	v_{\text{sr}} = \frac{336,7 + 342,0 + 345,7 + 343,8}{4}\,\si{\meter\per\second} = \SI{342.0}{\meter\per\second}
\end{equation}

% ---------- NIEPEWNOŚCI ----------
\section{Ocena niepewności pomiaru}

\subsection{Niepewność standardowa pojedynczego pomiaru $u(x)$}
Niepewność pomiaru położenia mikrofonu oszacowano metodą typu B \citep{onp}. Przyjęto niepewność maksymalną eksperymentatora $\Delta x = \SI{0.5}{\centi\meter}$. Zakładając rozkład jednostajny:
\begin{equation}
	u(x) = \frac{\Delta x}{\sqrt{3}} = \frac{0,5}{\sqrt{3}} \approx \SI{0.29}{\centi\meter}
\end{equation}

\subsection{Niepewność złożona $u(v)$ i rozszerzona $U(v)$}
Niepewność standardową $u(v)$ obliczono metodą typu A (odchylenie standardowe średniej) \citep{onp}.
Podstawiając $v_{\text{sr}} = 342,0$:
\begin{align*}
	u(v) & = \sqrt{\frac{1}{4(4-1)}\left[ (336,7 - 342,0)^2 + \dots + (343,8 - 342,0)^2 \right]}               \\
	     & = \sqrt{\frac{45,02}{12}} \approx \SI{1.936}{\meter\per\second} \approx \SI{1.9}{\meter\per\second}
\end{align*}

Niepewność rozszerzoną $U(v)$ obliczono dla poziomu ufności $95\%$ i $\nu = n-1 = 3$ stopni swobody, przyjmując współczynnik rozszerzenia $k = 3.18$ \citep{onp}:
\begin{equation}
	U(v) = k \cdot u(v) = 3,18 \cdot 1,936\,\si{\meter\per\second} \approx \SI{6.15}{\meter\per\second} \approx \SI{6.2}{\meter\per\second}
\end{equation}

% ---------- WNIOSKI ----------
\section{Wnioski}

\begin{enumerate}
	\item Wyznaczona metodą rury rezonansowej średnia prędkość dźwięku w powietrzu oraz jej niepewność standardowa wynoszą:
	      \begin{equation}
		      v = \num{342.0} \, \si{\meter\per\second}
	      \end{equation}
	      \begin{equation}
		      u(v) = \num{1.9} \, \si{\meter\per\second}
	      \end{equation}
	\item Wartość teoretyczna dla $T = \SI{293.15}{\kelvin}$ wynosi $v_{teor} = \SI{343.3}{\meter\per\second}$ \citep{fizyka_dla_szkół_wyższych_tom_2}.
	\item Błąd względny pomiaru wynosi:
	      $$ \delta = \frac{|342,0 - 343,3|}{343,3} \cdot 100\% \approx 0,36\% $$
	      Wynik eksperymentalny jest zgodny z wartością teoretyczną (różnica $|\Delta v| = \SI{1.3}{\meter\per\second}$ jest mniejsza od niepewności standardowej $u(v) \approx \SI{1.9}{\meter\per\second}$).
\end{enumerate}

\newpage

% ---------- WYKRESY ----------
\section{Wykresy}
\label{sec:wykresy}

\begin{figure}[H]
	\centering
	\includegraphics[angle=90, height=1.3\textwidth, keepaspectratio]{img/u_x_fixed.png}
	\caption{Zależność napięcia sygnału od położenia mikrofonu dla $f=\SI{800}{\hertz}$. Zaznaczono położenie węzła oraz skalę długości fali.}
	\label{fig:wykres_800}
\end{figure}

% ---------- LITERATURA ----------
\bibliographystyle{apalike}
\bibliography{bibliography}

\end{document}
