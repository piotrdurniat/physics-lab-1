\documentclass{article}
\usepackage{amsmath, amssymb}
\usepackage[polish]{babel}
\usepackage[utf8]{inputenc}

\title{Wstęp teoretyczny do ćwiczenia nr 11: \\
Badanie ruchu obrotowego bryły sztywnej}
\author{Piotr Durniat}
\date{Marzec 2025}

\begin{document}

\maketitle

\section{Podstawy teoretyczne}

\subsection{I zasada dynamiki ruchu obrotowego}
I zasada dynamiki ruchu obrotowego stwierdza, że ciało sztywne, na które nie działa moment siły, pozostaje w spoczynku lub porusza się ruchem obrotowym jednostajnym. Matematycznie można to zapisać jako:
\begin{equation}
    \sum \vec{M} = 0 \Rightarrow \frac{d\vec{L}}{dt} = 0
\end{equation}

Moment pędu wyraża się wzorem:
\begin{equation}
    \vec{L} = I \vec{\omega}
\end{equation}

\subsection{Tensor momentu bezwładności}
Tensor momentu bezwładności opisuje rozkład masy ciała względem osi obrotu. Jest to wielkość tensorowa drugiego rzędu, reprezentowana przez macierz 3×3:
\begin{equation}
    I = \begin{bmatrix} I_{xx} & I_{xy} & I_{xz} \\
                I_{yx} & I_{yy} & I_{yz} \\
                I_{zx} & I_{zy} & I_{zz}\end{bmatrix}
\end{equation}

\subsection{Elipsoida bezwładności}
Elipsoida bezwładności ciała sztywnego to powierzchnia, której odległość $r$ od środka masy w dowolnym kierunku jest związana z momentem bezwładności $I$ w tym kierunku zależnością:
\begin{equation}
    I = \frac{1}{r^2}
\end{equation}
Dla dowolnej osi obrotu zachodzi twierdzenie Steinera:
\begin{equation}
    I_1 = I_0 + M d^2
\end{equation}

\subsection{Wahadło fizyczne}
Dla małych wychyleń wahadła fizycznego równanie ruchu przyjmuje postać:
\begin{equation}
    \frac{d^2 \theta}{dt^2} + \frac{M g d}{I} \theta = 0
\end{equation}

Okres drgań wahadła fizycznego:
\begin{equation}
    T = 2\pi \sqrt{\frac{I}{M g d}}
\end{equation}

\subsection{Wahadło torsyjne}
Wahadło torsyjne wykonuje drgania skrętne pod wpływem momentu siły sprężystości:
\begin{equation}
    M = -k\theta
\end{equation}

Okres drgań wahadła torsyjnego:
\begin{equation}
    T = 2\pi \sqrt{\frac{I}{k}}
\end{equation}

\end{document}
