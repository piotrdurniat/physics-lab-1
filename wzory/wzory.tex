\documentclass[a4paper,12pt]{article}
\usepackage[left=2cm,right=2cm,top=2cm,bottom=2cm]{geometry} % Do ustawień marginesów
\usepackage{multicol} % Dla podziału na kolumny
\usepackage{ragged2e} % Dla justowania tekstu
\usepackage{graphicx} % Required for inserting images
\usepackage{float}
\usepackage{caption}
\usepackage{amsmath} % Math formulas
\usepackage{amssymb} % Symbols
\usepackage[svgnames]{xcolor}
\usepackage[colorlinks=true, urlcolor=blue, linkcolor=black, citecolor=orange]{hyperref} % Hyperlinks
\usepackage{polski} % Polish language
\usepackage[utf8]{inputenc} % Text encoding
\usepackage{enumitem} % Pakiet do elastycznego sterowania listami
\usepackage{indentfirst}
\usepackage{array}

\setlist[itemize]{itemsep=0pt, topsep=0pt}

\begin{document}

\section{Wzory}

\subsection{Niepewność standardowa typu A}
Niepewność standardowa typu A wyznaczana jest na podstawie rozrzutu wyników pomiarów:

\begin{equation}
    u_A(x) = \sqrt{\frac{1}{N-1} \sum_{i=1}^{N} (x_i - \bar{x})^2}
\end{equation}

gdzie:
\begin{itemize}
    \item $N$ - liczba pomiarów
    \item $x_i$ - i-ty wynik pomiaru
    \item $\bar{x}$ - wartość średnia wyników pomiarów
\end{itemize}

\subsection{Niepewność standardowa typu B}
Niepewność standardowa typu B wyznaczana jest na podstawie niepewności wzorcowania przyrządu pomiarowego:

\begin{equation}
    u_B(x) = \frac{\Delta_d x}{\sqrt{3}}
\end{equation}

gdzie:
\begin{itemize}
    \item $\Delta_d x$ - niepewność wzorcowania przyrządu pomiarowego
\end{itemize}

\subsection{Całkowita niepewność standardowa}
Całkowita niepewność standardowa jest pierwiastkiem z sumy kwadratów niepewności typu A i B:

\begin{equation}
    u_c(x) = \sqrt{u_A^2(x) + u_B^2(x)}
\end{equation}

\subsection{Prawo przenoszenia niepewności standardowych}
Dla funkcji wielu zmiennych $y = f(x_1, x_2, \ldots, x_K)$, niepewność standardowa złożona wyznaczana jest ze wzoru:

\begin{equation}
    u_c(y) = \sqrt{\sum_{k=1}^{K} \left( \frac{\partial f}{\partial x_k} \right)^2 u^2(x_k)}
\end{equation}

gdzie:
\begin{itemize}
    \item $\frac{\partial f}{\partial x_k}$ - pochodna cząstkowa funkcji $f$ względem zmiennej $x_k$
    \item $u(x_k)$ - niepewność standardowa zmiennej $x_k$
\end{itemize}

\subsection{Prawo przenoszenia niepewności maksymalnych}

\begin{equation}
    \Delta y = \sum_{i=1}^{n} \left | \frac{\partial f}{\partial x_i} \right | \cdot \Delta x_i
\end{equation}

\subsection{Niepewność względna}
Często używana jest również niepewność względna, która pozwala określić procentową niepewność wyniku:

\begin{equation}
    \frac{\Delta y}{|y|} = \sum_{i=1}^{n} \frac{\left | \frac{\partial f}{\partial x_i} \right | \cdot \Delta x_i}{|y|}
\end{equation}

W przypadku prostych zależności funkcyjnych (jak np. iloczyn czy iloraz), niepewność względna może być wyrażona jako suma niepewności względnych poszczególnych zmiennych:

\begin{equation}
    \frac{\Delta g}{|g|} = \frac{\Delta x}{|x|} + \frac{\Delta y}{|y|} + \ldots
\end{equation}

\subsection{Niepewność średniej arytmetycznej}
Niepewność średniej arytmetycznej z $N$ pomiarów wyznaczana jest ze wzoru:

\begin{equation}
    u(\bar{x}) = \frac{u(x)}{\sqrt{N}}
\end{equation}

\subsection{Niepewność pomiarowa współczynników prostej regresji liniowej}

Niepewności pomiarowe dla wyznaczonej prostej regresji liniowej $y = ax + b$ obliczono na podstawie odchylenia standardowego reszt $s_y$ oraz rozkładu punktów pomiarowych wzdłuż osi $x$, korzystając z następujących wzorów:

\[
    s_y = \sqrt{\frac{\sum_{i=1}^{n} (y_i - \hat{y}_i)^2}{n-2}}
\]

\[
    u_a = s_y \sqrt{\frac{n}{n \sum x_i^2 - \left( \sum x_i \right)^2}}
\]

\[
    u_b = s_y \sqrt{\frac{\sum x_i^2}{n \sum x_i^2 - \left( \sum x_i \right)^2}}
\]

gdzie $x_i$ to wartości zmiennej niezależnej, $y_i$ to wartości zmierzone, $\hat{y}_i$ to wartości przewidywane przez model regresji, a $n$ to liczba punktów pomiarowych. Dzielnik $n-2$ wynika z faktu, że model regresji liniowej ma dwa parametry ($a$ i $b$).

\subsection{Szczególne przypadki przenoszenia niepewności}

\subsubsection{Suma lub różnica}
Dla funkcji $f = x \pm y$:
\begin{equation}
    u(f) = \sqrt{u^2(x) + u^2(y)}
\end{equation}

\subsubsection{Iloczyn lub iloraz}
Dla funkcji $f = x \cdot y$ lub $f = \frac{x}{y}$:
\begin{equation}
    \frac{u(f)}{|f|} = \sqrt{\left(\frac{u(x)}{|x|}\right)^2 + \left(\frac{u(y)}{|y|}\right)^2}
\end{equation}

\subsubsection{Potęgowanie}
Dla funkcji $f = x^n$:
\begin{equation}
    \frac{u(f)}{|f|} = |n| \cdot \frac{u(x)}{|x|}
\end{equation}

\end{document}

