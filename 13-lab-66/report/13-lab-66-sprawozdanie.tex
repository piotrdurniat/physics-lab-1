\documentclass[a4paper,12pt]{article}
\usepackage[left=2cm,right=2cm,top=2cm,bottom=2cm]{geometry} % Do ustawień marginesów
\usepackage{multicol} % Dla podziału na kolumny
\usepackage{ragged2e} % Dla justowania tekstu
\usepackage{graphicx} % Required for inserting images
\usepackage{float}
\usepackage{caption}
\usepackage{amsmath} % Math formulas
\usepackage{amssymb} % Symbols
\usepackage[svgnames]{xcolor}
\usepackage[colorlinks=true, urlcolor=blue, linkcolor=black, citecolor=orange]{hyperref} % Hyperlinks
\usepackage{polski} % Polish language
\usepackage[utf8]{inputenc} % Text encoding
\usepackage{enumitem} % Pakiet do elastycznego sterowania listami
\usepackage{indentfirst}
\usepackage{array}
\usepackage{longtable}
\usepackage{pdflscape}
\setlist[itemize]{itemsep=0pt, topsep=0pt}
\usepackage[round]{natbib}

\begin{document}

% Górna część strony
\noindent
\begin{minipage}{0.5\textwidth}
    \raggedright
    \textbf{Piotr Durniat, 347264} \\
    II rok, Fizyka \\
    Wtorek, 8:00-10:15 \\
    \vspace{0.5cm}
    \vspace{0.5cm}
\end{minipage}%
\begin{minipage}{0.5\textwidth}
    \raggedleft
    14.10.2025 \\
    \vspace{0.5cm}
    Prowadząca: \\
    dr Sylwia Owczarek
\end{minipage}

% Tytuł ćwiczenia
\vspace{2cm}
\begin{center}
    \LARGE \textbf{Ćwiczenie nr 66} \\[0.5cm]
    \Large \textbf{Analiza widmowa za pomocą spektroskopu}
\end{center}

% Reszta treści
\vspace{1cm} % Kolejny odstęp
\noindent

\tableofcontents
\newpage

% ---------- WSTĘP TEORETYCZNY ----------
\section{Wstęp teoretyczny}

\subsection*{Emisja światła}

Emisja światła to proces emisji fotonów przez atomy. Zachodzi, gdy atomy przechodzą ze stanu wzbudzonego do stanu podstawowego, emitując energię w postaci światła.~\citep{Drynski1976}

\subsection*{Budowa atomu}

Atom składa się z jądra atomowego, zawierającego protony i neutrony, oraz elektronów na określonych poziomach energetycznych.~\citep{Drynski1976}

\subsection*{Rodzaje widm}

Widmo to obraz promieniowania rozłożonego na poszczególne długości fali. Główne rodzaje to:

\begin{itemize}
    \item Widmo emisyjne: Powstaje, gdy świecący gaz emituje światło. Składa się z kolorowych linii na ciemnym tle.
    \item Widmo absorpcyjne: Powstaje, gdy światło przechodzi przez gaz, widoczne jest w postaci czarnych linii na ciągłym spektrum.~\citep{fizyka_dla_szkół_wyższych_tom_3}
\end{itemize}

\subsection*{Serie widmowe}

Serie widmowe to uporządkowane grupy linii w widmie atomu, które odpowiadają przejściom elektronów z różnych wyższych poziomów energetycznych na ten sam, określony niższy poziom (lub odwrotnie). Przykładowo seria Balmera to grupa linii w widmie odpowiadających przejściom elektronów ze lub do stanu $n=2$ atomu wodoru. Opisuje ją wzór Balmera. ~\citep{fizyka_dla_szkół_wyższych_tom_3}

\subsection*{Budowa i zasada działania spektroskopu}

Spektroskop to urządzenie do analizy widm świetlnych. Składa się z:

\begin{itemize}
    \item Kolimatora (K): Tworzy z rozbieżnej wiązki światła wiązkę równoległą.
    \item Pryzmatu (P): Rozszczepia światło na skutek zjawiska dyspersji.
    \item Lunety (L): Umożliwia obserwację powstałego widma.
\end{itemize}

\subsection*{Rozszczepienie światła przez pryzmat (dyspersja)}

Dyspersja to zjawisko zależności współczynnika załamania światła od jego częstotliwości (lub długości fali) w danym ośrodku. Prowadzi to do rozszczepienia światła białego na barwy składowe podczas przejścia np. przez pryzmat.~\citep{Drynski1976}

\subsection*{Analiza widmowa za pomocą spektroskopu}

Substancję identyfikuje się poprzez porównanie jej unikalnego widma z widmami wzorcowymi.

% ---------- OPIS DOŚWIADCZENIA ----------
\section{Opis doświadczenia}

Doświadczenie polegało na analizie widmowej kilku pierwiastków przy użyciu spektroskopu pryzmatycznego.

W pierwszej części ćwiczenia spektroskop został skalibrowany przy użyciu wzorcowego źródła światła -- lampy helowej. Po uzyskaniu ostrych prążków widmowych, zarejestrowano położenia charakterystycznych linii helu na skali przyrządu, notując ich barwę oraz intensywność.

Następnie, bez zmiany ustawień spektroskopu, zmierzono widma emisyjne trzech nieznanych pierwiastków, oznaczonych jako 1, 2 i 3. Dla każdego z nich zarejestrowano położenia obserwowanych linii widmowych na tej samej skali.

% ---------- OPRACOWANIE WYNIKÓW POMIARÓW ----------
\section{Opracowanie wyników pomiarów}

% ---------- TABELE ----------
\subsection{Tabele pomiarowe}

\begin{table}[H]
    \centering
    \begin{tabular}{|r|c|l|}
        \hline
        Nr & Położenie linii [u] & Barwa \\ \hline
        1 & 2{,}0 & Ciemny czerwony \\ \hline
        2 & 2{,}7 & Jasny czerwony \\ \hline
        3 & 3{,}4 & Jasny żółty \\ \hline
        4 & 7{,}5 & Jasny zielony \\ \hline
        5 & 8{,}0 & Ciemny miętowy (zielono-niebieski) \\ \hline
        6 & 9{,}0 & Jasny niebieski \\ \hline
        7 & 10{,}5 & Jasny fioletowy \\ \hline
        8 & 11{,}2 & Ciemny fioletowy \\ \hline
    \end{tabular}
    \caption{Położenie linii w widmie helu (kolor światła: pomarańczowy).}
    \label{tab:hel}
\end{table}

\begin{table}[H]
    \centering
    \begin{tabular}{|c|l|}
        \hline
        Położenie linii [u] & Barwa \\ \hline
        2{,}5 & Żółty \\ \hline
        3{,}6 & Jasny zielony \\ \hline
        5{,}6 & Ciemny niebieski \\ \hline
        9{,}0 & Fioletowy \\ \hline
    \end{tabular}
    \caption{Położenie linii widmowych pierwiastka nr 1 (kolor światła: jasno niebieski/miętowy).}
\end{table}

\begin{table}[H]
    \centering
    \begin{tabular}{|c|l|}
        \hline
        Położenie linii [u] & Barwa \\ \hline
        0{,}0 & Ciemny czerwony \\ \hline
        0{,}1 & Jasny czerwony \\ \hline
        0{,}7 & Bardzo ciemny czerwony \\ \hline
        1{,}0 & Ciemny czerwony \\ \hline
        1{,}5 & Bardzo ciemny czerwony \\ \hline
        1{,}7 & Bardzo ciemny czerwono-pomarańczowy \\ \hline
        2{,}0 & Jasny pomarańczowy \\ \hline
        2{,}8 & Ciemny żółty \\ \hline
        3{,}0 & Ciemny zielony \\ \hline
        3{,}4 & Ciemny zielony \\ \hline
        4{,}5 & Jasny zielony \\ \hline
        6{,}8 & Bardzo jasny miętowy \\ \hline
        8{,}0 & Ciemny fioletowy \\ \hline
        8{,}4 & Ciemny fioletowy \\ \hline
        9{,}4 & Jasny fioletowy \\ \hline
        10{,}3 & Jasny fioletowy \\ \hline
        10{,}6 & Jasny fioletowy \\ \hline
    \end{tabular}
    \caption{Położenie linii widmowych pierwiastka nr 2 (kolor światła: różowy).}
\end{table}

\begin{table}[H]
    \centering
    \begin{tabular}{|c|l|}
        \hline
        Położenie linii [u] & Barwa \\ \hline
        1{,}1 & Jasny czerwony \\ \hline
        1{,}8 & Jasny pomarańczowy \\ \hline
        2{,}4 & Jasny żółty \\ \hline
        3{,}8 & Bardzo ciemny zielony \\ \hline
        4{,}9 & Bardzo ciemny zielony \\ \hline
        6{,}8 & Jasny fioletowy \\ \hline
    \end{tabular}
    \caption{Położenie linii widmowych pierwiastka nr 3 (kolor światła: czerwony).}
\end{table}

% ---------- OBLICZENIA ----------
\subsection{Wyznaczenie krzywej dyspersji}

Na podstawie pomiarów z tabeli \ref{tab:hel} i tabeli długości fal lini widmowych z instrukcji do ćwiczenia, dopasowano położenia linii do długości fal (tabela \ref{tab:hel_dispersion}). Do punktów dopasowano wielomian drugiego stopnia za pomocą funkcji \texttt{np.polyfit}. Dla funkcji postaci $\lambda(u) = au^2 + bu + c$ otrzymano następujące współczynniki:
\begin{itemize}
    \item $a = 2{,}404954$
    \item $b = -58{,}388917$
    \item $c = 797{,}218077$
\end{itemize}

\begin{table}[H]
    \centering
    \begin{tabular}{|r|r|c|r|}
        \hline
        Nr & Położenie linii [u] & Barwa  & Długość fali [nm] \\ \hline
        1 & 2{,}0 & Ciemny czerwony & 706,5  \\ \hline
        2 & 2{,}7 & Jasny czerwony & 667,8\\ \hline
        3 & 3{,}4 & Jasny żółty & 587,6 \\ \hline
        4 & 7{,}5 & Jasny zielony & 501,6 \\ \hline
        5 & 8{,}0 & Ciemny miętowy (zielono-niebieski) & 492,2\\ \hline
        6 & 9{,}0 & Jasny niebieski & 471,3 \\ \hline
        7 & 10{,}5 & Jasny fioletowy & 447,2 \\ \hline
        8 & 11{,}2 & Ciemny fioletowy & 438,8 \\ \hline
    \end{tabular}
    \caption{Przyporządkowane długości fali dla każdej linii w widmie helu.}
    \label{tab:hel_dispersion}
\end{table}

\subsection{Określenie długości fal dla nieznanych pierwiastków}

Na podstawie dopasowanej funkcji obliczono długości fal prążków dla nieznanych pierwiastków. Wyniki, zaokrąglone zgodnie z analizą niepewności, zebrano w tabelach poniżej.

\begin{table}[H]
    \centering
    \begin{tabular}{|c|r|}
        \hline
        Położenie linii [u] & Długość fali [nm] \\ \hline
        2{,}5 & 666 \\ \hline
        3{,}6 & 618 \\ \hline
        5{,}6 & 546 \\ \hline
        9{,}0 & 467 \\ \hline
    \end{tabular}
    \caption{Pierwiastek nr 1: obliczone długości fal.}
    \label{tab:unknown1}
\end{table}

\begin{table}[H]
    \centering
    \begin{tabular}{|c|r|}
        \hline
        Położenie linii [u] & Długość fali [nm] \\ \hline
        0{,}0 & 797 \\ \hline
        0{,}1 & 791 \\ \hline
        0{,}7 & 758 \\ \hline
        1{,}0 & 741 \\ \hline
        1{,}5 & 715 \\ \hline
        1{,}7 & 705 \\ \hline
        2{,}0 & 690 \\ \hline
        2{,}8 & 653 \\ \hline
        3{,}0 & 644 \\ \hline
        3{,}4 & 627 \\ \hline
        4{,}5 & 583 \\ \hline
        6{,}8 & 511 \\ \hline
        8{,}0 & 484 \\ \hline
        8{,}4 & 476 \\ \hline
        9{,}4 & 461 \\ \hline
        10{,}3 & 451 \\ \hline
        10{,}6 & 449 \\ \hline
    \end{tabular}
    \caption{Pierwiastek nr 2: obliczone długości fal.}
    \label{tab:unknown2}
\end{table}

\begin{table}[H]
    \centering
    \begin{tabular}{|c|r|}
        \hline
        Położenie linii [u] & Długość fali [nm] \\ \hline
        1{,}1 & 736 \\ \hline
        1{,}8 & 700 \\ \hline
        2{,}4 & 671 \\ \hline
        3{,}8 & 610 \\ \hline
        4{,}9 & 569 \\ \hline
        6{,}8 & 511 \\ \hline
    \end{tabular}
    \caption{Pierwiastek nr 3: obliczone długości fal.}
    \label{tab:unknown3}
\end{table}

\subsection{Identyfikacja pierwiastków}

Zgodność długości fal z wartościami tablicowymi była niska, stąd skorzystano głównie z analizy sekwencji kolorów i charakterystyki linii widm referencyjnych.

\begin{itemize}
    \item \textbf{Pierwiastek nr 1: Rtęć (Hg)}
          \begin{itemize}
              \item Obserwowana linia \textbf{żółta} (666 nm) najprawdopodobniej odpowiada liniom 577,0 nm i 579,1 nm.
              \item Obserwowana \textbf{jasna linia zielona} (618 nm) odpowiada linii zielonej rtęci (546,1 nm).
              \item Obserwowana \textbf{ciemna linia niebieska} (546 nm) odpowiada średniej intensywności linii niebieskiej (435,8 nm).
              \item Obserwowana linia \textbf{fioletowa} (467 nm) odpowiada liniom fioletowym (404,7 nm i 407,8 nm).
          \end{itemize}

    \item \textbf{Pierwiastek nr 2: Neon (Ne)}
          Cechą diagnostyczną widma tego pierwiastka jest duża liczba intensywnych linii w zakresie czerwonym i pomarańczowym na początku widma, co jest zgodne z obserwacjami.
          \begin{itemize}
              \item Zaobserwowane linie takie jak \textbf{"Ciemny czerwony", "Jasny czerwony"} i \textbf{"Jasny pomarańczowy"}, pasują do spektrum neonu w zakresie 614,3 nm, 640,2 nm, 659,9 nm, 724,5 nm.
              \item Obserwowana linia \textbf{żółta} odpowiada silnym liniom żółtym (585,2 nm i 594,5 nm).
              \item Pozostałe linie (zielone, fioletowe) również mają swoje odpowiedniki w widmie neonu.
          \end{itemize}

    \item \textbf{Pierwiastek nr 3: Argon (Ar)}
          \begin{itemize}
              \item Obserwowane linie \textbf{czerwona} i \textbf{pomarańczowa/żółta} odpowiadają liniom argonu (czerwone: 696,5 nm i 641,6 nm; żółta: 591,2 nm).
              \item Obserwowane linie \textbf{zielone} odpowiadają liniom w zakresie 549,5 nm - 565,0 nm.
              \item Obserwowana linia \textbf{fioletowa} może odpowiadać silnej linii niebieskiej (470,2 nm) lub linii fioletowej (415,8 nm).
          \end{itemize}
\end{itemize}


% ---------- NIEPEWNOŚCI ----------
\section{Ocena niepewności pomiaru}

\subsection{Niepewność położenia na skali}
Niepewność standardową położenia na skali $u(u)$ obliczono na podstawie wzoru na niepewność standardową typu B. Zakładając, że przedział niepewności odczytu wynosi $\pm a = \pm 1$ podziałkę, otrzymano:
\begin{equation*}
    u(u) = \frac{a}{\sqrt{3}} = \frac{1}{\sqrt{3}} \approx 0{,}58 \text{ u}
\end{equation*}


\subsection{Niepewność współczynników wielomianu}

Niepewności standardowe współczynników dopasowanego wielomianu wyznaczono z macierzy kowariancji. Otrzymane wartości wynoszą:

\begin{itemize}
    \item $a = 2{,}4, u(a) = 1{,}0$
    \item $b = -58, u(b) = 13$
    \item $c = 797, u(c) = 33$
\end{itemize}

\subsection{Niepewność długości fali}
Na podstawie wzoru na niepewność złożoną:
$$
    u_c^2(\lambda) = \sum_{i=1}^{n} \left(\frac{\partial \lambda}{\partial x_i}\right)^2 u^2(x_i)
$$
wyprowadzono wzór na niepewność długości fali $\lambda$:
$$
    u_c(\lambda) = \sqrt{ (2au + b)^2 u^2(u) + u^4 u^2(a) + u^2 u^2(b) + u^2(c) }
$$
Dla przykładowego pomiaru $u = 5{,}6$ [u] podstawiono wartości:
\begin{align*}
    u_c(\lambda) & = \sqrt{ (2 \cdot 2{,}4 \cdot 5{,}6 - 58)^2 \cdot 0{,}58^2 + (5{,}6)^4 \cdot (1{,}0)^2 + (5{,}6)^2 \cdot 13^2 + 33^2 } \\
    u_c(\lambda) & = \sqrt{ (26{,}88 - 58)^2 \cdot 0{,}3364 + 983{,}5 \cdot 1 + 31{,}36 \cdot 169 + 1089 }                                \\
    u_c(\lambda) & = \sqrt{ (-31{,}12)^2 \cdot 0{,}3364 + 983{,}5 + 5300 + 1089 }                                                         \\
    u_c(\lambda) & = \sqrt{ 968{,}4 \cdot 0{,}3364 + 7372{,}5 }                                                                           \\
    u_c(\lambda) & = \sqrt{ 325{,}8 + 7372{,}5 }                                                                                          \\
    u_c(\lambda) & = \sqrt{ 7698{,}3 } \approx 87{,}7 \text{ nm}
\end{align*}
Po zaokrągleniu wyniku do jednej cyfry znaczącej otrzymano ostateczną niepewność pomiaru:
$$
    u_c(\lambda) \approx 90 \text{ nm}
$$

gdzie:
\begin{itemize}
    \item $u_c(\lambda)$: Niepewność złożona długości fali $\lambda$.
    \item $u, a, b, c$: Zmierzone położenie i dopasowane współczynniki wielomianu.
    \item $u(u), u(a), u(b), u(c)$: Niepewności standardowe odpowiednich wielkości.
\end{itemize}
% ---------- WNIOSKI ----------
\section{Wnioski}

Na podstawie analizy sekwencji kolorów w obserwowanych widmach, zidentyfikowano następujące pierwiastki:
\begin{itemize}
    \item Pierwiastek nr 1: \textbf{Rtęć (Hg)}
    \item Pierwiastek nr 2: \textbf{Neon (Ne)}
    \item Pierwiastek nr 3: \textbf{Argon (Ar)}
\end{itemize}

Obliczone długości fal wykazały znaczne rozbieżności z wartościami tablicowymi.
Jedną z przyczyn może być dopasowanie do pomiarów wielomianu drugiego stopnia metodą najmniejszych kwadratów. Rzeczywista krzywa dyspersji może mieć inną charakterystykę. Na wykresie \ref{fig:helium_dispersion} widać niskie dopasowanie tej krzywej do punktów pomiarowych, widać również, że pomiar nr. 3 odstaje od pozostałych, co zaburza dopasowanie krzwej. Należało by rozważyć dopasowanie innej krzywej i odrzucenie wartości odstającej.
Drugą, możliwą przyczyną rozbieżności jest przesunięcie skali, do którego mogło dojść po kalibracji.
Dodatkowo trudność sprawiało odczytanie dokładnego położenia na skali, ponieważ była ona słabo widoczna, co mogło skutkować błędnymi odczytami.


Obliczona niepewność złożona długości fali jest duża, wynosi $u_c(\lambda) \approx 90 \text{ nm}$, co jest równe niepewności względnej $16\%$.


% ---------- WYKRESY ----------
\section{Wykresy}

\begin{figure}[H]
    \centering
    \includegraphics[width=1.3\textwidth,height=0.9\textheight,keepaspectratio,angle=90]{../images/helium_dispersion.png}
    \caption{Wykres krzywej dyspersji dla widma helu.}
    \label{fig:helium_dispersion}
\end{figure}

% \newpage

\bibliographystyle{apalike}
\bibliography{bibliography}

\end{document}