\documentclass[a4paper,12pt]{article}
\usepackage[left=2cm,right=2cm,top=2cm,bottom=2cm]{geometry} % Do ustawień marginesów
\usepackage{multicol} % Dla podziału na kolumny
\usepackage{ragged2e} % Dla justowania tekstu
\usepackage{graphicx} % Required for inserting images
\usepackage{float}
\usepackage{caption}
\usepackage{amsmath} % Math formulas
\usepackage{amssymb} % Symbols
\usepackage[svgnames]{xcolor}
\usepackage[colorlinks=true, urlcolor=blue, linkcolor=black, citecolor=orange]{hyperref} % Hyperlinks
\usepackage{polski} % Polish language
\usepackage[utf8]{inputenc} % Text encoding
\usepackage{enumitem} % Pakiet do elastycznego sterowania listami
\usepackage{indentfirst}
\usepackage{array}
\usepackage{longtable}
\setlist[itemize]{itemsep=0pt, topsep=0pt}

\begin{document}

% Górna część strony
\noindent
\begin{minipage}{0.5\textwidth}
    \raggedright
    \textbf{Piotr Durniat} \\
    II rok, Fizyka \\
    Wtorek, 8:00-10:15 \\
    \vspace{0.5cm}
    \vspace{0.5cm}
\end{minipage}%
\begin{minipage}{0.5\textwidth}
    \raggedleft
    Data wykonania pomiarów: \\
    14.10.2025 \\
    \vspace{0.5cm}
    Prowadząca: \\
    dr Sylwia Owczarek
\end{minipage}

% Tytuł ćwiczenia
\vspace{2cm}
\begin{center}
    \LARGE \textbf{Ćwiczenie nr 66} \\[0.5cm]
    \Large \textbf{Analiza widmowa za pomocą spektroskopu}
\end{center}

% Reszta treści
\vspace{1cm} % Kolejny odstęp
\noindent

% \tableofcontents
% \newpage

% ---------- WSTĘP TEORETYCZNY ----------
\section{Wstęp teoretyczny}

\subsection*{Emisja światła}

Emisja światła to proces emisji fotonów przez atomy. Zachodzi, gdy atomy przechodzą ze stanu wzbudzonego do stanu podstawowego, emitując energię w postaci światła.~\cite{Drynski1976}

\subsection*{Budowa atomu}

Atom składa się z jądra atomowego, zawierającego protony i neutrony, oraz elektronów na określonych poziomach energetycznych.~\cite{Drynski1976}

\subsection*{Rodzaje widm}

Widmo to obraz promieniowania rozłożonego na poszczególne długości fali. Główne rodzaje to:

\begin{itemize}
    \item Widmo emisyjne: Powstaje, gdy świecący gaz emituje światło. Składa się z kolorowych linii na ciemnym tle.

    \item Widmo absorpcyjne: Powstaje, gdy światło przechodzi przez gaz, widoczne jest w postaci czarnych linii na ciągłym spektrum.~\cite{fizyka_dla_szkół_wyższych_tom_3}
\end{itemize}


\subsection*{Serie widmowe}

To uporządkowane grupy linii w widmie atomu, które odpowiadają przejściom elektronów z różnych wyższych poziomów energetycznych na ten sam, określony niższy poziom (lub odwrotnie). Przykładowo seria Balmera to grupa linii w widmie odpowiadających przejściom elektronów ze lub do stanu $n=2$ atomu wodoru. Opisuje ją wzór Balmera. ~\cite{fizyka_dla_szkół_wyższych_tom_3}

\subsection*{Budowa i zasada działania spektroskopu}

Spektroskop to urządzenie do analizy widm świetlnych. Składa się z:

\begin{itemize}
    \item Kolimatora (K): Tworzy z rozbieżnej wiązki światła wiązkę równoległą.
    
    \item Pryzmatu (P): Rozszczepia światło na skutek zjawiska dyspersji.
    
    \item Lunety (L): Umożliwia obserwację powstałego widma.
\end{itemize}

\subsection*{Rozszczepienie światła przez pryzmat (dyspersja)}

Dyspersja to zjawisko zależności współczynnika załamania światła od jego częstotliwości (lub długości fali) w danym ośrodku. Prowadzi to do rozszczepienia światła białego na barwy składowe podczas przejścia np. przez pryzmat.~\cite{Drynski1976}


\subsection*{Analiza widmowa za pomocą spektroskopu}

Substancję identyfikuje się poprzez porównanie jej unikalnego widma z widmami wzorcowymi.

% ---------- OPIS DOŚWIADCZENIA ----------
% \section{Opis doświadczenia}

% ---------- OPRACOWANIE WYNIKÓW POMIARÓW ----------
% \section{Opracowanie wyników pomiarów}

% ---------- TABELE ----------
% \subsection{Tabele pomiarowe}

% ---------- OBLICZENIA ----------
% \subsection{...}

% ---------- NIEPEWNOŚCI ----------
% \section{Ocena niepewności pomiaru}

% ---------- WNIOSKI ----------
% \section{Wnioski}

% ---------- WYKRESY ----------
% \section{Wykresy}

\bibliographystyle{plain}
\bibliography{bibliography}

\end{document}
