\documentclass[a4paper,12pt]{article}
\usepackage[left=2cm,right=2cm,top=2cm,bottom=2cm]{geometry} 
\usepackage{multicol} 
\usepackage{ragged2e} 
\usepackage{graphicx} 
\usepackage{float}
\usepackage{caption}
\usepackage{amsmath} 
\usepackage{amssymb} 
\usepackage[svgnames]{xcolor}
\usepackage[colorlinks=true, urlcolor=blue, linkcolor=black, citecolor=orange]{hyperref} 
\usepackage{polski} 
\usepackage[utf8]{inputenc} 
\usepackage{enumitem} 
\usepackage{indentfirst}
\usepackage{array}
\usepackage{longtable}
\usepackage{pdflscape}
\usepackage[round]{natbib}
\setlist[itemize]{itemsep=0pt, topsep=0pt}
\usepackage{siunitx}

% SI setup preferences
\sisetup{output-decimal-marker={,}}
\sisetup{exponent-product = \cdot}
\sisetup{per-mode = symbol}

% LTeX: language=pl-PL

\begin{document}

\noindent
\begin{minipage}{0.5\textwidth}
	\raggedright
	% full name, index number
	\textbf{} \\
	II rok, Fizyka \\
	Wtorek, 8:00-10:15
	\vspace{0.5cm}
\end{minipage}
\begin{minipage}{0.5\textwidth}
	\raggedleft
	% date of the experiments
	29.04.2025 \\
	\vspace{0.5cm}
	Prowadząca: \\
	dr Sylwia Owczarek
\end{minipage}

\vspace{2cm}
\begin{center}
	% lab number
	\LARGE \textbf{Ćwiczenie nr 00} \\[0.5cm]
	% title
	\Large \textbf{Template}
\end{center}

\vspace{1cm}
\noindent

% \tableofcontents
% \newpage

% ---------- WSTĘP TEORETYCZNY ----------
\section{Wstęp teoretyczny}

% ---------- OPIS DOŚWIADCZENIA ----------
% \section{Opis doświadczenia}

% ---------- OPRACOWANIE WYNIKÓW POMIARÓW ----------
% \section{Opracowanie wyników pomiarów}

% ---------- TABELE ----------
% \subsection{Tabele pomiarowe}

% ---------- OBLICZENIA ----------
% \subsection{...}

% ---------- NIEPEWNOŚCI ----------
% \section{Ocena niepewności pomiaru}

% ---------- WNIOSKI ----------
% \section{Wnioski}

% ---------- WYKRESY ----------
% \section{Wykresy}
% \newpage

\bibliographystyle{apalike}
\bibliography{bibliography}

\end{document}
