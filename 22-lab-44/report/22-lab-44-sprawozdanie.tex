\documentclass[a4paper,12pt]{article}
\usepackage[left=2cm,right=2cm,top=2cm,bottom=2cm]{geometry}
\usepackage{multicol}
\usepackage{ragged2e}
\usepackage{graphicx}
\usepackage{float}
\usepackage{caption}
\usepackage{amsmath}
\usepackage{amssymb}
\usepackage[svgnames]{xcolor}
\usepackage[colorlinks=true, urlcolor=blue, linkcolor=black, citecolor=orange]{hyperref}
\usepackage{polski}
\usepackage[utf8]{inputenc}
\usepackage{enumitem}
\usepackage{indentfirst}
\usepackage{array}
\usepackage{longtable}
\usepackage{pdflscape}
\usepackage[round]{natbib}
\setlist[itemize]{itemsep=0pt, topsep=0pt}
\usepackage{siunitx}

% SI setup preferences
\sisetup{output-decimal-marker={,}}
\sisetup{exponent-product = \cdot}
\sisetup{per-mode = symbol}

% LTeX: language=pl-PL

\begin{document}

\noindent
\begin{minipage}{0.5\textwidth}
	\raggedright
	% full name and index number
	\textbf{Imię Nazwisko, Nr albumu} \\
	II rok, Fizyka \\
	Wtorek, 8:00-10:15
	\vspace{0.5cm}
\end{minipage}
\begin{minipage}{0.5\textwidth}
	\raggedleft
	% date
	DATA \\
	\vspace{0.5cm}
	Prowadząca: \\
	dr Sylwia Owczarek
\end{minipage}

% Tytuł ćwiczenia
\vspace{2cm}
\begin{center}
	\LARGE \textbf{Ćwiczenie nr 44} \\[0.5cm]
	\Large \textbf{Prawo Ohma dla prądu przemiennego}
\end{center}

\vspace{1cm}
\noindent

% ---------- WSTĘP TEORETYCZNY ----------
\section{Wstęp teoretyczny}

\subsection{Charakterystyka prądu przemiennego}
Prądem przemiennym nazywamy prąd elektryczny, którego wartość chwilowa i kierunek ulegają okresowym zmianom. W najprostszym przypadku zmiany te mają charakter harmoniczny (sinusoidalny) i mogą być opisane równaniami:
\begin{equation}
	u(t) = U_0 \sin(\omega t), \quad i(t) = I_0 \sin(\omega t + \varphi)
\end{equation}
gdzie $U_0, I_0$ to amplitudy napięcia i natężenia, $\omega = 2\pi f$ to częstość kołowa, a $\varphi$ oznacza przesunięcie fazowe między napięciem a natężeniem.

W obwodach prądu przemiennego najczęściej posługujemy się wartościami skutecznymi napięcia i natężenia, które dla przebiegów sinusoidalnych wiążą się z amplitudami zależnością:
\begin{equation}
	U_{sk} = \frac{U_0}{\sqrt{2}}, \quad I_{sk} = \frac{I_0}{\sqrt{2}}
\end{equation}

\subsection{Impedancja i uogólnione prawo Ohma}
W obwodach prądu przemiennego opór elektryczny nie zależy wyłącznie od rezystancji (oporu czynnego), ale również od obecności elementów gromadzących energię pola elektrycznego i magnetycznego (kondensatory, cewki). Całkowity opór obwodu nazywamy impedancją (zawadą) $Z$. Uogólnione prawo Ohma przyjmuje postać:
\begin{equation}
	Z = \frac{U_{sk}}{I_{sk}}
\end{equation}
Impedancja składa się z części rzeczywistej (rezystancja $R$) oraz części urojonej (reaktancja $X$):
\begin{equation}
	Z = \sqrt{R^2 + X^2}
\end{equation}

\subsection{Elementy R, L, C w obwodzie prądu przemiennego}

\subsubsection{Rezystor idealny}
Dla rezystora idealnego impedancja jest równa jego rezystancji ($Z=R$). Napięcie i natężenie są zgodne w fazie ($\varphi = 0$).

\subsubsection{Cewka indukcyjna}
Cewka idealna o indukcyjności $L$ stawia prądowi przemiennemu opór bierny zwany reaktancją indukcyjną (induktancją) $X_L$:
\begin{equation}
	X_L = \omega L = 2\pi f L
\end{equation}
W cewce idealnej napięcie wyprzedza natężenie prądu o kąt $\pi/2$. W rzeczywistości cewka posiada również opór czynny uzwojenia $R_L$, dlatego jej impedancja wynosi:
\begin{equation}
	Z_L = \sqrt{R_L^2 + (\omega L)^2}
\end{equation}

\subsubsection{Kondensator}
Kondensator o pojemności $C$ charakteryzuje się reaktancją pojemnościową (kapacytancją) $X_C$:
\begin{equation}
	X_C = \frac{1}{\omega C} = \frac{1}{2\pi f C}
\end{equation}
Na kondensatorze idealnym natężenie prądu wyprzedza napięcie o kąt $\pi/2$. Prąd płynący przez idealny kondensator i cewkę nie wykonuje pracy (moc czynna wynosi zero).

\subsection{Szeregowy obwód RLC}
W przypadku szeregowego połączenia rezystora, cewki i kondensatora, całkowita impedancja obwodu wynosi:
\begin{equation}
	Z = \sqrt{R^2 + \left(X_L - X_C\right)^2} = \sqrt{R^2 + \left(\omega L - \frac{1}{\omega C}\right)^2}
\end{equation}
Przesunięcie fazowe $\varphi$ w takim obwodzie wyraża się wzorem:
\begin{equation}
	\tan \varphi = \frac{X_L - X_C}{R}
\end{equation}

% ---------- OPIS DOŚWIADCZENIA ----------
% \section{Opis doświadczenia}

% ---------- OPRACOWANIE WYNIKÓW POMIARÓW ----------
\section{Opracowanie wyników pomiarów}

% ---------- TABELE ----------
\subsection{Tabele pomiarowe}


\noindent
\begin{minipage}[t]{0.48\textwidth}
	\small % Zmniejszenie czcionki dla tabel w tej kolumnie
	\centering

	\begin{table}[H]
		\centering
		\caption{Uzwojenie L1}
		\begin{tabular}{|c|c|c|}
			\hline
			\textbf{lp} & \textbf{I [mA]} & \textbf{U [V]} \\ \hline
			1           & 10,0            & 0,50           \\ \hline
			2           & 20,0            & 1,00           \\ \hline
			3           & 29,9            & 1,50           \\ \hline
			4           & 39,7            & 2,00           \\ \hline
			5           & 49,7            & 2,50           \\ \hline
			6           & 59,5            & 3,00           \\ \hline
			7           & 69,1            & 3,50           \\ \hline
			8           & 79,3            & 4,00           \\ \hline
			9           & 89,3            & 4,50           \\ \hline
			10          & 99,0            & 5,00           \\ \hline
		\end{tabular}
	\end{table}

	\begin{table}[H]
		\centering
		\caption{Uzwojenie L1+L2}
		\begin{tabular}{|c|c|c|}
			\hline
			\textbf{lp} & \textbf{I [mA]} & \textbf{U [V]} \\ \hline
			1           & 3,7             & 0,50           \\ \hline
			2           & 7,5             & 1,00           \\ \hline
			3           & 11,2            & 1,50           \\ \hline
			4           & 14,9            & 2,00           \\ \hline
			5           & 18,8            & 2,51           \\ \hline
			6           & 22,4            & 3,00           \\ \hline
			7           & 26,1            & 3,50           \\ \hline
			8           & 29,9            & 3,99           \\ \hline
			9           & 33,7            & 4,50           \\ \hline
			10          & 37,4            & 5,00           \\ \hline
		\end{tabular}
	\end{table}

	\begin{table}[H]
		\centering
		\caption{Uzwojenie L1+L2+L3}
		\begin{tabular}{|c|c|c|}
			\hline
			\textbf{lp} & \textbf{I [mA]} & \textbf{U [V]} \\ \hline
			1           & 1,5             & 0,48           \\ \hline
			2           & 3,0             & 0,96           \\ \hline
			3           & 4,7             & 1,50           \\ \hline
			4           & 6,3             & 1,98           \\ \hline
			5           & 7,9             & 2,50           \\ \hline
			6           & 9,4             & 2,97           \\ \hline
			7           & 11,1            & 3,48           \\ \hline
			8           & 12,7            & 3,98           \\ \hline
			9           & 14,3            & 4,49           \\ \hline
			10          & 16,0            & 5,02           \\ \hline
		\end{tabular}
	\end{table}
\end{minipage}
\hfill % Wypełnienie przestrzeni między kolumnami
\begin{minipage}[t]{0.48\textwidth}
	\small % Zmniejszenie czcionki dla tabel w tej kolumnie
	\centering

	\begin{table}[H]
		\centering
		\caption{Kondensator C}
		\begin{tabular}{|c|c|c|}
			\hline
			\textbf{lp} & \textbf{I [mA]} & \textbf{U [V]} \\ \hline
			1           & 0,48            & 0,48           \\ \hline
			2           & 1,00            & 0,96           \\ \hline
			3           & 1,60            & 1,49           \\ \hline
			4           & 2,10            & 1,98           \\ \hline
			5           & 2,70            & 2,49           \\ \hline
			6           & 3,20            & 2,99           \\ \hline
			7           & 3,80            & 3,51           \\ \hline
			8           & 4,40            & 4,01           \\ \hline
			9           & 5,00            & 4,53           \\ \hline
			10          & 5,50            & 5,02           \\ \hline
		\end{tabular}
	\end{table}

	\begin{table}[H]
		\centering
		\caption{Kondensator 2C}
		\begin{tabular}{|c|c|c|}
			\hline
			\textbf{lp} & \textbf{I [mA]} & \textbf{U [V]} \\ \hline
			1           & 1,0             & 0,50           \\ \hline
			2           & 2,2             & 1,02           \\ \hline
			3           & 3,3             & 1,50           \\ \hline
			4           & 4,3             & 1,98           \\ \hline
			5           & 5,5             & 2,47           \\ \hline
			6           & 6,8             & 3,05           \\ \hline
			7           & 7,8             & 3,52           \\ \hline
			8           & 8,9             & 3,98           \\ \hline
			9           & 10,1            & 4,52           \\ \hline
			10          & 11,3            & 5,03           \\ \hline
		\end{tabular}
	\end{table}

	\begin{table}[H]
		\centering
		\caption{Kondensator 4C}
		\begin{tabular}{|c|c|c|}
			\hline
			\textbf{lp} & \textbf{I [mA]} & \textbf{U [V]} \\ \hline
			1           & 2,4             & 0,48           \\ \hline
			2           & 5,0             & 0,99           \\ \hline
			3           & 7,7             & 1,52           \\ \hline
			4           & 10,1            & 2,00           \\ \hline
			5           & 12,8            & 2,53           \\ \hline
			6           & 15,5            & 3,05           \\ \hline
			7           & 17,7            & 3,48           \\ \hline
			8           & 20,4            & 4,01           \\ \hline
			9           & 22,4            & 4,51           \\ \hline
			10          & 25,5            & 5,02           \\ \hline
		\end{tabular}
	\end{table}
\end{minipage}

% Tabela zbiorcza poza kolumnami (na całą szerokość)
\begin{table}[H]
	\centering
	\caption{Układ L1+L2+L3+C1}
	\begin{tabular}{|c|c|c|}
		\hline
		\textbf{lp} & \textbf{I [mA]} & \textbf{U [V]} \\ \hline
		1           & 0,8             & 0,51           \\ \hline
		2           & 1,6             & 0,98           \\ \hline
		3           & 2,4             & 1,49           \\ \hline
		4           & 3,3             & 2,02           \\ \hline
		5           & 4,1             & 2,49           \\ \hline
		6           & 4,9             & 2,98           \\ \hline
		7           & 5,7             & 3,47           \\ \hline
		8           & 6,6             & 4,00           \\ \hline
		9           & 7,5             & 4,53           \\ \hline
		10          & 8,3             & 5,02           \\ \hline
	\end{tabular}
\end{table}

\subsubsection*{ Dane ogólne}


Częstotliwość prądu przemiennego: $f = 50\ \mathrm{Hz}$.
\[
	\omega = 2\pi f = 2\pi\cdot 50 = 100\pi = 314{,}16\ \mathrm{rad/s}
\]

Rezystancje uzwojeń cewek:
\[
	R_{ab}=36\ \Omega,\quad
	R_{bc}=16\ \Omega,\quad
	R_{cd}=33\ \Omega
\]

\[
	\begin{aligned}
		R_1       & = 36\ \Omega            \\
		R_{1+2}   & = 36+16 = 52\ \Omega    \\
		R_{1+2+3} & = 36+16+33 = 85\ \Omega
	\end{aligned}
\]

% ---------- OBLICZENIA ----------
\subsection{Wyznaczenie indukcyjności cewki}

Na podstawie pomiarów napięcia i natężenia prądu dla cewek, narysowano ich charakterystyki prądowo napięciowe i zamieszczono na wykresie \ref{fig:charakterystyka_cewek}.



% ---------- NIEPEWNOŚCI ----------
% \section{Ocena niepewności pomiaru}

% ---------- WNIOSKI ----------
% \section{Wnioski}

% ---------- WYKRESY ----------
\section{Wykresy}

\newpage

% Wykres cewek na osobnej stronie
\begin{figure}[p]
	\centering
	% Obrót o 90 stopni, dopasowanie do wysokości strony tekstowej
	\includegraphics[angle=90, width=\textwidth, height=0.9\textheight, keepaspectratio]{img/charakterystyka_cewek.png}
	\caption{Charakterystyki prądowo-napięciowe cewek $L_1$, $L_2$ oraz $L_3$.}
	\label{fig:charakterystyka_cewek}
\end{figure}

\clearpage

% Wykres kondensatorów na kolejnej osobnej stronie
\begin{figure}[p]
	\centering
	\includegraphics[angle=90, width=\textwidth, height=0.9\textheight, keepaspectratio]{img/charakterystyka_kondensatorow.png}
	\caption{Charakterystyki prądowo-napięciowe kondensatorów $C_1$, $C_2$ oraz $C_3$.}
	\label{fig:charakterystyka_kondensatorow}
\end{figure}

% ---------- LITERATURA ----------
\bibliographystyle{apalike}
\bibliography{bibliography}

\end{document}
