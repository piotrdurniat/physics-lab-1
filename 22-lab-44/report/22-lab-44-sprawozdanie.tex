\documentclass[a4paper,12pt]{article}
\usepackage[left=2cm,right=2cm,top=2cm,bottom=2cm]{geometry}
\usepackage{multicol}
\usepackage{ragged2e}
\usepackage{graphicx}
\usepackage{float}
\usepackage{caption}
\usepackage{amsmath}
\usepackage{amssymb}
\usepackage[svgnames]{xcolor}
\usepackage[colorlinks=true, urlcolor=blue, linkcolor=black, citecolor=orange]{hyperref}
\usepackage{polski}
\usepackage[utf8]{inputenc}
\usepackage{enumitem}
\usepackage{indentfirst}
\usepackage{array}
\usepackage{longtable}
\usepackage{pdflscape}
\usepackage[round]{natbib}
\setlist[itemize]{itemsep=0pt, topsep=0pt}
\usepackage{siunitx}

% SI setup preferences
\sisetup{output-decimal-marker={,}}
\sisetup{exponent-product = \cdot}
\sisetup{per-mode = symbol}

% LTeX: language=pl-PL

\begin{document}

\noindent
\begin{minipage}{0.5\textwidth}
	\raggedright
	% full name and index number
	\textbf{Imię Nazwisko, Nr albumu} \\
	II rok, Fizyka \\
	Wtorek, 8:00-10:15
	\vspace{0.5cm}
\end{minipage}
\begin{minipage}{0.5\textwidth}
	\raggedleft
	% date
	DATA \\
	\vspace{0.5cm}
	Prowadząca: \\
	dr Sylwia Owczarek
\end{minipage}

% Tytuł ćwiczenia
\vspace{2cm}
\begin{center}
	\LARGE \textbf{Ćwiczenie nr 44} \\[0.5cm]
	\Large \textbf{Prawo Ohma dla prądu przemiennego}
\end{center}

\vspace{1cm}
\noindent

% ---------- WSTĘP TEORETYCZNY ----------
\section{Wstęp teoretyczny}

\subsection{Charakterystyka prądu przemiennego}
Prądem przemiennym nazywamy prąd elektryczny, którego wartość chwilowa i kierunek ulegają okresowym zmianom. W najprostszym przypadku zmiany te mają charakter harmoniczny (sinusoidalny) i mogą być opisane równaniami:
\begin{equation}
	u(t) = U_0 \sin(\omega t), \quad i(t) = I_0 \sin(\omega t + \varphi)
\end{equation}
gdzie $U_0, I_0$ to amplitudy napięcia i natężenia, $\omega = 2\pi f$ to częstość kołowa, a $\varphi$ oznacza przesunięcie fazowe między napięciem a natężeniem.

W obwodach prądu przemiennego najczęściej posługujemy się wartościami skutecznymi napięcia i natężenia, które dla przebiegów sinusoidalnych wiążą się z amplitudami zależnością:
\begin{equation}
	U_{sk} = \frac{U_0}{\sqrt{2}}, \quad I_{sk} = \frac{I_0}{\sqrt{2}}
\end{equation}

\subsection{Impedancja i zawada}
W obwodach prądu przemiennego opór całkowity nazywamy **impedancją**. Jest to wielkość zespolona, składająca się z części rzeczywistej (rezystancja $R$) oraz części urojonej (reaktancja $X$).
Moduł impedancji nazywamy **zawadą** i oznaczamy symbolem $|Z|$.
\begin{equation}
	|Z| = \sqrt{R^2 + X^2}
\end{equation}
Uogólnione prawo Ohma dla wartości skutecznych przyjmuje postać:
\begin{equation}
	I_{sk} = \frac{U_{sk}}{|Z|}
\end{equation}

\subsection{Elementy R, L, C w obwodzie prądu przemiennego}

\subsubsection{Rezystor idealny}
Dla rezystora idealnego reaktancja wynosi zero, zatem zawada jest równa jego rezystancji ($|Z|=R$). Napięcie i natężenie są zgodne w fazie ($\varphi = 0$).

\subsubsection{Cewka indukcyjna}
Cewka idealna o indukcyjności $L$ stawia prądowi przemiennemu opór bierny zwany reaktancją indukcyjną (induktancją) $X_L$:
\begin{equation}
	X_L = \omega L = 2\pi f L
\end{equation}
W rzeczywistości cewka posiada również opór czynny uzwojenia $R_L$. Jej zawada wynosi zatem:
\begin{equation}
	|Z_L| = \sqrt{R_L^2 + X_L^2} = \sqrt{R_L^2 + (\omega L)^2}
\end{equation}
W cewce idealnej napięcie wyprzedza natężenie prądu o kąt $\pi/2$.

\subsubsection{Kondensator}
Kondensator o pojemności $C$ charakteryzuje się reaktancją pojemnościową (kapacytancją) $X_C$:
\begin{equation}
	X_C = \frac{1}{\omega C} = \frac{1}{2\pi f C}
\end{equation}
Dla kondensatora idealnego (gdzie $R=0$) zawada wynosi $|Z_C| = X_C$. Natężenie prądu wyprzedza napięcie o kąt $\pi/2$.

\subsection{Szeregowy obwód RLC}
W przypadku szeregowego połączenia rezystora, cewki i kondensatora, wypadkowa zawada obwodu wynosi:
\begin{equation}
	|Z| = \sqrt{R^2 + \left(X_L - X_C\right)^2} = \sqrt{R^2 + \left(\omega L - \frac{1}{\omega C}\right)^2}
\end{equation}
Przesunięcie fazowe $\varphi$ w takim obwodzie wyraża się wzorem:
\begin{equation}
	\tan \varphi = \frac{X_L - X_C}{R}
\end{equation}
% ---------- OPIS DOŚWIADCZENIA ----------
% \section{Opis doświadczenia}

% ---------- OPRACOWANIE WYNIKÓW POMIARÓW ----------
\section{Opracowanie wyników pomiarów}

% ---------- TABELE ----------
\subsection{Tabele pomiarowe}


\noindent
\begin{minipage}[t]{0.48\textwidth}
	\small % Zmniejszenie czcionki dla tabel w tej kolumnie
	\centering

	\begin{table}[H]
		\centering
		\caption{Uzwojenie L1}
		\begin{tabular}{|c|c|c|}
			\hline
			\textbf{lp} & \textbf{I [mA]} & \textbf{U [V]} \\ \hline
			1           & 10,0            & 0,50           \\ \hline
			2           & 20,0            & 1,00           \\ \hline
			3           & 29,9            & 1,50           \\ \hline
			4           & 39,7            & 2,00           \\ \hline
			5           & 49,7            & 2,50           \\ \hline
			6           & 59,5            & 3,00           \\ \hline
			7           & 69,1            & 3,50           \\ \hline
			8           & 79,3            & 4,00           \\ \hline
			9           & 89,3            & 4,50           \\ \hline
			10          & 99,0            & 5,00           \\ \hline
		\end{tabular}
	\end{table}

	\begin{table}[H]
		\centering
		\caption{Uzwojenie L1+L2}
		\begin{tabular}{|c|c|c|}
			\hline
			\textbf{lp} & \textbf{I [mA]} & \textbf{U [V]} \\ \hline
			1           & 3,7             & 0,50           \\ \hline
			2           & 7,5             & 1,00           \\ \hline
			3           & 11,2            & 1,50           \\ \hline
			4           & 14,9            & 2,00           \\ \hline
			5           & 18,8            & 2,51           \\ \hline
			6           & 22,4            & 3,00           \\ \hline
			7           & 26,1            & 3,50           \\ \hline
			8           & 29,9            & 3,99           \\ \hline
			9           & 33,7            & 4,50           \\ \hline
			10          & 37,4            & 5,00           \\ \hline
		\end{tabular}
	\end{table}

	\begin{table}[H]
		\centering
		\caption{Uzwojenie L1+L2+L3}
		\begin{tabular}{|c|c|c|}
			\hline
			\textbf{lp} & \textbf{I [mA]} & \textbf{U [V]} \\ \hline
			1           & 1,5             & 0,48           \\ \hline
			2           & 3,0             & 0,96           \\ \hline
			3           & 4,7             & 1,50           \\ \hline
			4           & 6,3             & 1,98           \\ \hline
			5           & 7,9             & 2,50           \\ \hline
			6           & 9,4             & 2,97           \\ \hline
			7           & 11,1            & 3,48           \\ \hline
			8           & 12,7            & 3,98           \\ \hline
			9           & 14,3            & 4,49           \\ \hline
			10          & 16,0            & 5,02           \\ \hline
		\end{tabular}
	\end{table}
\end{minipage}
\hfill % Wypełnienie przestrzeni między kolumnami
\begin{minipage}[t]{0.48\textwidth}
	\small % Zmniejszenie czcionki dla tabel w tej kolumnie
	\centering

	\begin{table}[H]
		\centering
		\caption{Kondensator C}
		\begin{tabular}{|c|c|c|}
			\hline
			\textbf{lp} & \textbf{I [mA]} & \textbf{U [V]} \\ \hline
			1           & 0,48            & 0,48           \\ \hline
			2           & 1,00            & 0,96           \\ \hline
			3           & 1,60            & 1,49           \\ \hline
			4           & 2,10            & 1,98           \\ \hline
			5           & 2,70            & 2,49           \\ \hline
			6           & 3,20            & 2,99           \\ \hline
			7           & 3,80            & 3,51           \\ \hline
			8           & 4,40            & 4,01           \\ \hline
			9           & 5,00            & 4,53           \\ \hline
			10          & 5,50            & 5,02           \\ \hline
		\end{tabular}
	\end{table}

	\begin{table}[H]
		\centering
		\caption{Kondensator 2C}
		\begin{tabular}{|c|c|c|}
			\hline
			\textbf{lp} & \textbf{I [mA]} & \textbf{U [V]} \\ \hline
			1           & 1,0             & 0,50           \\ \hline
			2           & 2,2             & 1,02           \\ \hline
			3           & 3,3             & 1,50           \\ \hline
			4           & 4,3             & 1,98           \\ \hline
			5           & 5,5             & 2,47           \\ \hline
			6           & 6,8             & 3,05           \\ \hline
			7           & 7,8             & 3,52           \\ \hline
			8           & 8,9             & 3,98           \\ \hline
			9           & 10,1            & 4,52           \\ \hline
			10          & 11,3            & 5,03           \\ \hline
		\end{tabular}
	\end{table}

	\begin{table}[H]
		\centering
		\caption{Kondensator 4C}
		\begin{tabular}{|c|c|c|}
			\hline
			\textbf{lp} & \textbf{I [mA]} & \textbf{U [V]} \\ \hline
			1           & 2,4             & 0,48           \\ \hline
			2           & 5,0             & 0,99           \\ \hline
			3           & 7,7             & 1,52           \\ \hline
			4           & 10,1            & 2,00           \\ \hline
			5           & 12,8            & 2,53           \\ \hline
			6           & 15,5            & 3,05           \\ \hline
			7           & 17,7            & 3,48           \\ \hline
			8           & 20,4            & 4,01           \\ \hline
			9           & 22,4            & 4,51           \\ \hline
			10          & 25,5            & 5,02           \\ \hline
		\end{tabular}
	\end{table}
\end{minipage}

% Tabela zbiorcza poza kolumnami (na całą szerokość)
\begin{table}[H]
	\centering
	\caption{Układ L1+L2+L3+C1}
	\begin{tabular}{|c|c|c|}
		\hline
		\textbf{lp} & \textbf{I [mA]} & \textbf{U [V]} \\ \hline
		1           & 0,8             & 0,51           \\ \hline
		2           & 1,6             & 0,98           \\ \hline
		3           & 2,4             & 1,49           \\ \hline
		4           & 3,3             & 2,02           \\ \hline
		5           & 4,1             & 2,49           \\ \hline
		6           & 4,9             & 2,98           \\ \hline
		7           & 5,7             & 3,47           \\ \hline
		8           & 6,6             & 4,00           \\ \hline
		9           & 7,5             & 4,53           \\ \hline
		10          & 8,3             & 5,02           \\ \hline
	\end{tabular}
\end{table}

\subsubsection*{ Dane ogólne}

Częstotliwość prądu przemiennego wynosi: $f = 50\ \mathrm{Hz}$, stąd częstość kołowa ($\omega$), wynosi:
\[
	\omega = 2\pi f = 2\pi\cdot 50 = 100\pi = 314{,}16\ \mathrm{rad/s}
\]

Rezystancje uzwojeń cewek wynoszą:
\[
	R_{ab}=36\ \Omega,\quad
	R_{bc}=16\ \Omega,\quad
	R_{cd}=33\ \Omega
\]

Stąd rezystancje układów cewek wynoszą:

\[
	\begin{aligned}
		R_1       & = 36\ \Omega            \\
		R_{1+2}   & = 36+16 = 52\ \Omega    \\
		R_{1+2+3} & = 36+16+33 = 85\ \Omega
	\end{aligned}
\]

% ---------- OBLICZENIA ----------
\subsection{Wyznaczenie indukcyjności cewki}

Na podstawie bezpośrednich pomiarów dla cewki $L_1$ oraz wyznaczonych teoretycznie impedancji cewek $L_2$ i $L_3$, sporządzono ich charakterystyki prądowo-napięciowe (rys. \ref{fig:charakterystyka_cewek}).

Do wyznaczenia parametrów tych charakterystyk wykorzystano metodę regresji liniowej. Przyjęto model liniowy $U(I) = a \cdot I + b$, gdzie współczynnik kierunkowy $a$ odpowiada wartości zawady $|Z|$ danego układu, a wyraz wolny $b$ powinien być bliski zeru.

Niepewność standardową współczynnika kierunkowego $u(a)$ wyznaczono zgodnie ze wzorem (6) z instrukcji \citep{ONP}:
$$
	u(a) = s_y \sqrt{\frac{n}{D}}
$$
Wyniki dopasowania prostych do danych pomiarowych dla narastających układów uzwojeń przedstawiono w tabeli \ref{tab:regresja_cewki}.

\begin{table}[H]
	\centering
	\begin{tabular}{|c|c|c|c|c|}
		\hline
		\textbf{Układ}      & \textbf{$a$ ($|Z|$)} [$\Omega$] & \textbf{$u(a)$} [$\Omega$] & \textbf{$b$ [V]} & \textbf{$R^2$} \\ \hline
		$L_1$ (a-b)         & 50,58                           & 0,07                       & -0,0091          & 1,0000         \\ \hline
		$L_1+L_2$ (a-c)     & 133,53                          & 0,18                       & 0,0047           & 1,0000         \\ \hline
		$L_1+L_2+L_3$ (a-d) & 312,39                          & 0,65                       & 0,0214           & 1,0000         \\ \hline
	\end{tabular}
	\caption{Współczynniki regresji liniowej wyznaczone dla charakterystyk cewek ($U = aI + b$).}
	\label{tab:regresja_cewki}
\end{table}

\subsubsection{Obliczenie indukcyjności całkowitych ($L_{ukl}$)}

Dla każdego badanego układu, znając rezystancję czynną $R$ (sumę rezystancji włączonych sekcji) oraz częstość kołową $\omega = 314,16\,\mathrm{rad/s}$, obliczono całkowitą indukcyjność $L_{ukl}$ przekształcając wzór na impedancję:
\begin{equation}
	L_{ukl} = \frac{1}{\omega}\sqrt{|Z|^2 - R^2}
\end{equation}
Niepewność $u(L_{ukl})$ obliczono metodą przenoszenia niepewności (wzór 15 w \citep{ONP}), traktując $|Z|$ jako zmienną pomiarową:

\begin{equation}
	u(L_{ukl}) = \left| \frac{\partial L}{\partial |Z|} \right| u(|Z|) = \left| \frac{1}{\omega} \frac{2|Z|}{2\sqrt{|Z|^2 - R^2}} \right| u(|Z|) = \frac{|Z|}{\omega (\omega L_{ukl})} u(|Z|) = \frac{|Z|}{\omega^2 L_{ukl}} u(|Z|)
\end{equation}

\vspace{0.3cm}
\noindent \textbf{Przykładowe obliczenie} (dla układu $L_1+L_2$, czyli zaciski a-c):
\begin{itemize}
	\item Dane: $|Z| = 133,53\,\Omega$, $u(|Z|) = 0,18\,\Omega$, $R = R_{ab} + R_{bc} = 36,0 + 16,0 = 52,0\,\Omega$.
	\item Obliczenie wartości $L_{(a-c)}$:
	      $$
		      L_{(a-c)} = \frac{1}{314,16}\sqrt{133,53^2 - 52,0^2} = \frac{\sqrt{17830,26 - 2704}}{314,16} \approx \frac{122,99}{314,16} \approx 0,3915\,\mathrm{H}
	      $$
	\item Obliczenie niepewności $u(L_{(a-c)})$:
	      $$
		      u(L_{(a-c)}) = \frac{133,53}{314,16^2 \cdot 0,3915} \cdot 0,18 \approx \frac{133,53}{38651,5} \cdot 0,18 \approx 0,0006\,\mathrm{H}
	      $$
\end{itemize}
Analogicznie obliczono $L_{(a-b)} \approx 0,1131\,\mathrm{H}$ oraz $L_{(a-d)} \approx 0,9568\,\mathrm{H}$.

\subsubsection{Obliczenie indukcyjności sekcji ($L_1, L_2, L_3$)}

Ponieważ pomiary wykonywano w układzie narastającym (szeregowym), indukcyjności poszczególnych sekcji wyznaczono różnicowo:
\begin{itemize}
	\item $L_1 = L_{(a-b)}$
	\item $L_2 = L_{(a-c)} - L_{(a-b)}$
	\item $L_3 = L_{(a-d)} - L_{(a-c)}$
\end{itemize}

Niepewność wyznaczenia indukcyjności sekcji (dla $L_2$ i $L_3$) obliczono jako pierwiastek z sumy kwadratów niepewności składowych (prawo przenoszenia niepewności dla różnicy):
$$
	u(L_{sekcja}) = \sqrt{u^2(L_{obecny}) + u^2(L_{poprzedni})}
$$

\vspace{0.3cm}
\noindent \textbf{Przykładowe obliczenie} (dla sekcji $L_2$):
\begin{itemize}
	\item Dane (z poprzedniego kroku): $L_{(a-c)} = 0,3915\,\mathrm{H}$, $L_{(a-b)} = 0,1131\,\mathrm{H}$.
	\item Wynik:
	      $$L_2 = 0,3915 - 0,1131 = 0,2784\,\mathrm{H}$$
	\item Niepewność:
	      $$u(L_2) = \sqrt{0,0006^2 + 0,0003^2} \approx 0,0007\,\mathrm{H}$$
\end{itemize}

Ostateczne wyniki dla poszczególnych sekcji zestawiono w tabeli \ref{tab:wyniki_sekcje}.

\begin{table}[H]
	\centering
	\begin{tabular}{|c|c|c|}
		\hline
		\textbf{Sekcja}     & \textbf{Indukcyjność $L$ [H]} & \textbf{Niepewność $u(L)$ [H]} \\ \hline
		$L_1$               & 0,1131                        & 0,0003                         \\ \hline
		$L_2$               & 0,2784                        & 0,0007                         \\ \hline
		$L_3$               & 0,5654                        & 0,0022                         \\ \hline \hline
		\textbf{Suma (a-d)} & \textbf{0,9568}               & -                              \\ \hline
	\end{tabular}
	\caption{Wyznaczone współczynniki samoindukcji poszczególnych sekcji cewki.}
	\label{tab:wyniki_sekcje}
\end{table}


\subsection{Wyznaczenie pojemności kondensatora}

Podobnie jak w przypadku cewek, dla trzech kondensatorów ($C_1$, $C_2$ oraz $C_3$) sporządzono charakterystyki prądowo-napięciowe (rys. \ref{fig:charakterystyka_kondensatorow}).
Przyjęto model liniowy $I(U) = a \cdot U$, gdzie współczynnik kierunkowy $a$ jest związany z pojemnością $C$ zależnością:
\begin{equation}
	a = \omega C \quad \Rightarrow \quad C = \frac{a}{\omega}
\end{equation}
gdzie $\omega = 314,16\,\mathrm{rad/s}$.

Niepewność standardową współczynnika kierunkowego $u(a)$ wyznaczono metodą regresji liniowej (zgodnie ze wzorem 6 z instrukcji ONP). Następnie, korzystając z prawa przenoszenia niepewności (wzór 15 z instrukcji ONP), wyznaczono niepewność złożoną pojemności $u(C)$.
Ponieważ $\omega$ przyjęto jako stałą bezbłędną, pochodna cząstkowa wynosi:
\begin{equation}
	u(C) = \left| \frac{\partial C}{\partial a} \right| u(a) = \left| \frac{1}{\omega} \right| u(a) = \frac{u(a)}{\omega}
\end{equation}

\vspace{0.3cm}
\noindent \textbf{Przykładowe obliczenie} (dla kondensatora $C_1$):
\begin{itemize}
	\item Dane z regresji: $a = 1,1107 \cdot 10^{-3}\,\mathrm{A/V}$, $u(a) = 6,22 \cdot 10^{-6}\,\mathrm{A/V}$.
	\item Obliczenie wartości $C_1$:
	      $$ C_1 = \frac{1,1107 \cdot 10^{-3}}{314,16} \approx 3,535 \cdot 10^{-6}\,\mathrm{F} = 3,54\,\mu\mathrm{F} $$
	\item Obliczenie niepewności $u(C_1)$:
	      $$ u(C_1) = \frac{6,22 \cdot 10^{-6}}{314,16} \approx 1,98 \cdot 10^{-8}\,\mathrm{F} \approx 0,02\,\mu\mathrm{F} $$
\end{itemize}

Wyniki dla wszystkich badanych elementów zestawiono w tabeli \ref{tab:wyniki_kondensatory}.

\begin{table}[H]
	\centering
	\caption{Wyznaczone parametry kondensatorów wraz z niepewnościami.}
	\label{tab:wyniki_kondensatory}
	\begin{tabular}{|c|c|c|c|c|}
		\hline
		\textbf{El.} & \textbf{$a$ [A/V]}     & \textbf{$u(a)$ [A/V]} & \textbf{$C$ [$\mu$F]} & \textbf{$u(C)$ [$\mu$F]} \\ \hline
		$C_1$        & $1,1107 \cdot 10^{-3}$ & $6,22 \cdot 10^{-6}$  & 3,54                  & 0,02                     \\ \hline
		$C_2$        & $2,2662 \cdot 10^{-3}$ & $7,49 \cdot 10^{-6}$  & 7,21                  & 0,02                     \\ \hline
		$C_3$        & $5,0502 \cdot 10^{-3}$ & $3,55 \cdot 10^{-5}$  & 16,08                 & 0,11                     \\ \hline
	\end{tabular}
\end{table}

\noindent
Analizując otrzymane wartości pojemności, wyznaczono ich stosunki względem pojemności podstawowej $C_1$:
\begin{itemize}
	\item $C_2 / C_1 \approx 2,04$ (oczekiwano wartości bliskiej 2, układ $2C$),
	\item $C_3 / C_1 \approx 4,55$ (oczekiwano wartości bliskiej 4, układ $4C$).
\end{itemize}
Wyniki potwierdzają, że badane elementy mają pojemności będące przybliżonymi wielokrotnościami pojemności $C_1$.

% ---------- WNIOSKI ----------
% \section{Wnioski}

% ---------- WYKRESY ----------
\section{Wykresy}

\newpage

\begin{figure}[H]
	\centering
	% Obrót o 90 stopni, dopasowanie do wysokości strony tekstowej
	\includegraphics[angle=90, width=\textwidth, height=0.9\textheight, keepaspectratio]{img/charakterystyka_cewek.png}
	\caption{Charakterystyki prądowo-napięciowe cewek $L_1$, $L_2$ oraz $L_3$.}
	\label{fig:charakterystyka_cewek}
\end{figure}

\newpage

\begin{figure}[H]
	\centering
	\includegraphics[angle=90, width=\textwidth, height=0.9\textheight, keepaspectratio]{img/charakterystyka_kondensatorow.png}
	\caption{Charakterystyki prądowo-napięciowe kondensatorów $C_1$, $C_2$ oraz $C_3$.}
	\label{fig:charakterystyka_kondensatorow}
\end{figure}

% ---------- LITERATURA ----------
\bibliographystyle{apalike}
\bibliography{bibliography}

\end{document}
