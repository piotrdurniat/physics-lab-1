\documentclass[a4paper,12pt]{article}
\usepackage[left=2cm,right=2cm,top=2cm,bottom=2cm]{geometry}
\usepackage{multicol}
\usepackage{ragged2e}
\usepackage{graphicx}
\usepackage{float}
\usepackage{caption}
\usepackage{amsmath}
\usepackage{amssymb}
\usepackage[svgnames]{xcolor}
\usepackage[colorlinks=true, urlcolor=blue, linkcolor=black, citecolor=orange]{hyperref}
\usepackage{polski}
\usepackage[utf8]{inputenc}
\usepackage{enumitem}
\usepackage{indentfirst}
\usepackage{array}
\usepackage{longtable}
\usepackage{pdflscape}
\usepackage[round]{natbib}
\setlist[itemize]{itemsep=0pt, topsep=0pt}
\usepackage{siunitx}

% SI setup preferences
\sisetup{output-decimal-marker={,}}
\sisetup{exponent-product = \cdot}
\sisetup{per-mode = symbol}

% LTeX: language=pl-PL

\begin{document}

% Górna część strony
\noindent
\begin{minipage}{0.5\textwidth}
	\raggedright
	\textbf{Imię Nazwisko, Nr albumu} \\ % Uzupełnij swoje dane
	II rok, Fizyka \\
	Wtorek, 8:00-10:15 % Sprawdź czy termin jest aktualny
	\vspace{0.5cm}
\end{minipage}
\begin{minipage}{0.5\textwidth}
	\raggedleft
	DATA \\ % Wpisz datę
	\vspace{0.5cm}
	Prowadząca: \\
	dr Sylwia Owczarek
\end{minipage}

% Tytuł ćwiczenia
\vspace{2cm}
\begin{center}
	\LARGE \textbf{Ćwiczenie nr 44} \\[0.5cm]
	\Large \textbf{Prawo Ohma dla prądu przemiennego}
\end{center}

\vspace{1cm}
\noindent

% ---------- WSTĘP TEORETYCZNY ----------
\section{Wstęp teoretyczny}

\subsection{Charakterystyka prądu przemiennego}
Prądem przemiennym nazywamy prąd elektryczny, którego wartość chwilowa i kierunek ulegają okresowym zmianom. W najprostszym przypadku zmiany te mają charakter harmoniczny (sinusoidalny) i mogą być opisane równaniami:
\begin{equation}
	u(t) = U_0 \sin(\omega t), \quad i(t) = I_0 \sin(\omega t + \varphi)
\end{equation}
gdzie $U_0, I_0$ to amplitudy napięcia i natężenia, $\omega = 2\pi f$ to częstość kołowa, a $\varphi$ oznacza przesunięcie fazowe między napięciem a natężeniem.

W obwodach prądu przemiennego najczęściej posługujemy się wartościami skutecznymi napięcia i natężenia, które dla przebiegów sinusoidalnych wiążą się z amplitudami zależnością:
\begin{equation}
	U_{sk} = \frac{U_0}{\sqrt{2}}, \quad I_{sk} = \frac{I_0}{\sqrt{2}}
\end{equation}

\subsection{Impedancja i uogólnione prawo Ohma}
W obwodach prądu przemiennego opór elektryczny nie zależy wyłącznie od rezystancji (oporu czynnego), ale również od obecności elementów gromadzących energię pola elektrycznego i magnetycznego (kondensatory, cewki). Całkowity opór obwodu nazywamy impedancją (zawadą) $Z$. Uogólnione prawo Ohma przyjmuje postać:
\begin{equation}
	Z = \frac{U_{sk}}{I_{sk}}
\end{equation}
Impedancja składa się z części rzeczywistej (rezystancja $R$) oraz części urojonej (reaktancja $X$):
\begin{equation}
	Z = \sqrt{R^2 + X^2}
\end{equation}

\subsection{Elementy R, L, C w obwodzie prądu przemiennego}

\subsubsection{Rezystor idealny}
Dla rezystora idealnego impedancja jest równa jego rezystancji ($Z=R$). Napięcie i natężenie są zgodne w fazie ($\varphi = 0$).

\subsubsection{Cewka indukcyjna}
Cewka idealna o indukcyjności $L$ stawia prądowi przemiennemu opór bierny zwany reaktancją indukcyjną (induktancją) $X_L$:
\begin{equation}
	X_L = \omega L = 2\pi f L
\end{equation}
W cewce idealnej napięcie wyprzedza natężenie prądu o kąt $\pi/2$. W rzeczywistości cewka posiada również opór czynny uzwojenia $R_L$, dlatego jej impedancja wynosi:
\begin{equation}
	Z_L = \sqrt{R_L^2 + (\omega L)^2}
\end{equation}

\subsubsection{Kondensator}
Kondensator o pojemności $C$ charakteryzuje się reaktancją pojemnościową (kapacytancją) $X_C$:
\begin{equation}
	X_C = \frac{1}{\omega C} = \frac{1}{2\pi f C}
\end{equation}
Na kondensatorze idealnym natężenie prądu wyprzedza napięcie o kąt $\pi/2$. Prąd płynący przez idealny kondensator i cewkę nie wykonuje pracy (moc czynna wynosi zero).

\subsection{Szeregowy obwód RLC}
W przypadku szeregowego połączenia rezystora, cewki i kondensatora, całkowita impedancja obwodu wynosi:
\begin{equation}
	Z = \sqrt{R^2 + \left(X_L - X_C\right)^2} = \sqrt{R^2 + \left(\omega L - \frac{1}{\omega C}\right)^2}
\end{equation}
Przesunięcie fazowe $\varphi$ w takim obwodzie wyraża się wzorem:
\begin{equation}
	\tan \varphi = \frac{X_L - X_C}{R}
\end{equation}

% ---------- OPIS DOŚWIADCZENIA ----------
% \section{Opis doświadczenia}

% ---------- LITERATURA ----------
\bibliographystyle{apalike}
\bibliography{bibliography}

\end{document}
