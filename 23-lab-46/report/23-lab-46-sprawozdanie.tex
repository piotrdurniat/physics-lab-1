\documentclass[a4paper,12pt]{article}
\usepackage[left=2cm,right=2cm,top=2cm,bottom=2cm]{geometry} 
\usepackage{multicol} 
\usepackage{ragged2e} 
\usepackage{graphicx} 
\usepackage{float}
\usepackage{caption}
\usepackage{amsmath} 
\usepackage{amssymb} 
\usepackage[svgnames]{xcolor}
\usepackage[colorlinks=true, urlcolor=blue, linkcolor=black, citecolor=orange]{hyperref} 
\usepackage{polski} 
\usepackage[utf8]{inputenc} 
\usepackage{enumitem} 
\usepackage{indentfirst}
\usepackage{array}
\usepackage{longtable}
\usepackage{pdflscape}
\usepackage[round]{natbib}
\setlist[itemize]{itemsep=0pt, topsep=0pt}
\usepackage{siunitx}

% SI setup preferences
\sisetup{output-decimal-marker={,}}
\sisetup{exponent-product = \cdot}
\sisetup{per-mode = symbol}

% LTeX: language=pl-PL
\begin{document}

% --- NAGŁÓWEK ---
\noindent
\begin{minipage}{0.5\textwidth}
	\raggedright
	\textbf{Piotr Durniat, 347264} \\
	II rok, Fizyka \\
	Wtorek, 8:00-10:15
	\vspace{0.5cm}
\end{minipage}
\begin{minipage}{0.5\textwidth}
	\raggedleft
	2026-01-13 \\
	\vspace{0.5cm}
	Prowadząca: \\
	dr Sylwia Owczarek
\end{minipage}

\vspace{2cm}
\begin{center}
	\LARGE \textbf{Ćwiczenie nr 46} \\[0.5cm]
	\Large \textbf{Prawa Ohma i Kirchhoffa}
\end{center}

\vspace{1cm}

% --- TREŚĆ ---
\section{Wstęp teoretyczny}

\subsection{Prawo Ohma i rezystancja}
Natężenie prądu elektrycznego $I$ definiuje się jako ilość ładunku przepływającego przez przekrój poprzeczny przewodnika w jednostce czasu ($I = dQ/dt$).  Dla wielu materiałów (zwanych omowymi) gęstość prądu $\vec{J}$ jest wprost proporcjonalna do natężenia pola elektrycznego $\vec{E}$. W ujęciu makroskopowym zależność ta, znana jako prawo Ohma, wiąże natężenie prądu $I$ płynącego przez element z napięciem $U$ przyłożonym do jego końców \citep{fizyka_dla_szkol_wyzszych_tom_3}:
\begin{equation}
	I = \frac{U}{R}
\end{equation}
Współczynnik proporcjonalności $R$ nazywamy rezystancją (oporem elektrycznym). Jednostką rezystancji w układzie SI jest om ($\Omega$). Rezystancja elementu zależy od jego geometrii oraz rezystywności materiału $\rho$:
\begin{equation}
	R = \rho \frac{l}{S}
\end{equation}
gdzie $l$ to długość przewodnika, a $S$ to pole jego przekroju poprzecznego \citep{fizyka_dla_szkol_wyzszych_tom_3}.

\subsection{Prawa Kirchhoffa}
Analiza złożonych obwodów elektrycznych opiera się na dwóch fundamentalnych zasadach wynikających z praw zachowania:

\begin{enumerate}
	\item \textbf{I Prawo Kirchhoffa (Węzłowe):} Wynika z zasady zachowania ładunku. Suma algebraiczna natężeń prądów wpływających do węzła i wypływających z niego jest równa zeru \citep{fizyka_dla_szkol_wyzszych_tom_3}:
	      \begin{equation}
		      \sum_{k} I_k = 0
	      \end{equation}

	\item \textbf{II Prawo Kirchhoffa (Oczkowe):} Wynika z zasady zachowania energii. W dowolnym zamkniętym obwodzie (oczku) suma algebraiczna zmian potencjałów (sił elektromotorycznych $\mathcal{E}$ oraz spadków napięć $IR$) wynosi zero \citep{fizyka_dla_szkol_wyzszych_tom_3}:
	      \begin{equation}
		      \sum_{k} \Delta V_k = 0 \quad \lub \quad \sum \mathcal{E} = \sum IR
	      \end{equation}
\end{enumerate}

\subsection{Łączenie rezystorów}
W obwodach prądu stałego wyróżniamy podstawowe konfiguracje łączenia rezystorów:

\begin{itemize}
	\item \textbf{Połączenie szeregowe:} Przez wszystkie elementy płynie ten sam prąd. Rezystancja zastępcza jest sumą rezystancji składowych:
	      \begin{equation}
		      R_{z} = \sum_{i} R_i
	      \end{equation}
	\item \textbf{Połączenie równoległe:} Na wszystkich elementach występuje to samo napięcie. Odwrotność rezystancji zastępczej jest sumą odwrotności rezystancji składowych \citep{fizyka_dla_szkol_wyzszych_tom_3}:
	      \begin{equation}
		      \frac{1}{R_{z}} = \sum_{i} \frac{1}{R_i}
	      \end{equation}
\end{itemize}

\subsection{Transfiguracja gwiazda-trójkąt}
W przypadku bardziej złożonych struktur, których nie można sprowadzić do połączeń szeregowych lub równoległych (np. mostki), stosuje się transformację układu połączonego w trójkąt ($\Delta$) na równoważny układ połączony w gwiazdę ($Y$) lub odwrotnie. Dla transformacji trójkąta (rezystancje $R_{12}, R_{23}, R_{31}$) w gwiazdę (rezystancje $R_1, R_2, R_3$), wzory na rezystancje zastępcze przyjmują postać \citep{fizyka_dla_szkol_wyzszych_tom_3}:
\begin{equation}
	R_1 = \frac{R_{12} R_{31}}{R_{12} + R_{23} + R_{31}}, \quad
	R_2 = \frac{R_{12} R_{23}}{R_{12} + R_{23} + R_{31}}, \quad
	R_3 = \frac{R_{31} R_{23}}{R_{12} + R_{23} + R_{31}}
\end{equation}
Zastosowanie tej transformacji pozwala na uproszczenie topologii obwodu i obliczenie rezystancji zastępczej metodami podstawowymi.

\bibliographystyle{apalike}
\bibliography{bibliography}

\end{document}
