\documentclass[a4paper,12pt]{article}
\usepackage[left=2cm,right=2cm,top=2cm,bottom=2cm]{geometry} 
\usepackage{multicol} 
\usepackage{ragged2e} 
\usepackage{graphicx} 
\usepackage{float}
\usepackage{caption}
\usepackage{amsmath} 
\usepackage{amssymb} 
\usepackage[svgnames]{xcolor}
\usepackage[colorlinks=true, urlcolor=blue, linkcolor=black, citecolor=orange]{hyperref} 
\usepackage{polski} 
\usepackage[utf8]{inputenc} 
\usepackage{enumitem} 
\usepackage{indentfirst}
\usepackage{array}
\usepackage{longtable}
\usepackage{pdflscape}
\usepackage[round]{natbib}
\setlist[itemize]{itemsep=0pt, topsep=0pt}
\usepackage{siunitx}

% SI setup preferences
\sisetup{output-decimal-marker={,}}
\sisetup{exponent-product = \cdot}
\sisetup{per-mode = symbol}

% LTeX: language=pl-PL
\begin{document}

% --- NAGŁÓWEK ---
\noindent
\begin{minipage}{0.5\textwidth}
	\raggedright
	\textbf{} \\
	II rok, Fizyka \\
	Wtorek, 8:00-10:15
	\vspace{0.5cm}
\end{minipage}
\begin{minipage}{0.5\textwidth}
	\raggedleft
	2026-01-13 \\
	\vspace{0.5cm}
	Prowadząca: \\
	dr Sylwia Owczarek
\end{minipage}

\vspace{2cm}
\begin{center}
	\LARGE \textbf{Ćwiczenie nr 46} \\[0.5cm]
	\Large \textbf{Prawa Ohma i Kirchhoffa}
\end{center}

\vspace{1cm}

% --- TREŚĆ ---
\section{Wstęp teoretyczny}

\subsection{Prawo Ohma i rezystancja}
Natężenie prądu elektrycznego $I$ definiuje się jako ilość ładunku przepływającego przez przekrój poprzeczny przewodnika w jednostce czasu ($I = dQ/dt$).  Dla wielu materiałów (zwanych omowymi) gęstość prądu $\vec{J}$ jest wprost proporcjonalna do natężenia pola elektrycznego $\vec{E}$. W ujęciu makroskopowym zależność ta, znana jako prawo Ohma, wiąże natężenie prądu $I$ płynącego przez element z napięciem $U$ przyłożonym do jego końców \citep{fizyka_dla_szkol_wyzszych_tom_3}:
\begin{equation}
	I = \frac{U}{R}
\end{equation}
Współczynnik proporcjonalności $R$ nazywamy rezystancją (oporem elektrycznym). Jednostką rezystancji w układzie SI jest om ($\Omega$). Rezystancja elementu zależy od jego geometrii oraz rezystywności materiału $\rho$:
\begin{equation}
	R = \rho \frac{l}{S}
\end{equation}
gdzie $l$ to długość przewodnika, a $S$ to pole jego przekroju poprzecznego \citep{fizyka_dla_szkol_wyzszych_tom_3}.

\subsection{Prawa Kirchhoffa}
Analiza złożonych obwodów elektrycznych opiera się na dwóch fundamentalnych zasadach wynikających z praw zachowania:

\begin{enumerate}
	\item \textbf{I Prawo Kirchhoffa (Węzłowe):} Wynika z zasady zachowania ładunku. Suma algebraiczna natężeń prądów wpływających do węzła i wypływających z niego jest równa zeru \citep{fizyka_dla_szkol_wyzszych_tom_3}:
	      \begin{equation}
		      \sum_{k} I_k = 0
	      \end{equation}

	\item \textbf{II Prawo Kirchhoffa (Oczkowe):} Wynika z zasady zachowania energii. W dowolnym zamkniętym obwodzie (oczku) suma algebraiczna zmian potencjałów (sił elektromotorycznych $\mathcal{E}$ oraz spadków napięć $IR$) wynosi zero \citep{fizyka_dla_szkol_wyzszych_tom_3}:
	      \begin{equation}
		      \sum_{k} \Delta V_k = 0 \quad \lub \quad \sum \mathcal{E} = \sum IR
	      \end{equation}
\end{enumerate}

\subsection{Łączenie rezystorów}
W obwodach prądu stałego wyróżniamy podstawowe konfiguracje łączenia rezystorów:

\begin{itemize}
	\item \textbf{Połączenie szeregowe:} Przez wszystkie elementy płynie ten sam prąd. Rezystancja zastępcza jest sumą rezystancji składowych:
	      \begin{equation}
		      R_{z} = \sum_{i} R_i
	      \end{equation}
	\item \textbf{Połączenie równoległe:} Na wszystkich elementach występuje to samo napięcie. Odwrotność rezystancji zastępczej jest sumą odwrotności rezystancji składowych \citep{fizyka_dla_szkol_wyzszych_tom_3}:
	      \begin{equation}
		      \frac{1}{R_{z}} = \sum_{i} \frac{1}{R_i}
	      \end{equation}
\end{itemize}

\subsection{Transfiguracja gwiazda-trójkąt}
W przypadku bardziej złożonych struktur, których nie można sprowadzić do połączeń szeregowych lub równoległych (np. mostki), stosuje się transformację układu połączonego w trójkąt ($\Delta$) na równoważny układ połączony w gwiazdę ($Y$) lub odwrotnie. Dla transformacji trójkąta (rezystancje $R_{12}, R_{23}, R_{31}$) w gwiazdę (rezystancje $R_1, R_2, R_3$), wzory na rezystancje zastępcze przyjmują postać \citep{fizyka_dla_szkol_wyzszych_tom_3}:
\begin{equation}
	R_1 = \frac{R_{12} R_{31}}{R_{12} + R_{23} + R_{31}}, \quad
	R_2 = \frac{R_{12} R_{23}}{R_{12} + R_{23} + R_{31}}, \quad
	R_3 = \frac{R_{31} R_{23}}{R_{12} + R_{23} + R_{31}}
\end{equation}
Zastosowanie tej transformacji pozwala na uproszczenie topologii obwodu i obliczenie rezystancji zastępczej metodami podstawowymi.

% ---------- OPIS DOŚWIADCZENIA ----------
% \section{Opis doświadczenia}

% ---------- OPRACOWANIE WYNIKÓW POMIARÓW ----------
\section{Opracowanie wyników pomiarów}

% ---------- TABELE ----------
\subsection{Tabele pomiarowe}

% Tabela 1: Wartości rezystancji
\begin{table}[h!]
	\centering
	\begin{tabular}{|c|c|c|c|c|}
		\hline
		$R_1$       & $R_2$       & $R_3$       & $R_4$       & $R_5$       \\
		$[k\Omega]$ & $[k\Omega]$ & $[k\Omega]$ & $[k\Omega]$ & $[k\Omega]$ \\
		\hline
		0,192       & 2,39        & 0,557       & 0,558       & 0,877       \\
		\hline
	\end{tabular}
	\caption{Zmierzone wartości rezystancji}
	\label{tab:rezystancje}
\end{table}

% Tabela 2: Pomiary dla Układu 1
\begin{table}[H]
	\centering
	\begin{tabular}{|c|c|c|c|c|}
		\hline
		$U_1$  & $U_2$  & $U_3$  & $U_4$  & $U_5$  \\
		$[V]$  & $[V]$  & $[V]$  & $[V]$  & $[V]$  \\
		\hline
		2,03   & 7,93   & 4,08   & 3,81   & ---    \\
		\hline
		\hline
		$I_1$  & $I_2$  & $I_3$  & $I_4$  & $I_5$  \\
		$[mA]$ & $[mA]$ & $[mA]$ & $[mA]$ & $[mA]$ \\
		\hline
		10,73  & 3,32   & 7,44   & 7,44   & ---    \\
		\hline
	\end{tabular}
	\caption{Wyniki pomiarów dla Układu 1}
	\label{tab:uklad1}
\end{table}

% Tabela 3: Pomiary dla Układu 2
\begin{table}[H]
	\centering
	\begin{tabular}{|c|c|c|c|c|}
		\hline
		$U_1$  & $U_2$  & $U_3$  & $U_4$  & $U_5$  \\
		$[V]$  & $[V]$  & $[V]$  & $[V]$  & $[V]$  \\
		\hline
		1,64   & 8,32   & 2,87   & 5,37   & 4,57   \\
		\hline
		\hline
		$I_1$  & $I_2$  & $I_3$  & $I_4$  & $I_5$  \\
		$[mA]$ & $[mA]$ & $[mA]$ & $[mA]$ & $[mA]$ \\
		\hline
		13,85  & 3,48   & 5,23   & 10,49  & 5,28   \\
		\hline
	\end{tabular}
	\caption{Wyniki pomiarów dla Układu 2}
	\label{tab:uklad2}
\end{table}

% Tabela 4: rezystancja zastępcza
\begin{table}[H]
	\centering
	\begin{tabular}{|c|}
		\hline
		$R_z$       \\
		$[k\Omega]$ \\
		\hline
		0,901       \\
		\hline
	\end{tabular}
	\caption{Zmierzona rezystancja zastępcza}
	\label{tab:rz}
\end{table}

% ---------- OBLICZENIA ----------
\subsection{Obliczenie rezystancji R1-R4}

Na podstawie pomiarów napięć i natężeń prądu dla układu z schematu 1, korzystając z prawa Ohma, obliczono wartości rezystancji R1 ÷ R4. Prawo Ohma wyraża się wzorem:
\begin{equation}
	R = \frac{U}{I}
\end{equation}


Wyniki obliczeń przedstawiono w tabeli \ref{tab:obliczone_rezystancje}.

\begin{table}[h!]
	\centering
	\begin{tabular}{|c|c|}
		\hline
		Rezystor & Wartość R $[\Omega]$ \\
		\hline
		$R_1$    & 189,19               \\
		$R_2$    & 2388,55              \\
		$R_3$    & 548,39               \\
		$R_4$    & 512,10               \\
		\hline
	\end{tabular}
	\caption{Obliczone wartości rezystancji R1-R4}
	\label{tab:obliczone_rezystancje}
\end{table}

Przykładowe obliczenie dla rezystora R1:
\begin{equation}
	R_1 = \frac{U_1}{I_1} = \frac{2{,}03}{0{,}01073} \approx 189{,}19 \, \Omega
\end{equation}

\subsubsection*{Porównanie wartości rezystancji obliczonych ze zmierzonymi}

% TODO: Porównanie rezystancji obliczonych z wartościami zmierzonymi

\subsubsection*{Porównanie wartości rezystancji

\subsection{Obliczenie natężeń z praw Kirchhoffa}

Z pierwszego prawa Kirchhoffa:

\begin{align*}
	 & I_1 - I_3 - I_2 = 0 \\
	 & I_3 - I_4 = 0       \\
\end{align*}

Z drugiego prawa Kirchhoffa:

\begin{align*}
	 & \mathcal{E} - U_1 - U_2 = 0 \\
	 & U_2 - U_3 - U_4 = 0
\end{align*}

Z prawa Ohma dla całego układu:

\begin{equation*}
	I_1 = \frac{\mathcal{E}}{R_z} = \frac{10,0}{901} =
\end{equation*}
\noindent
gdzie $\mathcal{E} = 10,0 V$ to napięcie zasilania układu. Stąd:

\subsubsection*{Porównanie wartości natężeń obliczonych ze zmierzonymi}

% TODO: Porównanie wartości natężenia zmierzonych i obliczonych

\subsection{Obliczenie oporu zastępczego}

Z prawa łączenia rezystorów szeregowo i równolegle:

\begin{align*}
	 & R_z = R_1 + \frac{1}{\frac{1}{R_2} + \frac{1}{R_3 + R_4}}
\end{align*}

\subsubsection*{Porównanie wartości rezystancji oblizonych ze zmierzonymi}

% TODO: Porównanie wartości rezystancji oblizonych ze zmierzonymi

% ---------- NIEPEWNOŚCI ----------
\section{Ocena niepewności pomiaru}

% TODO: Korzystając z prawa przenoszenia niepewności maksymalnej (wzór (18) ONP) wyznaczyć niepewności maksymalne wyznaczonych wartości oporów R1 ÷ R4 i oporu zastępczego Rz. Przyjąć dla miernika cyfrowego niepewność maksymalną (punkt 3 ONP) ± 1% zakresu pomiarowego, natomiast dla zasilacza 0,1 [V].



% ---------- WNIOSKI ----------
% \section{Wnioski}

% ---------- WYKRESY ----------
% \section{Wykresy}
% \newpage
\bibliographystyle{apalike}
\bibliography{bibliography}


\end{document}
