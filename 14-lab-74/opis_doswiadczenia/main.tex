\documentclass[a4paper,11pt]{article}
\usepackage[utf8]{inputenc}
\usepackage[T1]{fontenc}
\usepackage[polish]{babel}
\usepackage{enumitem}

\begin{document}

\setcounter{section}{1} % Ustawienie licznika tak, aby kolejna sekcja miała numer 2
\section{Opis doświadczenia}

Układ pomiarowy składał się z głowicy elektrooptycznej (zawierającej diodę elektroluminescencyjną i fotodiodę), dwukanałowego oscyloskopu, małego lusterka odniesienia oraz dużego lustra pomiarowego umieszczonego na ławie optycznej. Pomiary przeprowadzono według następującej procedury:

\begin{enumerate}
	\item Zestawiono układ pomiarowy, umieszczając małe lusterko odblaskowe na oknie głowicy, a duże lustro pomiarowe w pewnej odległości od niej, podłączając sygnał wyjściowy do oscyloskopu.
	\item Ustawiono podstawę czasu oscyloskopu na zakres 0,1 $\mu$s/dz oraz dobrano czułość toru pomiarowego tak, aby obserwować wyraźny impuls odniesienia pochodzący od małego lusterka.
	\item Odsunięto duże lustro na odległość około 10 m i wyregulowano położenie soczewki oraz lustra w celu uzyskania maksymalnej amplitudy drugiego impulsu (sygnału odbitego od dużego lustra).
	\item Zaobserwowano na ekranie oscyloskopu dwa impulsy: pierwszy pochodzący od małego lusterka (startowy) i drugi od dużego lustra (odbity).
	\item Wykonano serię pomiarów, zmieniając położenie dużego lustra tak, aby odległość między impulsami na ekranie oscyloskopu zmieniała się o określoną wartość działek.
	\item Dla każdego położenia zmierzono fizyczną odległość $L$ między źródłem światła a dużym lustrem oraz odczytano odpowiadający jej odstęp czasowy (odległość impulsów na ekranie).
	\item Na podstawie zebranych danych (zależności drogi przebytej przez światło $2L$ od czasu opóźnienia) wyznaczono prędkość światła w powietrzu.
\end{enumerate}

\end{document}
