\documentclass[a4paper,12pt]{article}
\usepackage[left=2cm,right=2cm,top=2cm,bottom=2cm]{geometry} % Do ustawień marginesów
\usepackage{multicol} % Dla podziału na kolumny
\usepackage{ragged2e} % Dla justowania tekstu
\usepackage{graphicx} % Required for inserting images
\usepackage{float}
\usepackage{caption}
\usepackage{amsmath} % Math formulas
\usepackage{amssymb} % Symbols
\usepackage[svgnames]{xcolor}
\usepackage[colorlinks=true, urlcolor=blue, linkcolor=black, citecolor=orange]{hyperref} % Hyperlinks
\usepackage{polski} % Polish language
\usepackage[utf8]{inputenc} % Text encoding
\usepackage{enumitem} % Pakiet do elastycznego sterowania listami
\usepackage{indentfirst}
\usepackage{array}
\usepackage{longtable}
\usepackage{pdflscape}
\setlist[itemize]{itemsep=0pt, topsep=0pt}
\usepackage[round]{natbib}
\usepackage{siunitx}
\sisetup{output-decimal-marker={,}}
\sisetup{exponent-product = \cdot}
% LTeX: language=pl-PL

\begin{document}

% Górna część strony
\noindent
\begin{minipage}{0.5\textwidth}
    \raggedright
    \textbf{Piotr Durniat, 347364} \\
    II rok, Fizyka \\
    Wtorek, 8:00-10:15 \\
    \vspace{0.5cm}
    \vspace{0.5cm}
\end{minipage}%
\begin{minipage}{0.5\textwidth}
    \raggedleft
    22.10.2025 \\
    \vspace{0.5cm}
    Prowadząca: \\
    dr Sylwia Owczarek
\end{minipage}

% Tytuł ćwiczenia
\vspace{2cm}
\begin{center}
    \LARGE \textbf{Ćwiczenie nr 74} \\[0.5cm]
    \Large \textbf{Wyznaczanie prędkości światła}
\end{center}

% Reszta treści
\vspace{1cm} % Kolejny odstęp
\noindent

% \tableofcontents
% \newpage

% ---------- WSTĘP TEORETYCZNY ----------
\section{Wstęp teoretyczny}

\subsection*{Fale elektromagnetyczne}

Fale elektromagnetyczne (EM) to fale poprzeczne składające się ze zmieniających się w czasie pól elektrycznego i magnetycznego, które drgają w kierunkach prostopadłych do kierunku rozchodzenia się fali i do siebie nawzajem.
Przykładowymi falami elektromagnetycznymi są: światło widzialne,fale radiowe, podczerwień, ultrafiolet i promieniowanie rentgenowskie.
W przeciwieństwie do fal mechanicznych, fale EM nie potrzebują ośrodka do propagacji, mogą rozchodzić się w próżni.
Polaryzacja fali EM jest zdefiniowana jako kierunek równoległy do wektora natężenia pola elektrycznego $\vec{E}$~\citep{fizyka_dla_szkol_wyzszych_tom_3}.

\subsection*{Równanie falowe, prędkość fali}

Istnienie i zachowanie fal elektromagnetycznych jest opisywane przez równania Maxwella. Z tych równań można wyprowadzić równanie falowe, które dla fali płaskiej poruszającej się w próżni wzdłuż osi x przyjmuje postać:

$$
    \frac{\partial^2 E_y}{\partial x^2} = \epsilon_0 \mu_0 \frac{\partial^2 E_y}{\partial t^2}
$$
Z porównania tego równania z ogólną postacią równania falowego wynika, że prędkość fali elektromagnetycznej w próżni, oznaczana jako $c$, jest stałą fizyczną i wynosi:
$$
    c = \frac{1}{\sqrt{\epsilon_0 \mu_0}}
$$
gdzie:
\begin{itemize}
    \item $\epsilon_0$ to przenikalność elektryczna próżni,
    \item $\mu_0$ to przenikalność magnetyczna próżni.
\end{itemize}

Wartość tej prędkości wynosi $c = 299 \ 792 \ 458 \ \text{m/s}$.\citep{fizyka_dla_szkol_wyzszych_tom_3}

\subsection*{Własności fal w próżni i ośrodkach. Prędkość rozchodzenia się fal a współczynnik załamania ośrodka}

Prędkość światła w ośrodku materialnym ($v$) jest mniejsza niż w próżni ($c$), ponieważ światło oddziałuje z atomami w danym ośrodku. Zależność ta opisywan jest przez stałą materiałowa zwaną współczynnikiem załamania $n$:

$$
    n = \frac{c}{v}
$$

Współczynnik załamania $n$ jest zawsze większy lub równy 1, ponieważ $v$ jest zawsze mniejsze lub równe $c$. Dla próżni wynosi 1.
Wartość współczynnika załamania zależy od rodzaju ośrodka oraz od od długości fali światła.
Zależność współczynnika załamania od długości światła prowadzi do zjawiska rozszczepienia (dyspersji)~\citep{fizyka_dla_szkol_wyzszych_tom_3}.

\subsection*{Metody pomiaru prędkości światła}

Historycznie prędkość światła mierzono wieloma metodami, zarówno astronomicznymi, jak i laboratoryjnymi.
\begin{itemize}
    \item \textbf{Metoda astronomiczna Rømera (1675):} Ole Rømer, obserwując zaćmienia księżyca Jowisza -- Io, zauważył, że okresy jego obiegu wydają się zmieniać w zależności od położenia Ziemi na orbicie. Zinterpretował to jako efekt skończonej prędkości światła, które potrzebuje więcej lub mniej czasu, aby dotrzeć do Ziemi, gdy ta oddala się od Jowisza lub się do niego zbliża. Na tej podstawie oszacował prędkość światła na $2 \cdot 10^8 \ \text{m/s}$.
    \item \textbf{Metoda koła zębatego Fizeau (1849):} Armand Fizeau użył obracającego się koła zębatego, aby "ciąć" wiązkę światła na impulsy. Impulsy te przebywały odległość do odległego lustra i z powrotem. Prędkość obrotu koła była tak dobrana, aby powracające światło było blokowane przez kolejny ząb. Znając prędkość obrotową koła, liczbę zębów i odległość, Fizeau wyznaczył prędkość światła na $3,15 \cdot 10^8 \ \text{m/s}$.
    \item \textbf{Metoda wirującego zwierciadła Foucaulta (1862):} Jean Foucault udoskonalił metodę Fizeau, zastępując koło zębate wirującym lustrem, co pozwoliło na dokładniejszy pomiar. Jego wynik wyniósł ($2,98 \cdot 10^8 \ \text{m/s}$) i różni się tylko o 0,6\% od dzisiejszej wartości~\citep{fizyka_dla_szkol_wyzszych_tom_3}
\end{itemize}

\subsection*{Zasada działania oscyloskopu}

Oscyloskop to przyrząd pomiarowy służący do obserwacji i analizy sygnałów elektrycznych w funkcji czasu. Posiada ekran, na którym wyświetlany jest wykres sygnału.

\begin{itemize}
    \item \textbf{Oś pozioma (X)} reprezentuje \textbf{czas}. Plamka świetlna przesuwa się po niej ze stałą, regulowaną prędkością, tworząc podstawę czasu. Ustawienie podstawy czasu (pokrętło B w instrukcji) pozwala na "rozciąganie" lub "ściskanie" obserwowanego przebiegu w osi czasu.
    \item \textbf{Oś pionowa (Y)} reprezentuje \textbf{napięcie (amplitudę)} sygnału wejściowego. Badany sygnał powoduje odchylenie plamki w pionie, proporcjonalne do jego chwilowej wartości. Czułość (pokrętło A w instrukcji) reguluje wzmocnienie sygnału w osi pionowej.
\end{itemize}

W rezultacie na ekranie powstaje obraz zależności napięcia sygnału od czasu. W tym ćwiczeniu oscyloskop jest używany do pomiaru przesunięcia czasowego ($\Delta t$) między dwoma impulsami świetlnymi.


% ---------- OPIS DOŚWIADCZENIA ----------
\section{Opis doświadczenia}

% ---------- OPRACOWANIE WYNIKÓW POMIARÓW ----------
\section{Opracowanie wyników pomiarów}

% ---------- TABELE ----------
\subsection{Tabele pomiarowe}

\begin{table}[H]
    \centering
    \begin{tabular}{|r|c|}
        \hline
        \textbf{Odległość od lustra [m]} & \textbf{Odległość między impulsami [l. podziałek]} \\ \hline
        \num{10.000} & $\frac{\num{3.5}}{5}$ \\ \hline
        \num{8.500}  & $\frac{3}{5}$ \\ \hline
        \num{6.250}  & $\frac{2}{5}$ \\ \hline
        \num{11.450} & $\frac{4}{5}$ \\ \hline
        \num{13.860} & $\frac{5}{5}$ \\ \hline
        \num{3.950}  & $\frac{1}{5}$ \\ \hline
    \end{tabular}
    \caption{Pomiary odległości źródła światła od lustra oraz odległości między impulsami.}
\end{table}

% ---------- OBLICZENIA ----------
\subsection{Czas opóźnienia impulsu światła}

Czas opóźnienia ($t$) został obliczony na podstawie wzoru \eqref{eq:czas_opoznienia} i przedstawiony w tabeli \ref{tab:tabela_opóznień}.

\begin{equation}
    \label{eq:czas_opoznienia}
    t = \Delta d \cdot T
\end{equation}

gdzie:
\begin{itemize}
    \item $\Delta d$ to liczba podziałek odczytana z oscyloskopu,
    \item $T$ to podstawa czasu, która wynosi ($T = \SI{0.1}{\micro\second/\text{działkę}}$).
\end{itemize}

\begin{table}[h!]
    \centering
    \resizebox{0.95\textwidth}{!}{ % Zostawione resizebox zgodnie z prośbą
        \begin{tabular}{|r|r|r|}
            \hline
            \textbf{Odległość od lustra {[}m{]}} & \textbf{Liczba podziałek ($\Delta d$)}  & \textbf{Czas opóźnienia $t$ {[}ns{]}} \\ \hline
            \num{10.000} & \num{0.7} &  \num{70} \\ \hline
            \num{8.500} & \num{0.6} & \num{60} \\ \hline
            \num{6.250} & \num{0.4} & \num{40} \\ \hline
            \num{11.450} & \num{0.8} & \num{80} \\ \hline
            \num{13.860} &  \num{1.0} & \num{100} \\ \hline
            \num{3.950} &  \num{0.2} & \num{20} \\ \hline
        \end{tabular}
    }
    \caption{Czas opóźnienia impulsu światła.}
    \label{tab:tabela_opóznień}
\end{table}

\subsubsection*{Przykładowe obliczenia}

Dla odległości od lustra \SI{10.000}{\meter}, liczba podziałek odczytana z oscyloskopu wyniosła $\frac{3.5}{5}$.

\begin{align*}
    t & = \Delta d \cdot T = 0{,}7 \cdot 0.1 = \SI{0.07}{\micro\second} = \SI{70}{\nano\second}
\end{align*}

\subsection{Prędkość światła dla każdego pomiaru}

Prędkość światła ($v$) została obliczona dla każdego pomiaru na podstawie wzoru \eqref{eq:speed_of_light}, a wyniki zapisano w tabeli \ref{tab:speed_of_light}

\begin{equation}
    \label{eq:speed_of_light}
    v = \frac{2L}{t}
\end{equation}

gdzie:
\begin{itemize}
    \item $L$ to odległość od lustra [m],
    \item $t$ to czas opóźnienia [s].
\end{itemize}

\begin{table}[h!]
    \centering
    \begin{tabular}{|c|c|c|}
        \hline
        \textbf{Odległość od lustra $L$ [m]} & \textbf{Czas opóźnienia $t$ [ns]} & \textbf{Prędkość światła $v$ [m/s]} \\ \hline
        \num{10.000} & \num{70} & \num{2.857e8} \\ \hline
        \num{8.500} & \num{60} & \num{2.833e8} \\ \hline
        \num{6.250} & \num{40} & \num{3.125e8} \\ \hline
        \num{11.450} & \num{80} & \num{2.863e8} \\ \hline
        \num{13.860} & \num{100} & \num{2.772e8} \\ \hline
        \num{3.950} & \num{20} & \num{3.950e8} \\ \hline
    \end{tabular}
    % }
    \caption{Prędkość światła dla poszczególnych pomiarów}
    \label{tab:speed_of_light}
\end{table}

\subsubsection*{Przykładowe obliczenia}

Dla odległości od lustra $L = \SI{10.000}{\meter}$ i czasu opóźnienia $t = \SI{70}{\nano\second}$ obliczona prędkość światła $v$:

\begin{align*}
    v & = \frac{2L}{t} = \frac{2 \cdot \SI{10.000}{\meter}}{\SI{70e-9}{\second}} = \frac{\SI{20.000}{\meter}}{\SI{70e-9}{\second}} \approx \SI{2.857e8}{\meter/\second}
\end{align*}

\subsection{Ostateczna prędkość światła}

Ostateczną wartość prędkości światła obliczono jako średnią arytmetyczną prędkości, otrzymanych z pierwszych 5 pomiarów. Pomiar dla odległości \SI{3.950}{\meter} został odrzucony, gdyż dokładne określenie odległości między impulsami było dla niego problematyczne -- impulsy nakładały się, a otrzymana wartość prędkości odstaje od pozostałych pomiarów. Otrzymano wynik:

\begin{equation*}
    \bar{v} \approx \SI{2.8900e8}{\frac{m}{s}}
\end{equation*}


% ---------- NIEPEWNOŚCI ----------
\section{Ocena niepewności pomiaru}

\subsection{Niepewności maksymalne pomiarów bezpośrednich}

Położenie między impulsami można było odczytać z dokładnością do $\num{0,5}$ podziałki, stąd maksymalna niepewność odległości impulsów $\Delta_d(\Delta d) = 1 \text{ podziałka}$

Odległość między źródłem światła a lustrem mierzono za pomocą miarki z podziałką \SI{0.001}{\meter}. Ze względu na to, że trudności sprawiało dokładne ułożenie miarki, to niepewność pomiaru odległości oszacowano na $\Delta_d L = \SI{0.01}{\meter}$.

\subsection{Niepewności standardowe pomiarów bezpośrednich}

Niepewność standardowa pomiaru odległości do lustra została obliczona na podstawie wzoru:
\begin{equation*}
    u(L) = \frac{\Delta_d L}{\sqrt{3}} = \frac{\SI{0.01}{\meter}}{\sqrt{3}} \approx \SI{0.0058}{\meter}
\end{equation*}

Niepewność standardowa pomiaru odległości między podziałkami została obliczona na podstawie wzoru:


\begin{equation*}
    u(\Delta d) = \frac{\Delta_d(\Delta d)}{\sqrt{3}} = \frac{1 \text{ działka}}{\sqrt{3}} \approx 0{,}58 \text{ działki}
\end{equation*}

\subsection{Niepewność standardowa czasu opóźnienia}

Czas opóźnienia \( t \) obliczany jest ze wzoru \( t = \Delta d \cdot T \). Zakładając, że niepewność podstawy czasu \( u(T) \approx 0 \), niepewność standardową \( u(t) \) wyznaczono z prawa propagacji niepewności dla wielkości nieskorelowanych:
$$
    u(t) = \sqrt{ \left(\frac{\partial t}{\partial \Delta d}\right)^2 u^2(\Delta d) + \left(\frac{\partial t}{\partial T}\right)^2 u^2(T) }
$$

$$
    u(t) = \sqrt{ \left( T \right)^2 u^2(\Delta d) + \left( \Delta d \right)^2 u^2(T) }
    =  \sqrt{ T^2 u^2(\Delta d) + (\Delta d)^2 \cdot 0 } = |T| \cdot u(\Delta d)
$$

Podstawiając wartość \( T = \SI{0.1e-6}{\second/\text{działkę}} \) oraz \( u(\Delta d) \approx 0{,}58 \text{ działki} \):

$$
    u(t) \approx \num{0.1e-6} \cdot 0{,}58  \approx \SI{5.8e-8}{\second}
$$

\subsection{Niepewność standardowa prędkości}

Niepewność standardową prędkości obliczono ze wzoru na niepewność złożonąch \eqref{eq:niepewnosc_zlozona}.

\begin{equation}
    \label{eq:niepewnosc_zlozona}
    u_c(y) = \sqrt{ \sum_{i=1}^{n} \left(\frac{\partial f}{\partial x_i}\right)^2 u^2(x_i) }
\end{equation}
Dla wzoru $v = \frac{2L}{t}$ otrzymano równanie:

\begin{equation*}
    u_c(v) = \sqrt{ \left(\frac{\partial v}{\partial L}\right)^2 u^2(L) + \left(\frac{\partial v}{\partial t}\right)^2 u^2(t) } = \sqrt{ \left(\frac{2}{t}\right)^2 u^2(L) + \left(-\frac{2L}{t^2}\right)^2 u^2(t) }
\end{equation*}
Otrzymane wartości zapisano w tabeli \ref{tab:final_uncertainties}.

\begin{table}[h!]
    \centering
    \begin{tabular}{|r|r|r|r|}
        \hline
        {\textbf{$L$ [\si{\meter}]}} & {\textbf{$t$ [\si{\second}]}} & {\textbf{$v$ [\si{\meter/\second}]}} & {\textbf{$u_c(v)$ [\si{\meter/\second}]}} \\
        \hline
        \num{10.000} & \num{7.0e-8} & \num{2.857e8} & \num{2.4e8} \\
        \hline
        \num{8.500}  & \num{6.0e-8} & \num{2.833e8} & \num{2.7e8} \\
        \hline
        \num{6.250}  & \num{4.0e-8} & \num{3.125e8} & \num{4.5e8} \\
        \hline
        \num{11.450} & \num{8.0e-8} & \num{2.862e8} & \num{2.1e8} \\
        \hline
        \num{13.860} & \num{1.0e-7} & \num{2.772e8} & \num{1.6e8} \\
        \hline
        \num{3.950}  & \num{2.0e-8} & \num{3.950e8} & \num{1.1e9} \\
        \hline
    \end{tabular}
    \caption{Obliczone prędkości światła i ich niepewności złożone}
    \label{tab:final_uncertainties} % Caption and Label moved to the bottom
\end{table}

\subsection*{Przykładowe obliczenia (dla L = \SI{10.000}{\meter})}

Dla \(L=\SI{10.000}{\meter}\), \(t=\SI{7.0e-8}{\second}\), \(v \approx \SI{2.857e8}{\meter/\second}\), \(u(L) \approx \SI{0.00577}{\meter}\), \(u(t) \approx \SI{5.77e-8}{\second}\), niepewność złożona wynosi:
\begin{align*}
    u_c(v) & = \SI{2.857e8}{\meter/\second} \cdot \sqrt{ \left( \frac{\SI{0.00577}{\meter}}{\SI{10.000}{\meter}} \right)^2 + \left( \frac{\SI{5.77e-8}{\second}}{\SI{7.0e-8}{\second}} \right)^2 } \approx \SI{2.4e8}{\meter/\second}
\end{align*}

\subsection{Niepewność standardowa średniej prędkości światła}

Średnią prędkość światła \( \bar{v} \approx \SI{2.8900e8}{\meter/\second} \) obliczono jako średnią arytmetyczną z \( N=5 \) pomiarów (po odrzuceniu ostatniego). Niepewność standardową tej średniej \( u(\bar{v}) \) wyznaczono metodą typu A, korzystając ze wzoru na odchylenie standardowe średniej arytmetycznej \eqref{eq:odchylenie-std}:
\begin{equation}
    \label{eq:odchylenie-std}
    u(\bar{v}) = \sqrt{\frac{1}{N(N-1)} \sum_{i=1}^{N} (v_i - \bar{v})^2}
\end{equation}
gdzie \( v_i \) to wynik \(i\)-tego pomiaru prędkości. Podstawiając wartości (w \si{\meter/\second}) dla pierwszych pięciu pomiarów:
\begin{align*}
    u(\bar{v}) & = \sqrt{\frac{1}{5(5-1)} \left[ (\num{2.857e8} - \num{2.890e8})^2 + \dots + (\num{2.772e8} - \num{2.890e8})^2 \right]} \approx \SI{6.09e6}{\meter/\second}
\end{align*}


% ---------- WNIOSKI ----------
\section{Wnioski}


Na podstawie przeprowadzonych pomiarów i obliczeń wyznaczono średnią prędkość światła w powietrzu. Po odrzuceniu jednego pomiaru obarczonego dużym błędem, jako wynik końcowy przyjęto średnią arytmetyczną z pozostałych pięciu pomiarów.

Wyznaczona średnia prędkość światła wyniosła $\bar{v} = \SI{2.890e8}{\meter/\second}$, a jej niepewność standardowa $u(\bar{v}) = \SI{0.061e8}{\meter/\second}$

% ---------- WYKRESY ----------
% \section{Wykresy}

\bibliographystyle{apalike}
\bibliography{bibliography}

\end{document}
