\section{Wstęp teoretyczny}

Celem eksperymentu jest laboratoryjne odwzorowanie procesów powstawania kraterów na powierzchniach ciał niebieskich oraz weryfikacja zależności między energią kinetyczną uderzającego obiektu a wielkością powstałego krateru.

W eksperymencie wykorzystujemy kilka podstawowych zjawisk fizycznych. Spadek swobodny to ruch ciała pod wpływem wyłącznie siły grawitacji, opisany równaniami:
\begin{align*}
    h &= \frac{1}{2}gt^2 \\
    v &= gt
\end{align*}
gdzie $h$ -- wysokość, $v$ -- prędkość, $g$ -- przyspieszenie ziemskie, $t$ -- czas.

Zgodnie z zasadą zachowania energii, energia całkowita układu izolowanego pozostaje stała:
\begin{align*}
    E_p + E_k = \text{const}
\end{align*}
Dla kulki spadającej z wysokości $h$ zachodzi przemiana energii potencjalnej w kinetyczną:
\begin{align*}
    E_p = mgh \rightarrow E_k = \frac{1}{2}mv^2
\end{align*}

Uderzenie spadającej kulki w piasek jest zderzeniem niesprężystym, podczas którego część energii kinetycznej zostaje przekształcona w energię deformacji ośrodka. Rozpatrujemy dwie hipotezy dotyczące zależności między energią kinetyczną uderzającego obiektu a średnicą powstałego krateru:
\begin{align*}
    \text{Model I:} \quad E_k &\propto D^3 \quad \text{(energia przeznaczona głównie na deformację objętości)} \\
    \text{Model II:} \quad E_k &\propto D^4 \quad \text{(część energii zamieniana na potencjalną materiału wyrzuconego)}
\end{align*}

Aby zlinearyzować zależność potęgową, stosujemy logarytmowanie stronami:
\begin{align*}
    E_k &\propto D^n \\
    \log(E_k) &= \log(A) + n\log(D)
\end{align*}
Wykreślając zależność $\log(D)$ od $\log(E_k)$ w układzie współrzędnych, otrzymujemy linię prostą o współczynniku kierunkowym $1/n$, co pozwala określić, który z modeli ($n=3$ czy $n=4$) lepiej opisuje eksperyment.

W skali astronomicznej, uderzenia meteorytów w powierzchnie planet i księżyców są zderzeniami niesprężystymi o ogromnych energiach. Krater w Arizonie (średnica 1200 m) powstał w wyniku uderzenia meteorytu o masie $M_m \approx 3 \times 10^8$ kg z prędkością $v_m \approx 12000$ m/s. Przeprowadzany eksperyment stanowi model tego zjawiska w skali laboratoryjnej, umożliwiający ekstrapolację wyników do zjawisk astronomicznych.
