\documentclass[a4paper,12pt]{article}
\usepackage[left=2cm,right=2cm,top=2cm,bottom=2cm]{geometry} % Do ustawień marginesów
\usepackage{multicol} % Dla podziału na kolumny
\usepackage{ragged2e} % Dla justowania tekstu
\usepackage{graphicx} % Required for inserting images
\usepackage{float}
\usepackage{caption}
\usepackage{amsmath} % Math formulas
\usepackage{amssymb} % Symbols
\usepackage[svgnames]{xcolor}
\usepackage[colorlinks=true, urlcolor=blue, linkcolor=black, citecolor=orange]{hyperref} % Hyperlinks
\usepackage{polski} % Polish language
\usepackage[utf8]{inputenc} % Text encoding
\usepackage{enumitem} % Pakiet do elastycznego sterowania listami
\usepackage{indentfirst}
\usepackage{array}
\usepackage{booktabs}
\documentclass{article}
\usepackage{longtable}

\begin{document}


% Górna część strony
\noindent
\begin{minipage}{0.5\textwidth}
    \raggedright
    \textbf{Piotr Durniat} \\
    I rok, Fizyka \\
    Wtorek, 8:00-10:15 \\
    \vspace{0.5cm}
    \vspace{0.5cm}
\end{minipage}%
\begin{minipage}{0.5\textwidth}
    \raggedleft
    % Data wykonania pomiarów: \\
    19.03.2025  \\
    \vspace{0.5cm} % Dodatkowa linia przerwy
    Prowadząca: \\
    dr Iwona Mróz
\end{minipage}

% Tytuł ćwiczenia
\begin{center}
    \LARGE \textbf{Ćwiczenie nr 12 - Tabele pomiarowe} \\[0.5cm]
\end{center}

\noindent

\begin{table}[h]
    \renewcommand{\arraystretch}{1.3}
    \setlength{\tabcolsep}{20pt}
    \begin{longtable}{|c|c|}
        \hline
        Rodzaj kulki & $m$ [kg] \\
        \hline
        Mała & \\
        Średnia & \\
        Duża & \\
        \hline
    \end{longtable}
    \caption{Pomiary masy kulki}
\end{table}

\begin{table}[H]
    \renewcommand{\arraystretch}{1.3}
    \setlength{\tabcolsep}{25pt}
    \begin{longtable}{|c|c|c|c|c|c|}
        \hline
        \textbf{Wysokość [m]} & 0.25 & 0.5 & 1.0 & 1.5 & 2.0 \\
        \hline
        \hline
        \textbf{Nr pomiaru } & \multicolumn{5}{c|}{\textbf{Średnica krateru [m] (mała kulka)}} \\
        \hline
        1. & & & & & \\
        \hline
        2. & & & & & \\
        \hline
        3. & & & & & \\
        \hline
        4. & & & & & \\
        \hline
        5. & & & & & \\
        \hline
    \end{longtable}
\end{table}

\begin{table}[h]
    \renewcommand{\arraystretch}{1.3}
    \setlength{\tabcolsep}{25pt}
    \begin{tabular}{|c|c|c|c|c|}
        \hline
        \textbf{Wysokość [m]} & 0,5 & 1,0 & 1,5 & 2,0 \\
        \hline
        \hline
        \textbf{Nr pomiaru} & \multicolumn{4}{c|}{\textbf{Średnica krateru [m] (średnia kulka)}} \\
        \hline
        1. & & & & \\
        \hline
        2. & & & & \\
        \hline
        3. & & & & \\
        \hline
        4. & & & & \\
        \hline
        5. & & & & \\
        \hline
    \end{tabular}
    \label{tab:srednia_kulka}
\end{table}


\begin{table}[h]
    \renewcommand{\arraystretch}{1.3}
    \setlength{\tabcolsep}{25pt}
    \begin{tabular}{|c|c|c|}
        \hline
        \textbf{Wysokość [m]} & 1,5 & 2,0 \\
        \hline
        \hline
        \textbf{Nr pomiaru} & \multicolumn{2}{c|}{\textbf{Średnica krateru [m] (duża kulka) }} \\
        \hline
        1. & & \\
        \hline
        2. & & \\
        \hline
        3. & & \\
        \hline
        4. & & \\
        \hline
        5. & & \\
        \hline
    \end{tabular}
    \label{tab:duza_kulka}
\end{table}




\end{document}
