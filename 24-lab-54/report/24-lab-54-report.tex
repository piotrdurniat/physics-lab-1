\documentclass[a4paper,12pt]{article}
\usepackage[left=2cm,right=2cm,top=2cm,bottom=2cm]{geometry} 
\usepackage{multicol} 
\usepackage{ragged2e} 
\usepackage{graphicx} 
\usepackage{float}
\usepackage{caption}
\usepackage{amsmath} 
\usepackage{amssymb} 
\usepackage[svgnames]{xcolor}
\usepackage[colorlinks=true, urlcolor=blue, linkcolor=black, citecolor=orange]{hyperref} 
\usepackage{polski} 
\usepackage[utf8]{inputenc} 
\usepackage{enumitem} 
\usepackage{indentfirst}
\usepackage{array}
\usepackage{longtable}
\usepackage{pdflscape}
\usepackage[round]{natbib}
\setlist[itemize]{itemsep=0pt, topsep=0pt}
\usepackage{siunitx}

% SI setup preferences
\sisetup{output-decimal-marker={,}}
\sisetup{exponent-product = \cdot}
\sisetup{per-mode = symbol}

% LTeX: language=pl-PL

\begin{document}

\noindent
\begin{minipage}{0.5\textwidth}
	\raggedright
	% full name, index number
	\textbf{} \\
	II rok, Fizyka \\
	Wtorek, 8:00-10:15
	\vspace{0.5cm}
\end{minipage}
\begin{minipage}{0.5\textwidth}
	\raggedleft
	% date of the experiments
	29.04.2025 \\
	\vspace{0.5cm}
	Prowadząca: \\
	dr Sylwia Owczarek
\end{minipage}

\vspace{2cm}
\begin{center}
	% lab number
	\LARGE \textbf{Ćwiczenie nr 54} \\[0.5cm]
	% title
	\Large \textbf{Drgania relaksacyjne}
\end{center}

\vspace{1cm}
\noindent

% \tableofcontents
% \newpage

% ---------- WSTĘP TEORETYCZNY ----------
\section{Wstęp teoretyczny}

% ---------- OPIS DOŚWIADCZENIA ----------
% \section{Opis doświadczenia}

% ---------- OPRACOWANIE WYNIKÓW POMIARÓW ----------
\section{Opracowanie wyników pomiarów}

% ---------- TABELE ----------
\subsection{Tabele pomiarowe}

% Tabela 1: Pomiary czasu dla znanych pojemności (Seria I, R = 1,4 MΩ)
\begin{table}[H]
	\centering
	\begin{tabular}{|c|c|}
		\hline
		$C$             & $t_{20}$     \\
		$[\mu\text{F}]$ & $[\text{s}]$ \\
		\hline
		1,0             & 12,63        \\
		2,0             & 25,62        \\
		3,0             & 37,75        \\
		4,5             & 58,00        \\
		5,0             & 64,00        \\
		6,5             & 83,84        \\
		7,0             & 91,84        \\
		8,0             & 104,03       \\
		9,0             & 117,28       \\
		\hline
	\end{tabular}
	\caption{Zmierzone czasy trwania 20 cykli drgań dla znanych pojemności (Seria I, $R = \num{1,4e6}\,\Omega$)}
	\label{tab:pomiary_seria1}
\end{table}

% Tabela 2: Pomiary czasu dla znanych pojemności (Seria II, R = 2,6 MΩ)
\begin{table}[H]
	\centering
	\begin{tabular}{|c|c|}
		\hline
		$C$             & $t_{20}$     \\
		$[\mu\text{F}]$ & $[\text{s}]$ \\
		\hline
		1,0             & 23,32        \\
		2,0             & 46,37        \\
		3,0             & 69,15        \\
		4,5             & 103,84       \\
		5,0             & 110,07       \\
		6,5             & 150,91       \\
		7,0             & 163,43       \\
		8,0             & 187,75       \\
		9,0             & 210,66       \\
		\hline
	\end{tabular}
	\caption{Zmierzone czasy trwania 20 cykli drgań dla znanych pojemności (Seria II, $R = \num{2,6e6}\,\Omega$)}
	\label{tab:pomiary_seria2}
\end{table}

% Tabela 3: Pomiary czasu dla nieznanej pojemności $C_x$
\begin{table}[H]
	\centering
	\begin{tabular}{|c|c|}
		\hline
		Rezystancja        & $t_{20}$     \\
		$[\text{M}\Omega]$ & $[\text{s}]$ \\
		\hline
		1,4                & 45,65        \\
		2,6                & 89,22        \\
		\hline
	\end{tabular}
	\caption{Zmierzone czasy trwania 20 cykli drgań dla nieznanej pojemności $C_x$ przy dwóch wartościach rezystancji}
	\label{tab:pomiary_cx}
\end{table}

% Tabela 4: Parametry układu pomiarowego
\begin{table}[H]
	\centering
	\begin{tabular}{|c|c|}
		\hline
		Parametr                 & Wartość     \\
		\hline
		Napięcie zasilania $U_0$ & \num{140} V \\
		Liczba cykli $N$         & 20          \\
		\hline
	\end{tabular}
	\caption{Stałe parametry układu pomiarowego}
	\label{tab:parametry}
\end{table}

% ---------- OBLICZENIA ----------
\subsection{Wyznaczenie okresów drgań relaksacyjnych $T$ dla pomiarów z kondensatorami o znanej pojemności}

Okres drgań relaksacyjnych obliczono ze wzoru:
\begin{equation}
	T = \frac{t_{20}}{20}
\end{equation}
gdzie $t_{20}$ to zmierzony czas trwania 20 cykli drgań.

Wyniki obliczeń okresów przedstawiono w tabelach \ref{tab:okresy_seria1} i \ref{tab:okresy_seria2}.

\begin{table}[H]
	\centering
	\begin{tabular}{|c|c|c|}
		\hline
		$C$             & $t_{20}$     & $T$          \\
		$[\mu\text{F}]$ & $[\text{s}]$ & $[\text{s}]$ \\
		\hline
		1,0             & 12,63        & 0,6315       \\
		2,0             & 25,62        & 1,2810       \\
		3,0             & 37,75        & 1,8875       \\
		4,5             & 58,00        & 2,9000       \\
		5,0             & 64,00        & 3,2000       \\
		6,5             & 83,84        & 4,1920       \\
		7,0             & 91,84        & 4,5920       \\
		8,0             & 104,03       & 5,2015       \\
		9,0             & 117,28       & 5,8640       \\
		\hline
	\end{tabular}
	\caption{Okresy drgań relaksacyjnych dla znanych pojemności (Seria I, $R = \num{1,4e6}\,\Omega$)}
	\label{tab:okresy_seria1}
\end{table}

\begin{table}[H]
	\centering
	\begin{tabular}{|c|c|c|}
		\hline
		$C$             & $t_{20}$     & $T$          \\
		$[\mu\text{F}]$ & $[\text{s}]$ & $[\text{s}]$ \\
		\hline
		1,0             & 23,32        & 1,1660       \\
		2,0             & 46,37        & 2,3185       \\
		3,0             & 69,15        & 3,4575       \\
		4,5             & 103,84       & 5,1920       \\
		5,0             & 110,07       & 5,5035       \\
		6,5             & 150,91       & 7,5455       \\
		7,0             & 163,43       & 8,1715       \\
		8,0             & 187,75       & 9,3875       \\
		9,0             & 210,66       & 10,5330      \\
		\hline
	\end{tabular}
	\caption{Okresy drgań relaksacyjnych dla znanych pojemności (Seria II, $R = \num{2,6e6}\,\Omega$)}
	\label{tab:okresy_seria2}
\end{table}

Przykładowe obliczenie dla $C = 1,0\,\mu\text{F}$ (Seria I):
\begin{equation}
	T = \frac{12,63}{20} = 0,6315\,\text{s}
\end{equation}

\subsection{Wyznaczenie stałej $K$ z równania $T = K \cdot R \cdot C$}

Ze wzoru $T = K \cdot R \cdot C$ wyznaczono wartości stałej $K$ dla każdego pomiaru:
\begin{equation}
	K = \frac{T}{R \cdot C}
\end{equation}

\subsubsection*{Wartości stałej $K$ dla poszczególnych pomiarów}

\begin{table}[H]
	\centering
	\begin{tabular}{|c|c|c|c|}
		\hline
		$C$             & $T$ [s] & $K$    \\
		$[\mu\text{F}]$ &         &        \\
		\hline
		1,0             & 0,6315  & 0,4511 \\
		2,0             & 1,2810  & 0,4575 \\
		3,0             & 1,8875  & 0,4494 \\
		4,5             & 2,9000  & 0,4603 \\
		5,0             & 3,2000  & 0,4571 \\
		6,5             & 4,1920  & 0,4607 \\
		7,0             & 4,5920  & 0,4686 \\
		8,0             & 5,2015  & 0,4644 \\
		9,0             & 5,8640  & 0,4654 \\
		\hline
	\end{tabular}
	\caption{Wartości stałej $K$ dla Serii I ($R = \num{1,4e6}\,\Omega$)}
	\label{tab:K_seria1}
\end{table}

\begin{table}[H]
	\centering
	\begin{tabular}{|c|c|c|c|}
		\hline
		$C$             & $T$ [s] & $K$    \\
		$[\mu\text{F}]$ &         &        \\
		\hline
		1,0             & 1,1660  & 0,4485 \\
		2,0             & 2,3185  & 0,4459 \\
		3,0             & 3,4575  & 0,4433 \\
		4,5             & 5,1920  & 0,4438 \\
		5,0             & 5,5035  & 0,4233 \\
		6,5             & 7,5455  & 0,4465 \\
		7,0             & 8,1715  & 0,4490 \\
		8,0             & 9,3875  & 0,4513 \\
		9,0             & 10,5330 & 0,4501 \\
		\hline
	\end{tabular}
	\caption{Wartości stałej $K$ dla Serii II ($R = \num{2,6e6}\,\Omega$)}
	\label{tab:K_seria2}
\end{table}

\subsubsection*{Średnie wartości stałej $K$}

Obliczono średnie wartości stałej $K$ dla każdej serii oraz wartość łączną:
\begin{itemize}
	\item $K_{sr}$ (Seria I): 0,4594
	\item $K_{sr}$ (Seria II): 0,4446
	\item $K_{sr}$ (Całość): 0,4520
\end{itemize}

Wartość łączna $K_{sr} = 0,4520$ została wykorzystana do dalszych obliczeń pojemności $C_x$.

Przykładowe obliczenie dla $C = 1,0\,\mu\text{F}$ (Seria I):
\begin{equation}
	K = \frac{0,6315}{\num{1,4e6} \cdot \num{1,0e-6}} = 0,4511
\end{equation}

% ---------- NIEPEWNOŚCI ----------
% \section{Ocena niepewności pomiaru}

% ---------- WNIOSKI ----------
% \section{Wnioski}

% ---------- WYKRESY ----------
% \section{Wykresy}
% \newpage

\bibliographystyle{apalike}
\bibliography{bibliography}

\end{document}
