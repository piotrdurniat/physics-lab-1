\documentclass[a4paper,12pt]{article}
\usepackage[left=2cm,right=2cm,top=2cm,bottom=2cm]{geometry} % Do ustawień marginesów
\usepackage{multicol} % Dla podziału na kolumny
\usepackage{ragged2e} % Dla justowania tekstu
\usepackage{graphicx} % Required for inserting images
\usepackage{float}
\usepackage{caption}
\usepackage{amsmath} % Math formulas
\usepackage{amssymb} % Symbols
\usepackage[svgnames]{xcolor}
\usepackage[colorlinks=true, urlcolor=blue, linkcolor=black, citecolor=orange]{hyperref} % Hyperlinks
\usepackage{polski} % Polish language
\usepackage[utf8]{inputenc} % Text encoding
\usepackage{enumitem} % Pakiet do elastycznego sterowania listami
\usepackage{indentfirst}
\usepackage{array}

\begin{document}

% Górna część strony
\noindent
\begin{minipage}{0.5\textwidth}
    \raggedright
    \textbf{Piotr Durniat} \\
    I rok, Fizyka \\
    Wtorek, 8:00-10:15 \\
    \vspace{0.5cm}
    \vspace{0.5cm}
\end{minipage}%
\begin{minipage}{0.5\textwidth}
    \raggedleft
    Data wykonania pomiarów: \\
    14.04.2025 \\
    \vspace{0.5cm} % Dodatkowa linia przerwy
    Prowadząca: \\
    dr Iwona Mróz
\end{minipage}

% Tytuł ćwiczenia
\vspace{2cm} % Odstęp
\begin{center}
    \LARGE \textbf{Ćwiczenie nr 25} \\[0.5cm]
    \Large \textbf{Wyznaczanie współczynnika rozszerzalności cieplnej metali za pomocą dylatometru}
\end{center}

% Reszta treści
\vspace{1cm} % Kolejny odstęp
\noindent

\tableofcontents
\newpage

% ---------- WSTĘP TEORETYCZNY ----------
\section{Wstęp teoretyczny}

Zjawisko rozszerzalności cieplnej ciał stałych wynika z niesymetrycznego kształtu krzywej energii potencjalnej oddziaływań międzyatomowych. W ciałach stałych atomy nie są ściśle unieruchomione, lecz drgają wokół położeń równowagi. Oddziaływania międzyatomowe można opisać jako wypadkową sił przyciągania i odpychania:

\begin{equation}
    F = F_{\text{odpychania}} + F_{\text{przyciągania}} = \frac{A}{x^{13}} - \frac{B}{x^7}
\end{equation}

gdzie $A$ i $B$ są stałymi charakterystycznymi dla danego materiału.

Energia potencjalna oddziaływań $U(x)$ w okolicy położenia równowagi $x_0$ może być przybliżona wielomianem:

\begin{equation}
    U(\xi) = \frac{1}{2}k\xi^2 - \frac{1}{3}s\xi^3
\end{equation}

gdzie $\xi = x - x_0$, a asymetria potencjału jest reprezentowana przez człon trzeciego stopnia.

Ze względu na asymetrię potencjału, wraz ze wzrostem temperatury (a co za tym idzie - energii drgań atomów) średnie położenie atomów przesuwa się w stronę większych odległości międzyatomowych, co makroskopowo obserwujemy jako rozszerzalność cieplną.

Liniową rozszerzalność cieplną opisuje wzór:

\begin{equation}
    \Delta L = L_0 \alpha (T - T_0)
\end{equation}

gdzie:
\begin{itemize}
    \item $\Delta L$ - zmiana długości ciała
    \item $L_0$ - początkowa długość ciała w temperaturze $T_0$
    \item $\alpha$ - współczynnik rozszerzalności liniowej [K$^{-1}$]
    \item $T - T_0$ - zmiana temperatury
\end{itemize}

Współczynnik rozszerzalności liniowej $\alpha$ definiuje się jako względną zmianę długości przypadającą na jednostkową zmianę temperatury:

\begin{equation}
    \alpha = \frac{\Delta L}{L_0 \cdot (T - T_0)}
\end{equation}

Wartość współczynnika $\alpha$ jest charakterystyczna dla danego materiału i może służyć do jego identyfikacji. Typowe wartości $\alpha$ dla metali mieszczą się w zakresie od około $10 \cdot 10^{-6}$ do $25 \cdot 10^{-6}$ K$^{-1}$.

Rozszerzalność objętościowa ciał stałych jest związana z rozszerzalnością liniową zależnością:

\begin{equation}
    \beta = 3\alpha
\end{equation}

gdzie $\beta$ jest współczynnikiem rozszerzalności objętościowej.

W celu wyznaczenia współczynnika rozszerzalności liniowej metali wykorzystuje się dylatometr, który umożliwia precyzyjny pomiar przyrostu długości próbki pod wpływem zmiany temperatury.

Wstęp teoretyczny opracowano na podstawie materiałów dodatkowych do ćwiczenia nr~25~\cite{lab-25-wstep}.

% ---------- OPIS DOŚWIADCZENIA ----------
\section{Opis doświadczenia}


Celem doświadczenia było wyznaczenie współczynnika rozszerzalności cieplnej dla czterech różnych rurek metalowych.

\begin{enumerate}
    \item Przygotowano stanowisko pomiarowe z czajnikiem elektrycznym do wytworzenia pary wodnej.

    \item Zmierzono długości początkowe czterech rurek metalowych (oznaczonych kolorami: złota, miedziana, szara i srebrna) w temperaturze pokojowej za pomocą taśmy mierniczej.

    \item Odczytano temperaturę otoczenia.

    \item Umieszczono kolejno każdą rurkę w uchwytach dylatometru i ustawiono wstępne położenie za pomocą śruby regulacyjnej.

    \item Po wytworzeniu pary wodnej, podłączono wąż doprowadzający parę do badanej rurki.

    \item Obserwowano i zapisano wydłużenie rurki wskazywane przez czujnik mikrometryczny.

    \item Pomiary powtórzono dwukrotnie dla każdej rurki.

    \item Odczytano ciśnienie atmosferyczne na barometrze rtęciowym i wyznaczono odpowiadającą mu temperaturę wrzenia wody.

    \item Na podstawie uzyskanych danych obliczono współczynniki rozszerzalności cieplnej dla poszczególnych rurek.

    \item Porównując otrzymane wartości z danymi tablicowymi, zidentyfikowano materiały, z których wykonane były badane rurki.
\end{enumerate}



% ---------- OPRACOWANIE WYNIKÓW POMIARÓW ----------
\section{Opracowanie wyników pomiarów}

% ---------- TABELE ----------
\subsection{Tabele pomiarowe}


\begin{table}[H]
    \centering
    \begin{tabular}{|c|c|}
        \hline
        \textbf{Nr} & \textbf{Kolor} \\
        \hline
        1 & złota \\
        \hline
        2 & miedziana \\
        \hline
        3 & szara \\
        \hline
        4 & srebrna \\
        \hline
    \end{tabular}
    \caption{Oznaczenie rurek}
\end{table}

\begin{table}[H]
    \centering
    \begin{tabular}{|l|c|c|c|c|}
        \hline
        Nr & $L_1$ [cm] & $L_2$ [cm] & $L_3$ [cm] & $L_4$ [cm] \\
        \hline
        1 & 75,5 & 75,5 & 75,5 & 75,5 \\
        2 & 75,5 & 75,5 & 75,5 & 75,5 \\
        3 & 75,5 & 75,5 & 75,5 & 75,5 \\
        4 & 75,5 & 75,5 & 75,5 & 75,5 \\
        5 & 75,5 & 75,5 & 75,5 & 75,5 \\
        6 & 75,5 & 75,5 & 75,5 & 75,5 \\
        \hline
    \end{tabular}
    \caption{Początkowe długości rurki.}
\end{table}

\begin{table}[H]
    \centering
    \begin{tabular}{|l|c|c||c|c||c|c||c|c|}
        \hline
        Nr & $L_{11}$ & $L_{12}$ & $L_{21}$ & $L_{22}$ & $L_{31}$ & $L_{32}$ & $L_{41}$ & $L_{42}$ \\
        & [mm] & [mm] & [mm] & [mm] & [mm] & [mm] & [mm] & [mm] \\
        \hline
        1 & 8,87 & 9,93 & 8,96 & 9,91 & 9,13 & 10,07 & 8,88 & 9,56 \\
        2 & 8,94 & 10,01 & 8,90 & 9,85 & 8,93 & 9,87 & 8,84 & 9,49 \\
        \hline
    \end{tabular}
    \caption{Pomiary długości rurki $x$ przed ($L_{x1}$) i po ($L_{x2}$) ogrzaniu.}
\end{table}

\begin{table}[H]
    \centering
    \begin{tabular}{|c|c|c|c|c|}
        \hline
        Nr & $\Delta L_1$ [mm] & $\Delta L_2$ [mm] & $\Delta L_3$ [mm] & $\Delta L_4$ [mm] \\
        \hline
        1 & 1,06 & 0,95 & 0,94 & 0,68 \\
        2 & 1,07 & 0,95 & 0,94 & 0,65 \\
        \hline
    \end{tabular}
    \caption{Pomiary wydłużenia rurki ($\Delta L_x = L_{x2} - L_{x1}$).}
    \label{tab:pomiary_wydluzenia}
\end{table}



% ---------- OBLICZENIA ----------
\subsection{Średnia wartość początkowej długości rurki}

Długości początkowe rurek nie wykazały rozrzutu wszystkie pomiary wyniosły $0.755\,\text{m}$.

\subsection{Średnia wartość wydłużenia rurki}

Na podstawie tabeli \ref{tab:pomiary_wydluzenia} obliczono średnie wydłużenie rurki dla każdej z czterech rurek i zapisano w tabeli \ref{tab:srednie_wydluzenie}.

\begin{table}[H]
    \centering
    \begin{tabular}{|c|c|c|c|}
        \hline
        $\Delta L_1$ [m] & $\Delta L_2$ [m] & $\Delta L_3$ [m] & $\Delta L_4$ [m] \\
        \hline
        0,001065 & 0,000950 & 0,000940 & 0,000665 \\
        \hline
    \end{tabular}
    \caption{Średnie wydłużenie rurki.}
    \label{tab:srednie_wydluzenie}
\end{table}

Przykładowe obliczenia:

\begin{equation*}
    \Delta L_1 = \frac{1,06 + 1,07}{2} \cdot 10^{-3} = 0,001065\,\text{m}
\end{equation*}

\subsection{Temperatura wrzenia wody}

Zmierzone ciśnienie atmosferyczne wyniosło $p = 744.2 \text{mmHg} = 99218{,}52\,\text{Pa}$. Dla tego określenia korzystając ze strony \cite{temperatura_wrzenia} określono temperaturę wrzenia wody jako $T_w = 99{,}4^\circ\text{C}$.


\subsection{Współczynnik rozszerzalności cieplnej}

Współczynnik rozszerzalności cieplnej obliczono na podstawie wzoru:

\begin{equation}
    \label{eq:wspolczynnik_rozszerzalnosci}
    \alpha = \frac{\Delta L}{L_0 \cdot (T_2 - T_1)}
\end{equation}

gdzie:
\begin{itemize}
    \item $\Delta L$ - średnie wydłużenie rurki
    \item $L_0 = 0.755\,\text{m}$ - długość początkowa rurki
    \item $T_1 = 24.0^\circ\text{C}$ - temperatura początkowa
    \item $T_2 = 99{,}4^\circ\text{C}$ - temperatura końcowa
\end{itemize}

Wyniki obliczeń zapisano w tabeli \ref{tab:wspolczynnik_rozszerzalnosci}.

\begin{table}[H]
    \centering
    \begin{tabular}{|c|c|}
        \hline
        Rurka & $\alpha$ [$\frac{1}{\text{K}}$] \\
        \hline
        1 & $18{,}708$ $\cdot 10^{-6}$ \\
        \hline
        2 & $16{,}688$ $\cdot 10^{-6}$ \\
        \hline
        3 & $16{,}512$ $\cdot 10^{-6}$ \\
        \hline
        4 & $11{,}682$ $\cdot 10^{-6}$ \\
        \hline
    \end{tabular}
    \caption{Współczynnik rozszerzalności cieplnej dla poszczególnych rurek.}
    \label{tab:wspolczynnik_rozszerzalnosci}
\end{table}

Przykładowe obliczenia:

\begin{equation}
    \alpha_1 = \frac{0{,}001065}{0{,}755 \cdot (99{,}4 - 24{,}0)} = 0{,}000018708\,\frac{1}{\text{K}}
\end{equation}

\subsection{Określenie materiału rurek}

Porównując wartości współczynnika rozszerzalności cieplnej z wartościami tablicowymi określono materiał rurek (biorąc pod uwagę najbliższą wartość oraz zgodność z kolorami rurek), który został zapisany w tabeli \ref{tab:materiały}.

\begin{table}[H]
    \centering
    \begin{tabular}{|c|c|c|}
        \hline
        \textbf{Materiał} & \multicolumn{2}{c|}{\textbf{Współczynnik rozszerzalności cieplnej [$\frac{1}{\text{K}}$]}} \\
        \cline{2-3}
        & \textbf{Wartość tablicowa} & \textbf{Wartość zmierzona} \\
        \hline
        Mosiądz & $19{,}0$ $\cdot 10^{-6}$ & $18{,}708$ $\cdot 10^{-6}$ \\
        \hline
        Miedź & $17{,}0$ $\cdot 10^{-6}$ & $16{,}688$ $\cdot 10^{-6}$ \\
        \hline
        Stal nierdzewna & $17{,}3$ $\cdot 10^{-6}$ & $16{,}512$ $\cdot 10^{-6}$ \\
        \hline
        Żelazo & $11{,}8$ $\cdot 10^{-6}$ & $11{,}682$ $\cdot 10^{-6}$ \\
        \hline
    \end{tabular}
    \caption{Tablicowe wartości współczynnika rozszerzalności cieplnej dla różnych materiałów (źródło: \cite{wspolczynnik_rozszeralnosci}) oraz wartości zmierzone.}
    \label{tab:materiały}
\end{table}

\section{Ocena niepewności pomiaru}


\subsection{Niepewności wzorcowania}

Niepewności wzorcowania na podstawie użytych przyrządów zostały zapisane w tabeli \ref{tab:niepewnosci}.

\begin{table}[H]
    \centering
    \begin{tabular}{|c|c|}
        \hline
        \textbf{Wielkość} & \textbf{Niepewność wzorcowania} \\
        \hline
        Długość początkowa rurki ($\Delta_d L$) & $0.01\,\text{m}$ \\
        \hline
        Długość rurki (przed i po ogrzaniu) ($\Delta_d L_{xx}$) & $0.00001\,\text{m}$ \\
        \hline
        Temperatura ($\Delta_d T$) & $0.1^\circ\text{C}$ \\
        \hline
    \end{tabular}
    \caption{Niepewności wzorcowania.}
    \label{tab:niepewnosci}
\end{table}

\subsection{Niepewność standardowa początkowej długości rurki}

Niepewność standardowa początkowej długości rurki obliczono na podstawie wzoru na niepewność typu B, ze względu na to, że nie wystąpił rozrzut wyników pomiarów.

\begin{equation*}
    u(L_0) = \frac{\Delta_d L_0}{\sqrt{3}} = \frac{0.01}{\sqrt{3}} = 0.00578\,\text{m}
\end{equation*}

\subsection{Niepewność standardowa długości rurki}

Niepewność standardowa długości rurki obliczono na podstawie wzoru:

\begin{equation*}
    u(L_{xx}) = \frac{\Delta_d L_{xx}}{\sqrt{3}} = \frac{0.00001}{\sqrt{3}} = 5.8 \cdot 10^{-6}\,\text{m}
\end{equation*}

\subsection{Niepewność wydłużenia rurki}

Wydłużenie rurki określa wzór:

\begin{equation*}
    \Delta L_{xx} = L_{x2} - L_{x1}
\end{equation*}

Stąd niepewność standardowa wydłużenia rurki wynosi z prawa przenoszenia niepewności:

\begin{equation*}
    u(\Delta L_{xx}) = 2 u(L_{xx}) = 2 \cdot 5{,}8 \cdot 10^{-6} = 1{,}2 \cdot 10^{-5}\,\text{m}
\end{equation*}

\subsection{Niepewność standardowa temperatury}

Niepewność standardowa temperatury obliczono na podstawie wzoru:

\begin{equation*}
    u(T) = \frac{\Delta_d T}{\sqrt{3}} = \frac{0{,}1}{\sqrt{3}} = 0{,}058\,\text{K}
\end{equation*}


\subsection{Niepewność standardowa różnicy temperatur}

Niepewność standardowa różnicy temperatur obliczono na podstawie wzoru:

\begin{equation*}
    u(T_2 - T_1) = 2 \cdot u(T) = 2 \cdot 0{,}058 = 0{,}12\,\text{K}
\end{equation*}


\subsection{Niepewność standardowa współczynnika rozszerzalności cieplnej}

Współczynnik rozszerzalności cieplnej obliczono na podstawie wzoru \ref{eq:wspolczynnik_rozszerzalnosci_obliczenie}.

\begin{equation*}
    \label{eq:wspolczynnik_rozszerzalnosci_obliczenie}
    \alpha = \frac{\Delta L}{L_0 \cdot \Delta T}
\end{equation*}

Stąd korzystając z prawa przenoszenia niepewności otrzymujemy:

\begin{equation*}
    u(\alpha) = \sqrt{\left( \frac{\partial \alpha}{\partial \Delta L} \right)^2 u^2(\Delta L) + \left( \frac{\partial \alpha}{\partial L_0} \right)^2 u^2(L_0) + \left( \frac{\partial \alpha}{\partial \Delta T} \right)^2 u^2(\Delta T)}
\end{equation*}

Obliczając pochodne cząstkowe otrzymujemy:

\begin{align*}
    \frac{\partial \alpha}{\partial \Delta L} & = \frac{1}{L_0 \cdot \Delta T}           \\
    \frac{\partial \alpha}{\partial L_0}      & = -\frac{\Delta L}{L_0^2 \cdot \Delta T} \\
    \frac{\partial \alpha}{\partial \Delta T} & = -\frac{\Delta L}{L_0 \cdot \Delta T^2}
\end{align*}

Ostatecznie otrzymujemy:

\begin{equation*}
    u(\alpha) = \sqrt{\left( \frac{1}{L_0 \cdot \Delta T} \right)^2 u^2(\Delta L) + \left( -\frac{\Delta L}{L_0^2 \cdot \Delta T} \right)^2 u^2(L_0) + \left( -\frac{\Delta L}{L_0 \cdot \Delta T^2} \right)^2 u^2(\Delta T)}
\end{equation*}

Dla wszystkich rurek obliczono wartości i zapisano w tabeli \ref{tab:niepewnosci_wspolczynnik_rozszerzalnosci}.

\begin{table}[H]
    \centering
    \begin{tabular}{|c|c|}
        \hline
        Rurka & $u(\alpha)$ [$\cdot 10^{-6}\,\frac{1}{\text{K}}$] \\
        \hline
        1 & 0,21 \\
        2 & 0,20 \\
        3 & 0,20 \\
        4 & 0,20 \\
        \hline
    \end{tabular}
    \caption{Niepewność standardowa współczynnika rozszerzalności cieplnej.}
    \label{tab:niepewnosci_wspolczynnik_rozszerzalnosci}
\end{table}

Przykładowe obliczenia dla rurki nr 1:

\begin{align*}
    u(\alpha_1) & = \sqrt{\left(\frac{1}{0{,}755 \cdot 75{,}4}\right)^2 \cdot (1{,}2 \cdot 10^{-5})^2 +}  \\
                & \quad \left(-\frac{0{,}001065}{(0{,}755)^2 \cdot 75{,}4}\right)^2 \cdot (0{,}00578)^2 + \\
                & \quad \left(-\frac{0{,}001065}{0{,}755 \cdot (75{,}4)^2}\right)^2 \cdot (0{,}0578)^2 =  \\
                & = 0{,}21 \cdot 10^{-6} \cdot \text{K}^{-1}
\end{align*}

% ---------- WNIOSKI ----------
\section{Wnioski}

W przeprowadzonym doświadczeniu wyznaczono współczynniki rozszerzalności cieplnej dla czterech różnych rurek metalowych za pomocą dylatometru. Na podstawie uzyskanych wyników można sformułować następujące wnioski:

\begin{enumerate}

    \item Na podstawie wyznaczonych współczynników rozszerzalności oraz wyglądu rurek zidentyfikowano następujące materiały:
          \begin{itemize}
              \item Rurka 1 (złota) - mosiądz ($\alpha = 18,708 \cdot 10^{-6}$ K$^{-1}$, $u_c(\alpha) = 0{,}21 \cdot 10^{-6}$ K$^{-1}$)
              \item Rurka 2 (miedziana) - miedź ($\alpha = 16,688 \cdot 10^{-6}$ K$^{-1}$, $u_c(\alpha) = 0{,}20 \cdot 10^{-6}$ K$^{-1}$)
              \item Rurka 3 (szara) - stal nierdzewna ($\alpha = 16,512 \cdot 10^{-6}$ K$^{-1}$, $u_c(\alpha) = 0{,}20 \cdot 10^{-6}$ K$^{-1}$)
              \item Rurka 4 (srebrna) - żelazo ($\alpha = 11,682 \cdot 10^{-6}$ K$^{-1}$, $u_c(\alpha) = 0{,}20 \cdot 10^{-6}$ K$^{-1}$)
          \end{itemize}

    \item Zaobserwowano wyraźne różnice w rozszerzalności cieplnej różnych metali - mosiądz charakteryzuje się najwyższym współczynnikiem rozszerzalności, podczas gdy żelazo najniższym spośród badanych materiałów.

    \item Niepewności pomiarowe były na poziomie $0{,}20 \cdot 10^{-6}$ K$^{-1}$.

    \item Największy wpływ na niepewność wyznaczonego współczynnika rozszerzalności miały niepewności pomiaru długości początkowej rurki oraz niepewność pomiaru temperatury.

    \item Dzięki zastosowaniu pary wodnej do podgrzewania rurek, zapewniono ogrzewanie rurek do stałej temperatury, co pozwoliło na uzyskanie powtarzalnych wyników - dla obu powtórzeń wyniki były zbliżone.
\end{enumerate}

Przeprowadzone doświadczenie potwierdza, że współczynnik rozszerzalności cieplnej jest charakterystyczną cechą materiału, która może służyć do jego identyfikacji.

% ---------- WYKRESY ----------
% \section{Wykresy}

\bibliographystyle{plain}
\bibliography{bibliography}

\end{document}
