\documentclass[a4paper,12pt]{article}
\usepackage[left=2cm,right=2cm,top=2cm,bottom=2cm]{geometry} % Do ustawień marginesów
\usepackage{multicol} % Dla podziału na kolumny
\usepackage{ragged2e} % Dla justowania tekstu
\usepackage{graphicx} % Required for inserting images
\usepackage{float}
\usepackage{caption}
\usepackage{amsmath} % Math formulas
\usepackage{amssymb} % Symbols
\usepackage[svgnames]{xcolor}
\usepackage[colorlinks=true, urlcolor=blue, linkcolor=black, citecolor=orange]{hyperref} % Hyperlinks
\usepackage{polski} % Polish language
\usepackage[utf8]{inputenc} % Text encoding
\usepackage{enumitem} % Pakiet do elastycznego sterowania listami
\usepackage{indentfirst}
\usepackage{array}
\usepackage{longtable}
\usepackage{pdflscape}
\usepackage[round]{natbib}
\setlist[itemize]{itemsep=0pt, topsep=0pt}
\usepackage{siunitx}
\sisetup{output-decimal-marker={,}}
\sisetup{exponent-product = \cdot}
% LTeX: language=pl-PL

\begin{document}

% Górna część strony
\noindent
\begin{minipage}{0.5\textwidth}
    \raggedright
    \textbf{Piotr Durniat, 347264} \\
    II rok, Fizyka \\
    Wtorek, 8:00-10:15
    \vspace{0.5cm}
\end{minipage}
\begin{minipage}{0.5\textwidth}
    \raggedleft
    28.10.2025 \\
    \vspace{0.5cm}
    Prowadząca: \\
    dr Sylwia Owczarek
\end{minipage}

% Tytuł ćwiczenia
\vspace{2cm}
\begin{center}
    \LARGE \textbf{Ćwiczenie nr 65} \\[0.5cm]
    \Large \textbf{Wyznaczanie promienia krzywizny soczewki za pomocą pierścieni Newtona}
\end{center}

% Reszta treści
\vspace{1cm} % Kolejny odstęp
\noindent

% \tableofcontents
% \newpage

% ---------- WSTĘP TEORETYCZNY ----------
\section{Wstęp teoretyczny}

\subsection*{Interferencja światła i spójność}
Interferencja jest zjawiskiem charakterystycznym dla ruchów falowych, obserwowanym, gdy fale nakładają się na siebie. Zjawisko to jest ściśle związane z pojęciem spójności (koherencji) fal. Gdy fale pochodzące ze spójnych źródeł (np. drgających ze stałą częstością) interferują, powstaje regularna struktura miejsc, w których drgania się wygaszają i wzmacniają~\citep{Drynski1976}.

\subsection*{Warunki interferencji i prążki jednakowej grubości}
Wzmocnienie (maksimum) występuje, gdy różnica dróg optycznych interferujących promieni jest równa całkowitej wielokrotności długości fali (\(k\lambda\)), a wygaszenie (minimum) dla nieparzystej wielokrotności połówek długości fali (\((2k+1)\frac{\lambda}{2}\)). Należy uwzględnić ewentualną zmianę fazy o \(\pi\) (odpowiadającą zmianie drogi optycznej o \(\lambda/2\)) przy odbiciu od ośrodka optycznie gęstszego~\citep{Drynski1976}.

W przypadku oświetlenia cienkiej płytki o zmiennej grubości światłem rozciągłym, obserwuje się prążki interferencyjne umiejscowione na powierzchni płytki, biegnące przez punkty o jednakowej grubości. Są to tzw. \textbf{krzywe jednakowej grubości}~\citep{Drynski1976}.

\subsection*{Pierścienie Newtona}

Najdogodniejszym sposobem uzyskania regularnych prążków jednakowej grubości jest użycie \textbf{zestawu Newtona}. Składa się on z płasko-równoległej płytki szklanej i soczewki płasko-wypukłej o dużym promieniu krzywizny, położonej wypukłą stroną na płytce. Między soczewką a płytką tworzy się klin powietrzny o grubości \(d\) rosnącej wraz z odległością \(r\) od punktu styku~\citep{Drynski1976}.

Gdy układ jest oświetlony prostopadle światłem jednorodnym (monochromatycznym), interferencja zachodzi między promieniami odbitymi od dolnej powierzchni soczewki i od górnej powierzchni płytki szklanej. Ponieważ grubość klina powietrznego jest stała wzdłuż okręgu o środku w punkcie styku, powstają koncentryczne prążki interferencyjne zwane \textbf{pierścieniami Newtona}. W centrum obserwuje się ciemny prążek (\(k=0\)), gdyż przy odbiciu od płytki szklanej (ośrodek gęstszy) następuje zmiana fazy o \(\pi\)~\citep{Drynski1976}.

Dla padania prostopadłego (\(\cos\beta \approx 1\)), warunek na \(k\)-ty ciemny pierścień (minimum), uwzględniający zmianę fazy przy odbiciu od płytki, ma postać:
$$
    2nd = k\lambda
$$
Dla klina powietrznego (\(n=1\)):
$$
    2d_k = k\lambda
$$
gdzie \(d_k\) to grubość warstwy powietrza dla \(k\)-tego ciemnego pierścienia (\(k=0, 1, 2...\))~\citep{Drynski1976}.

\subsection*{Zasada pomiaru promienia krzywizny soczewki}
Grubość warstwy powietrza \(d\) w odległości \(r\) od punktu styku można powiązać z promieniem krzywizny \(R\) soczewki. Z geometrii układu (rys. 157 w \citep{Drynski1976}) wynika zależność \(2Rd - d^2 = r^2\). Ponieważ \(d\) jest bardzo małe w porównaniu z \(R\), można pominąć człon \(d^2\), co daje przybliżony wzór:
$$
    d \approx \frac{r^2}{2R}
$$
Podstawiając to do warunku na \(k\)-ty ciemny pierścień (\(2d_k = k\lambda\)), otrzymujemy:
$$
    2 \frac{r_k^2}{2R} = k\lambda
$$
$$
    r_k^2 = k R \lambda
$$
gdzie \(r_k\) to promień \(k\)-tego ciemnego pierścienia. Stąd można wyznaczyć promień krzywizny soczewki \(R\):
~\citep{Drynski1976}.
$$
    R = \frac{r_k^2}{k\lambda}
$$

% ---------- OPIS DOŚWIADCZENIA ----------
% \section{Opis doświadczenia}

% ---------- OPRACOWANIE WYNIKÓW POMIARÓW ----------
\section{Opracowanie wyników pomiarów}

% ---------- TABELE ----------
\subsection{Tabele pomiarowe}

Dla wybranych numerów prążków $n \in [2,6]$ (licząc od środkowego prążka) zmierzono położenia ich lewego, prawego, dolnego i górnego brzegu (oznaczone odpowiednio $x_{min}$, $x_{max}$, $y_{min}$ i $y_{max}$). Ze względu na niezerową grubość prążków, pomiary dokonywano do ich, w osi $X$ -- lewej i w osi $Y$ -- dolnej krawędzi. Pomiary zostały zapisane w tabeli \ref{tab:measurements}.

\begin{table}[H]
    \centering
    \begin{tabular}{|r|r|r|r|r|}
        \hline
        \textbf{$n$} & \textbf{$x_{\max}\,[\SI{e-3}{\meter}]$} & \textbf{$x_{\min}\,[\SI{e-3}{\meter}]$} & \textbf{$y_{\max}\,[\SI{e-3}{\meter}]$} & \textbf{$y_{\min}\,[\SI{e-3}{\meter}]$}
        \\ \hline
        2 & \num{23.62} & \num{19.89} & \num{7.92} & \num{4.23} \\ \hline
        3 & \num{24.21} & \num{19.34} & \num{8.54} & \num{3.70} \\ \hline
        4 & \num{24.58} & \num{18.78} & \num{8.89} & \num{3.23} \\ \hline
        5 & \num{25.09} & \num{18.44} & \num{9.39} & \num{2.86} \\ \hline
        6 & \num{25.38} & \num{18.04} & \num{9.72} & \num{2.46} \\ \hline
    \end{tabular}
    \caption{Położenia lewej, prawej, dolnej i górnej krawędzi prążków.}
    \label{tab:measurements}
\end{table}

% ---------- OBLICZENIA ----------
\subsection{Średnie promienie prążków}

Dla każdego prążka z tabeli \ref{tab:measurements} wyznaczono średnicę w kierunku $X$: $D_{X,\,n} = x_{max,\,n} - x_{min,\,n}$ i w kierunku $Y$: $D_{Y,\,n} = y_{max,\,n} - y_{min,\,n}$. Następnie obliczono średni promień $r_{n}$ jako połowę średniej arytmetycznej tych średnic \eqref{eq:avg_radius}. Wyniki zapisano w tabeli \ref{tab:radii}:
\begin{equation}
    \label{eq:avg_radius}
    r_{n} = \frac{D_{X,\,n} + D_{Y,\,n}}{4},
\end{equation}
gdzie indeks $n$ określa numer prążka.

\begin{table}[H]
    \centering
    \begin{tabular}{|c|r|r|r|}
        \hline
        \textbf{$n$} & \textbf{$D_{X}\,[\SI{e-3}{\meter}]$} & \textbf{$D_{Y}\,[\SI{e-3}{\meter}]$} & \textbf{$r_{n}\,[\SI{e-3}{\meter}]$} \\
        \hline
        \num{2} & \num{3.73} & \num{3.69} & \num{1.8550} \\ \hline
        \num{3} & \num{4.87} & \num{4.84} & \num{2.4275} \\ \hline
        \num{4} & \num{5.80} & \num{5.66} & \num{2.8650} \\ \hline
        \num{5} & \num{6.65} & \num{6.53} & \num{3.2950} \\ \hline
        \num{6} & \num{7.34} & \num{7.26} & \num{3.6500} \\ \hline
    \end{tabular}
    \caption{Średnice (pozioma i pionowa) oraz średnie promienie prążków.}
    \label{tab:radii}
\end{table}


\subsubsection*{Przykładowe obliczenia}
Dla $n=2$:
\begin{align*}
    D_{X,\,2} & = (\num{23.62} - \num{19.89})\cdot \num{e-3} = \SI{3.73e-3}{\meter}           \\
    D_{Y,\,2} & = (\num{7.92} - \num{4.23})\cdot \num{e-3}= \SI{3.69e-3}{\meter}              \\
    r_{2}     & = \frac{(\num{3.73} + \num{3.69})\cdot \num{e-3}}{4} = \SI{1.8550e-3}{\meter}
\end{align*}

\subsection{Promień krzywizny soczewki}

Na podstawie wartości z tab. \ref{tab:radii} wyznaczono promienie krzywizny soczewki $R_i$ (dla każego $i$-tego pomiaru) ze wzoru~\eqref{eq:lens_radius}. Wartości zapisano w tabeli~\ref{tab:radii_and_curvature}.

\begin{equation}
    \label{eq:lens_radius}
    R_n = \frac{r_n^2}{n \cdot \lambda},
\end{equation}
gdzie:
\begin{itemize}
    \item $n$ - numer prążka,
    \item $r_n$ - promień $n$-tego prążka,
    \item $\lambda = \SI{589e-9}{\meter}$ - długość fali dla lampy sodowej.
\end{itemize}

\begin{table}[H]
    \centering
    \begin{tabular}{|c|r|}
        \hline
        \textbf{$n$} & \textbf{$R_n\,[\si{\meter}]$} \\ \hline
        \num{2} & \num{2.921} \\ \hline
        \num{3} & \num{3.335} \\ \hline
        \num{4} & \num{3.484} \\ \hline
        \num{5} & \num{3.687} \\ \hline
        \num{6} & \num{3.770} \\ \hline
    \end{tabular}
    \caption{Promienie krzywizny dla każdego prążka.}
    \label{tab:radii_and_curvature}
\end{table}

Ostateczna wartość promienia krzywizny $\bar R$ została obliczona jako średnia arytmetyczna promieni krzywizny dla wszystkich prążków \eqref{eq:avg_lens_curv} i wyniosła $\SI{3.439}{\meter}$.
\begin{equation}
    \label{eq:avg_lens_curv}
    \bar R =  \frac{1}{5}\sum\limits_{n=2}^{6} R_n
\end{equation}

\subsubsection*{Przykładowe obliczenia}

\begin{align*}
    R_2     & = \frac{(\num{1.8550e-3})^2}{2 \cdot \num{589e-9}}  \approx \SI{2.921}{\meter}                             \\
    \bar{R} & = \frac{\num{2.921} + \num{3.335} + \num{3.484} + \num{3.687} + \num{3.770}}{5} \approx \SI{3.439}{\meter}
\end{align*}

% ---------- NIEPEWNOŚCI ----------
\section{Ocena niepewności pomiaru}

% ---------- WNIOSKI ----------
% \section{Wnioski}

% ---------- WYKRESY ----------
% \section{Wykresy}

\bibliographystyle{apalike}
\bibliography{bibliography}

\end{document}
