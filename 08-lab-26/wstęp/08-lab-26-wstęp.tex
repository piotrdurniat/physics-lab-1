\documentclass[12pt,a4paper]{article}
\usepackage[utf8]{inputenc}
\usepackage[T1]{fontenc}
\usepackage{polski}
\usepackage{amsmath}
\usepackage{graphicx}
\usepackage{booktabs}
\usepackage{float}

\begin{document}

\section{Wstęp teoretyczny}

Ciepło właściwe substancji $c$ określa ilość energii potrzebnej do podwyższenia temperatury jednostkowej masy ciała o jednostkę temperatury. Jest ono definiowane jako:

\begin{equation}
    c = \frac{Q}{m \cdot \Delta T}
\end{equation}

gdzie $Q$ to dostarczona energia cieplna, $m$ to masa ciała, a $\Delta T$ to zmiana temperatury.

W doświadczeniu wykorzystujemy kalorymetr, który pozwala na pomiar ciepła właściwego ciał stałych. Metoda opiera się na zasadzie bilansu cieplnego, zgodnie z którą suma ciepła oddanego i pobranego w układzie izolowanym jest równa zeru:

\begin{equation}
    Q_1 + Q_2 = 0
\end{equation}

gdzie $Q_1$ to ciepło oddane przez ciało o wyższej temperaturze (wartość ujemna), a $Q_2$ to ciepło pobrane przez ciało o niższej temperaturze (wartość dodatnia).

Dla badanego ciała stałego o masie $m_c$, temperaturze początkowej $T_c$ i cieple właściwym $c_p$, które zostaje umieszczone w wodzie o masie $m_w$, temperaturze początkowej $T_p$ i cieple właściwym $c_w$, przy uwzględnieniu pojemności cieplnej naczynka kalorymetrycznego $K_n = m_n \cdot c_n$, bilans cieplny przyjmuje postać:

\begin{equation}
    m_c \cdot c_p \cdot (T_k - T_c) + [m_w \cdot c_w + m_n \cdot c_n] \cdot (T_k - T_p) = 0
\end{equation}

gdzie $T_k$ to temperatura końcowa układu.

Przekształcając powyższe równanie, otrzymujemy wzór na ciepło właściwe badanego ciała:

\begin{equation}
    c_p = \frac{[m_w \cdot c_w + m_n \cdot c_n] \cdot (T_p - T_k)}{m_c \cdot (T_k - T_c)}
\end{equation}

Ponieważ $T_p < T_k < T_c$, wyrażenie $(T_p - T_k)$ jest ujemne, a $(T_k - T_c)$ również ujemne, co daje w rezultacie dodatnią wartość $c_p$.

Prawo Dulonga-Petita stanowi, że molowe ciepło właściwe pierwiastków stałych w temperaturze pokojowej jest w przybliżeniu stałe i wynosi około $3R \approx 25$ J/(mol·K), gdzie $R$ to stała gazowa. Prawo to jest przybliżeniem i sprawdza się głównie dla metali i prostych substancji krystalicznych w temperaturze pokojowej.

W rzeczywistym przebiegu doświadczenia występuje wymiana ciepła z otoczeniem, co wprowadza błąd systematyczny. Aby go zminimalizować, stosuje się metodę interpolacji do wyznaczenia rzeczywistych temperatur początkowej i końcowej, analizując zmiany temperatury w czasie przed i po osiągnięciu stanu równowagi.

\end{document}
