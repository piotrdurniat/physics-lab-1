\documentclass{article}
\usepackage{graphicx} % Required for inserting images
\usepackage[margin=2cm]{geometry} % Add smaller margins
\usepackage[utf8]{inputenc}
\usepackage[T1]{fontenc}
\usepackage{polski}
\usepackage{amsmath}
\usepackage{graphicx}
\usepackage{booktabs}
\usepackage{float}


\title{Wstęp cw 26}
\author{}
\date{April 2025}

\begin{document}

\maketitle

\section*{Wstęp teoretyczny}

\subsection*{Pojęcia ciepła i temperatury}

\textbf{Ciepło} jest formą przekazywania energii wewnętrznej pomiędzy układami w wyniku różnicy temperatur. W przeciwieństwie do pracy, ciepło nie jest związane z uporządkowanym ruchem makroskopowym cząstek, lecz z mikroskopowymi procesami wewnętrznymi.
\textbf{Temperatura} jest skalarną wielkością fizyczną określającą stan termiczny ciała i jest miarą średniej energii kinetycznej ruchu termicznego cząsteczek tworzących substancję. Jednostką temperatury w układzie SI jest kelwin (K).

\subsection*{Zasada zachowania energii w odniesieniu do ciepła. Bilans cieplny}

Zasada zachowania energii głosi, że energia nie może być tworzona ani niszczona — może jedynie zmieniać formę.
W kontekście procesów cieplnych oznacza to, że suma przekazanych ilości ciepła między wszystkimi składnikami układu izolowanego musi wynosić zero:
\[
    \sum_i Q_i = 0
\]
gdzie \( Q_i \) to ilość ciepła wymieniona przez \( i \)-ty składnik układu.

Dla układu składającego się z badanego ciała, wody i kalorymetru bilans cieplny przyjmuje postać:
\[
    m_c c_c (T_c - T_k) + m_w c_w (T_p - T_k) + m_k c_k (T_p - T_k) = 0
\]
gdzie:
\( m \) — masa,
\( c \) — ciepło właściwe,
\( T \) — temperatura,
a indeksy \( c \), \( w \), \( k \) odpowiadają kolejno badanemu ciału, wodzie i kalorymetrowi, natomiast \( T_p \) i \( T_k \) oznaczają temperaturę początkową i końcową.

\subsection*{Kalorymetria}

\textbf{Kalorymetria} to dział fizyki zajmujący się pomiarem ilości ciepła wymienianego podczas procesów fizycznych i chemicznych. Podstawowym urządzeniem wykorzystywanym w tych pomiarach jest \textbf{kalorymetr} — naczynie zapewniające minimalną wymianę ciepła z otoczeniem. Uwzględnienie pojemności cieplnej kalorymetru jest niezbędne dla poprawnego przeprowadzenia bilansu cieplnego.

\subsection*{Pojemność cieplna ciał stałych, ciepło właściwe i jego zależność od temperatury}

\textbf{Pojemność cieplna} \( C \) ciała określa ilość ciepła potrzebną do zmiany temperatury całego ciała o jednostkę temperatury:
\[
    C = \frac{Q}{\Delta T}
\]
\textbf{Ciepło właściwe} \( c \) definiuje się jako ilość ciepła potrzebną do zmiany temperatury jednostki masy substancji o jednostkę temperatury:
\[
    c = \frac{Q}{m \Delta T}
\]
Ciepło właściwe ciał stałych w ogólności zależy od temperatury, zwłaszcza w zakresie niskich temperatur (kilka–kilkadziesiąt K). W temperaturze pokojowej dla większości metali ciepło właściwe można jednak traktować jako stałe.

\subsection*{Prawo Dulonga i Petita}

\textbf{Prawo Dulonga i Petita} stwierdza, że dla większości ciał stałych w temperaturze pokojowej molowa pojemność cieplna wynosi około:
\[
    C_m \approx 3R
\]
gdzie \( C_m \) to molowa pojemność cieplna, a \( R \) to stała gazowa (\( R \approx 8{,}314\, \mathrm{J\,mol^{-1}K^{-1}} \)).
Przekłada się to na przybliżoną wartość ciepła właściwego metali:
\[
    c \approx \frac{3R}{M}
\]
gdzie \( M \) to masa molowa substancji. Prawo to wynika z klasycznego modelu drgań atomów w sieci krystalicznej i obowiązuje w temperaturach dużo wyższych od temperatury Debye'a.

\subsection*{Idealny i rzeczywisty przebieg wymiany ciepła w kalorymetrze — metoda interpolacji}

W idealnym przypadku wymiana ciepła między ciałem a wodą zachodzi natychmiastowo, bez strat do otoczenia. W rzeczywistości proces trwa pewien czas i w jego trakcie zachodzi częściowa wymiana ciepła z otoczeniem oraz zmiana temperatury wody i ciała.

Aby zminimalizować błędy pomiarowe wynikające z nieidealności procesu, stosuje się \textbf{metodę interpolacji}: sporządza się wykres temperatury w funkcji czasu przed wrzuceniem ciała oraz po wrzuceniu, a następnie ekstrapoluje się te przebiegi do chwili zetknięcia (\( t=0 \)), wyznaczając interpolowane wartości temperatury odpowiadające hipotetycznie nieskończenie szybkiemu przebiegowi wymiany ciepła.

\end{document}
