\documentclass[a4paper,12pt]{article}
\usepackage[left=2cm,right=2cm,top=2cm,bottom=2cm]{geometry} 
\usepackage{multicol} 
\usepackage{ragged2e} 
\usepackage{graphicx} 
\usepackage{float}
\usepackage{caption}
\usepackage{amsmath} 
\usepackage{amssymb} 
\usepackage[svgnames]{xcolor}
\usepackage[colorlinks=true, urlcolor=blue, linkcolor=black, citecolor=orange]{hyperref} 
\usepackage{polski} 
\usepackage[utf8]{inputenc} 
\usepackage{enumitem} 
\usepackage{indentfirst}
\usepackage{array}
\usepackage{longtable}
\usepackage{pdflscape}
\usepackage[round]{natbib}
\setlist[itemize]{itemsep=0pt, topsep=0pt}
\usepackage{siunitx}

% SI setup preferences
\sisetup{output-decimal-marker={,}}
\sisetup{exponent-product = \cdot}
\sisetup{per-mode = symbol}

% LTeX: language=pl-PL

\begin{document}

\noindent
\begin{minipage}{0.5\textwidth}
	\raggedright
	% full name, index number
	\textbf{Piotr Durniat, 347264} \\
	II rok, Fizyka \\
	Wtorek, 8:00-10:15
	\vspace{0.5cm}
\end{minipage}
\begin{minipage}{0.5\textwidth}
	\raggedleft
	% date of the experiments
	16.12.2025 \\
	\vspace{0.5cm}
	Prowadząca: \\
	dr Sylwia Owczarek
\end{minipage}

\vspace{2cm}
\begin{center}
	% lab number
	\LARGE \textbf{Ćwiczenie nr 42} \\[0.5cm]
	% title
	\Large \textbf{Mostek Wheatstone'a}
\end{center}

\vspace{1cm}
\noindent

% \tableofcontents
% \newpage

% ---------- WSTĘP TEORETYCZNY ----------
\section{Wstęp teoretyczny}

\subsection{Mostek Wheatstone'a}

Mostek Wheatstone'a to układ czterech przewodników (rezystorów), który pozwala na wyznaczenie wartości nieznanej rezystancji w sposób niewymagający bezpośredniego pomiaru natężenia prądu ani napięcia \citep{Drynski1976}. Układ ten składa się z dwóch równoległych gałęzi zasilanych ze źródła prądu stałego, połączonych poprzeczną gałęzią (mostkiem) zawierającą czuły galwanometr.

Stan równowagi mostka osiągany jest wtedy, gdy przez galwanometr nie płynie prąd. Fizycznie oznacza to, że potencjały w punktach węzłowych mostka są jednakowe, a napięcie między nimi wynosi zero \citep{Drynski1976}. Warunek ten jest spełniony tylko wtedy, gdy cztery rezystancje spełniają następującą proporcję:

\begin{equation}
	\label{eq:wheatstone_proporcja}
	\frac{R_1}{R_2} = \frac{R_3}{R_4}
\end{equation}

Gdy jeden z czterech rezystorów jest nieznany, można wyznaczyć jego rezystancję z powyższej proporcji \eqref{eq:wheatstone_proporcja}, znając wartości pozostałych trzech elementów.

\subsubsection{Liniowy mostek Wheatstone'a}

W liniowym mostku Wheatstone'a rezystancje $R_1$ i $R_2$ stanowią fragmenty drutu oporowego o stałym przekroju $q$ i rezystywności $\rho$, rozpiętego na podstawce wzdłuż skali milimetrowej \citep{Drynski1976}. Zgodnie z drugim prawem Ohma rezystancja takiego przewodu wyraża się wzorem $R = \rho \frac{l}{q}$.

Ponieważ drut jest jednorodny na całej długości, parametry materiałowe i geometryczne upraszczają się, dzięki czemu stosunek rezystancji $\frac{R_1}{R_2}$ można zastąpić bezpośrednio stosunkiem długości odcinków drutu $\frac{l_1}{l_2}$ \citep{Drynski1976}.
W ten sposób dla nieznanej rezystancji $R_x = R_3$ i znanej rezystancji z opornicy $R_4 = R$, można określić wartość $R_x$ jako:

\begin{equation}
	\label{eq:mostek_liniowy}
	R_x = R \frac{l_1}{l_2}
\end{equation}

Najmniejszy błąd pomiaru uzyskuje się w sytuacji, gdy stosunek długości jest bliski jedności ($l_1 \approx l_2$), co odpowiada ustawieniu suwaka w pobliżu środka skali \citep{Drynski1976}.

% ---------- OPIS DOŚWIADCZENIA ----------
\section{Opis doświadczenia}

Pomiary przeprowadzono zgodnie z następującą procedurą:

\begin{itemize}
	\item Zmontowano układ pomiarowy według schematu ideowego, składający się z badanego rezystora $R_x$, znanej opornicy dekadowej $R_w$, rezystancji zabezpieczającej $R_z$, listwy pomiarowej (dzielącej się na odcinki $l_1$ i $l_2$) oraz miliamperomierza służącego do wykrywania prądu w gałęzi mostka.
	\item Zadbano o to, aby napięcie zasilające układ nie przekraczało dopuszczalnej wartości \SI{3}{\volt}.
	\item Dla każdego z pięciu badanych rezystorów ($R_1 - R_5$) znaleziono położenie równowagi mostka, dobierając nastawy opornicy dekadowej tak, aby przez miliamperomierz nie płynął prąd.
	\item Pomiary dla każdego elementu wykonano przy trzech ustalonych położeniach suwaka na listwie pomiarowej: w $1/3$, $1/2$ oraz $2/3$ jej całkowitej długości.
	\item Analogiczną procedurę pomiarową powtórzono dla wskazanych przez prowadzącego układów połączeń rezystorów (szeregowego i równoległego).
\end{itemize}

% ---------- OPRACOWANIE WYNIKÓW POMIARÓW ----------
\section{Opracowanie wyników pomiarów}


% ---------- TABELE ----------
\subsection{Tabele pomiarowe}

Zmierzone wartości nastaw opornicy dekadowej $R_w$ dla trzech ustawień suwaka listwy $L$, dla badanych rezystorów $R_1,\,R_2,\,...,\,R_5$ zapisano w tabeli \ref{tab:nastawy_opornicy_dekadowej}. Jako $R_s$ i $R_r$ oznaczono rezystancję zastępczą układu rezystorów 1-3, połączonych odpowiednio szeregowo i równolegle.

\begin{table}[H]
	\centering
	\begin{tabular}{|c|c|c|c|c|c||c|c|}
		\hline
		                     & \multicolumn{7}{c|}{\textbf{Nastawa opornicy } $R_w$ $[\Omega]$ \textbf{dla rezystora:}}                                                                     \\ \cline{2-8}
		\textbf{Długość} $L$ & $R_1$                                                                                    & $R_2$ & $R_3$ & $R_4$ & $R_5$ & $R_r$ (równol.) & $R_s$ (szereg.) \\ \hline
		$1/3$                & 97                                                                                       & 202   & 832   & 2320  & 6100  & 64              & 1060            \\ \hline
		$1/2$                & 52                                                                                       & 90    & 452   & 1340  & 3299  & 32              & 560             \\ \hline
		$2/3$                & 27,6                                                                                     & 49,3  & 232   & 653   & 1720  & 16,4            & 309             \\ \hline
	\end{tabular}
	\caption{Wartości nastaw opornicy dekadowej $R_w$ dla poszczególnych badanych rezystancji.}
	\label{tab:nastawy_opornicy_dekadowej}
\end{table}

\begin{table}[H]
	\centering
	\begin{tabular}{|c|c|c|c|c|c||c|c|}
		\hline
		                     & \multicolumn{7}{c|}{\textbf{Napięcie} $U$ [\text{V}] \textbf{przy pomiarze rezystora:}}                                                 \\ \cline{2-8}
		\textbf{Długość} $L$ & $R_1$                                                                                   & $R_2$ & $R_3$ & $R_4$ & $R_5$ & $R_r$ & $R_s$ \\ \hline
		$1/3$                & 0,67                                                                                    & 0,63  & 0,62  & 0,65  & 0,61  & 0,64  & 0,62  \\ \hline
		$1/2$                & 0,64                                                                                    & 0,63  & 0,62  & 0,62  & 0,62  & 0,64  & 0,61  \\ \hline
		$2/3$                & 0,64                                                                                    & 0,62  & 0,61  & 0,61  & 0,62  & 0,66  & 0,61  \\ \hline
	\end{tabular}
	\caption{Wartości napięcia zasilania $U$ podczas pomiarów poszczególnych rezystancji.}
\end{table}

% ---------- OBLICZENIA I WYNIKI ----------
\section{Opracowanie wyników pomiarów}

\subsection{Wartości badanych rezystancji i ich wartości średnie}

Dla każdego podziału listwy $L$ obliczono stosunek $\frac{l_1}{l_2}$:

\begin{table}[h]
	\centering
	\begin{tabular}{|c|c|}
		\hline
		\textbf{Ustawienie} $L$ & \textbf{Stosunek} $\frac{l_1}{l_2}$ \\ \hline
		$1/3$                   & 0,5                                 \\ \hline
		$1/2$                   & 1,0                                 \\ \hline
		$2/3$                   & 2,0                                 \\ \hline
	\end{tabular}
	\caption{Wartości stosunku podziału listwy dla badanych ustawień $L$.}
\end{table}

Następnie korzystając ze wzoru \eqref{eq:mostek_liniowy}, dla każdej nastawy opornicy dekadowej $R_w$ i długości listwy $L$, obliczono wartości nieznanej rezystancji $R_x$. Wyniki przedstawiono w tabeli poniżej.

\begin{table}[H]
	\centering
	\begin{tabular}{|c|c|c|c|}
		\hline
		\textbf{Rezystor} & \textbf{$R_x\,[\Omega]$ dla $L=1/3$} & \textbf{$R_x\,[\Omega]$ dla $L=1/2$} & \textbf{$R_x\,[\Omega]$ dla $L=2/3$} \\ \hline
		$R_1$             & 48,5                                 & 52,0                                 & 55,2                                 \\ \hline
		$R_2$             & 101,0                                & 90,0                                 & 98,6                                 \\ \hline
		$R_3$             & 416,0                                & 452,0                                & 464,0                                \\ \hline
		$R_4$             & 1160,0                               & 1340,0                               & 1306,0                               \\ \hline
		$R_5$             & 3050,0                               & 3299,0                               & 3440,0                               \\ \hline \hline
		$R_r$             & 32,0                                 & 32,0                                 & 32,8                                 \\ \hline
		$R_s$             & 530,0                                & 560,0                                & 618,0                                \\ \hline
	\end{tabular}
	\caption{Wyznaczone wartości badanych rezystancji dla różnych ustawień mostka.}
	\label{tab:wyniki_Rx}
\end{table}

\noindent \textbf{Przykładowe obliczenie} (dla rezystora $R_2$ przy ustawieniu $L=1/3$): \\
Wartość $R_w = 202\,\Omega$, stosunek $\frac{l_1}{l_2} = 0,5$.
\begin{equation*}
	R_2^{(1/3)} = 202 \cdot 0,5 = 101,0\,\Omega
\end{equation*}

\vspace{0.5cm}
Na podstawie wyznaczonych wartości cząstkowych obliczono średnie wartości rezystancji dla każdego badanego elementu:

\begin{table}[H]
	\centering
	\begin{tabular}{|c|c|}
		\hline
		\textbf{Rezystor} & \textbf{Średnia rezystancja $\overline{R}$ $[\Omega]$} \\ \hline
		$R_1$             & 51,9                                                   \\ \hline
		$R_2$             & 96,5                                                   \\ \hline
		$R_3$             & 444,0                                                  \\ \hline
		$R_4$             & 1268,7                                                 \\ \hline
		$R_5$             & 3263,0                                                 \\ \hline \hline
		$R_r$             & 32,3                                                   \\ \hline
		$R_s$             & 569,3                                                  \\ \hline
	\end{tabular}
	\caption{Średnie wartości badanych rezystancji.}
	\label{tab:srednie_Rx}
\end{table}

\noindent \textbf{Przykładowe obliczenie} (dla rezystora $R_1$): \\
Korzystając z wartości wyznaczonych dla trzech ustawień suwaka ($48,5\,\Omega$, $52,0\,\Omega$ oraz $55,2\,\Omega$):
\begin{equation*}
	\overline{R_1} = \frac{48,5 + 52,0 + 55,2}{3} = \frac{155,7}{3} = 51,9\,\Omega
\end{equation*}

\subsection{Ocena niepewności pomiaru}

Niepewność pomiaru $\Delta R$ oszacowano metodą maksymalnego rozrzutu wyników, zgodnie ze wzorem:
\begin{equation}
	\Delta R = \frac{R_{max} - R_{min}}{2}
\end{equation}
Wyniki obliczeń niepewności przedstawiono w tabeli poniżej.

\begin{table}[H]
	\centering
	\begin{tabular}{|c|c|}
		\hline
		\textbf{Rezystor} & \textbf{Niepewność $\Delta R$ $[\Omega]$} \\ \hline
		$R_1$             & 3,4                                       \\ \hline
		$R_2$             & 5,5                                       \\ \hline
		$R_3$             & 24,0                                      \\ \hline
		$R_4$             & 90,0                                      \\ \hline
		$R_5$             & 195,0                                     \\ \hline \hline
		$R_r$             & 0,4                                       \\ \hline
		$R_s$             & 44,0                                      \\ \hline
	\end{tabular}
	\caption{Niepewności maksymalne wyznaczonych rezystancji.}
	\label{tab:niepewnosci}
\end{table}

\noindent \textbf{Przykładowe obliczenie} (dla rezystora $R_2$): \\
Wartości skrajne wynoszą $R_{max} = 101,0\,\Omega$ oraz $R_{min} = 90,0\,\Omega$.
\begin{equation*}
	\Delta R_2 = \frac{101,0 - 90,0}{2} = 5,5\,\Omega
\end{equation*}

Ostateczne wyniki pomiarów (wraz z niepewnościami) wynoszą:
\begin{itemize}
	\item $R_1 = (51,9 \pm 3,4)\,\Omega$
	\item $R_2 = (96,5 \pm 5,5)\,\Omega$
	\item $R_3 = (444 \pm 24)\,\Omega$
	\item $R_4 = (1269 \pm 90)\,\Omega$
	\item $R_5 = (3263 \pm 195)\,\Omega$
	\item $R_r = (32,3 \pm 0,4)\,\Omega$
	\item $R_s = (569 \pm 44)\,\Omega$
\end{itemize}

\subsection{Weryfikacja prawa łączenia rezystorów}

Korzystając z wyznaczonych wcześniej średnich wartości rezystancji $R_1, R_2$ oraz $R_3$, obliczono teoretyczne wartości rezystancji zastępczej dla połączenia szeregowego ($R_{s,teor}$) i równoległego ($R_{r,teor}$).

Dla połączenia szeregowego:
\begin{equation}
	R_{s,teor} = R_1 + R_2 + R_3 = 51,9 + 96,5 + 444,0 = 592,4\,\Omega
\end{equation}

Dla połączenia równoległego:
\begin{equation}
	R_{r,teor} = \left( \frac{1}{R_1} + \frac{1}{R_2} + \frac{1}{R_3} \right)^{-1} = \left( \frac{1}{51,9} + \frac{1}{96,5} + \frac{1}{444,0} \right)^{-1} \approx 31,4\,\Omega
\end{equation}

Następnie porównano wartości teoretyczne z wynikami zmierzonymi bezpośrednio ($R_{pom}$), wyznaczając błąd bezwzględny $|\Delta R| = |R_{pom} - R_{teor}|$ oraz błąd względny $\delta = \frac{|\Delta R|}{R_{teor}} \cdot 100\%$.

\begin{table}[H]
	\centering
	\begin{tabular}{|c|c|c|c|c|}
		\hline
		\textbf{Połączenie} & \textbf{$R_{pom}$ [$\Omega$]} & \textbf{$R_{teor}$ [$\Omega$]} & \textbf{$|\Delta R|$ [$\Omega$]} & \textbf{$\delta$ [\%]} \\ \hline
		Szeregowe ($R_s$)   & 569,3                         & 592,4                          & 23,1                             & 3,90                   \\ \hline
		Równoległe ($R_r$)  & 32,3                          & 31,4                           & 0,9                              & 2,86                   \\ \hline
	\end{tabular}
	\caption{Porównanie zmierzonych rezystancji zastępczych z wartościami teoretycznymi.}
	\label{tab:porownanie_teoria}
\end{table}

% ---------- WNIOSKI ----------
\section{Wnioski}

W ramach ćwiczenia wyznaczono wartości rezystancji pięciu nieznanych rezystorów oraz ich połączeń (szeregowego i równoległego) metodą mostka Wheatstone’a. Pomiary wykonano dla trzech różnych ustawień suwaka na listwie ($L=1/3, 1/2, 2/3$), a jako wynik końcowy przyjęto wartość średnią.

Ostateczne wyniki pomiarów wraz z niepewnościami maksymalnymi wynoszą:
\begin{itemize}
	\item $R_1 = (51,9 \pm 3,4)\,\Omega$
	\item $R_2 = (96,5 \pm 5,5)\,\Omega$
	\item $R_3 = (444 \pm 24)\,\Omega$
	\item $R_4 = (1269 \pm 90)\,\Omega$
	\item $R_5 = (3263 \pm 195)\,\Omega$
	\item $R_r = (32,3 \pm 0,4)\,\Omega$ (połączenie równoległe $R_1, R_2, R_3$)
	\item $R_s = (569 \pm 44)\,\Omega$ (połączenie szeregowe $R_1, R_2, R_3$)
\end{itemize}

Przeprowadzono weryfikację praw łączenia rezystorów, porównując zmierzone wartości rezystancji zastępczych z wartościami teoretycznymi obliczonymi na podstawie składowych $R_1, R_2, R_3$:

\begin{enumerate}
	\item Dla połączenia szeregowego wartość teoretyczna wynosi $R_{s,teor} = 592,4\,\Omega$, natomiast wartość zmierzona $R_{s,pom} = 569,3\,\Omega$. Różnica wynosi $23,1\,\Omega$, co stanowi błąd względny $\delta \approx 3,90\%$.
	\item Dla połączenia równoległego wartość teoretyczna wynosi $R_{r,teor} = 31,4\,\Omega$, natomiast wartość zmierzona $R_{r,pom} = 32,3\,\Omega$. Różnica wynosi $0,9\,\Omega$, co stanowi błąd względny $\delta \approx 2,86\%$.
\end{enumerate}

Uzyskane wyniki charakteryzują się wysoką zgodnością z teorią. Błędy względne w obu przypadkach nie przekraczają $4\%$, co potwierdza poprawność praw Kirchhoffa oraz skuteczność metody mostka Wheatstone’a w pomiarach rezystancji. Różnice między wartościami zmierzonymi a teoretycznymi mieszczą się w granicach oszacowanych niepewności pomiarowych (dla $R_s$ różnica $23,1\,\Omega$ mieści się w niepewności $\pm 44\,\Omega$).

% ---------- WYKRESY ----------
% \section{Wykresy}
% \newpage

\bibliographystyle{apalike}
\bibliography{bibliography}

\end{document}
