\documentclass[a4paper,12pt]{article}
\usepackage[left=2cm,right=2cm,top=2cm,bottom=2cm]{geometry} 
\usepackage{multicol} 
\usepackage{ragged2e} 
\usepackage{graphicx} 
\usepackage{float}
\usepackage{caption}
\usepackage{amsmath} 
\usepackage{amssymb} 
\usepackage[svgnames]{xcolor}
\usepackage[colorlinks=true, urlcolor=blue, linkcolor=black, citecolor=orange]{hyperref} 
\usepackage{polski} 
\usepackage[utf8]{inputenc} 
\usepackage{enumitem} 
\usepackage{indentfirst}
\usepackage{array}
\usepackage{longtable}
\usepackage{pdflscape}
\usepackage[round]{natbib}
\setlist[itemize]{itemsep=0pt, topsep=0pt}
\usepackage{siunitx}

% SI setup preferences
\sisetup{output-decimal-marker={,}}
\sisetup{exponent-product = \cdot}
\sisetup{per-mode = symbol}

% LTeX: language=pl-PL

\begin{document}

\noindent
\begin{minipage}{0.5\textwidth}
	\raggedright
	% full name, index number
	\textbf{} \\
	II rok, Fizyka \\
	Wtorek, 8:00-10:15
	\vspace{0.5cm}
\end{minipage}
\begin{minipage}{0.5\textwidth}
	\raggedleft
	% date of the experiments
	29.04.2025 \\
	\vspace{0.5cm}
	Prowadząca: \\
	dr Sylwia Owczarek
\end{minipage}

\vspace{2cm}
\begin{center}
	% lab number
	\LARGE \textbf{Ćwiczenie nr 42} \\[0.5cm]
	% title
	\Large \textbf{Mostek Wheatstone'a}
\end{center}

\vspace{1cm}
\noindent

% \tableofcontents
% \newpage

% ---------- WSTĘP TEORETYCZNY ----------
\section{Wstęp teoretyczny}

\subsection{Mostek Wheatstone'a}

Przez mostek nie płynie prąd, tylko wtedy, gdy cztery rezystancje spełniają następującą zależność:

\begin{equation}
	\label{eq:wheatstone_proporcja}
	\frac{R_1}{R_2} = \frac{R_3}{R_4}
\end{equation}

Gdy jeden z czterech rezystorów jest nieznany, to można wyznaczyć jego rezystancję z powyższej proporcji \eqref{eq:wheatstone_proporcja}.

\subsubsection{Liniowy mostek Wheatstone'a}

W liniowym mostku Wheatstone'a opory $R_1$ i $R_2$ stworzone są z drutu oporowego, rozpiętego na podstawce wzdłuż skali milimetrowej. Dla takiego mostka, korzystając z prawa Ohma, stosunek $\frac{R_1}{R_2}$ można zastąpić stosunkiem $\frac{l_1}{l_2}$.

W ten sposób dla nieznanego oporu $R_x = R_3$ i znanego oporu $R_4 = R$, można określić wartość $R_x$ jako:

\begin{equation}
	\label{eq:mostek_liniowy}
	R_x = R \frac{l_1}{l_2}
\end{equation}


% ---------- OPIS DOŚWIADCZENIA ----------
% \section{Opis doświadczenia}

% ---------- OPRACOWANIE WYNIKÓW POMIARÓW ----------
\section{Opracowanie wyników pomiarów}


% ---------- TABELE ----------
\subsection{Tabele pomiarowe}

Zmierzone wartości nastaw opornicy dekadowej $R_w$ dla trzech ustawień suwaka listwy $L$ i badanych rezystorów $R_1,\,R_2,\,...,\,R_5$ zapisano w tabeli \ref{tab:nastawy_opornicy_dekadowej}. Przez $R_s$ i $R_r$ oznaczono rezystancję rezystorów 1-3, połączonych odpowiednio szeregowo i równolegle.

\begin{table}[H]
	\centering
	\begin{tabular}{|c|c|c|c|c|c||c|c|}
		\hline
		                     & \multicolumn{7}{c|}{\textbf{Nastawa opornicy } $R_w$ $[\Omega]$ \textbf{dla rezystora:}}                                                                     \\ \cline{2-8}
		\textbf{Długość} $L$ & $R_1$                                                                                    & $R_2$ & $R_3$ & $R_4$ & $R_5$ & $R_r$ (równol.) & $R_s$ (szereg.) \\ \hline
		$1/3$                & 97                                                                                       & 202   & 832   & 2320  & 6100  & 64              & 1060            \\ \hline
		$1/2$                & 52                                                                                       & 90    & 452   & 1340  & 3299  & 32              & 560             \\ \hline
		$2/3$                & 27,6                                                                                     & 49,3  & 232   & 653   & 1720  & 16,4            & 309             \\ \hline
	\end{tabular}
	\caption{Wartości nastaw opornicy dekadowej $R_w$ dla poszczególnych badanych oporów.}
	\label{tab:nastawy_opornicy_dekadowej}
\end{table}

\begin{table}[H]
	\centering
	\begin{tabular}{|c|c|c|c|c|c||c|c|}
		\hline
		                     & \multicolumn{7}{c|}{\textbf{Napięcie} $U$ [\text{V}] \textbf{przy pomiarze rezystora:}}                                                 \\ \cline{2-8}
		\textbf{Długość} $L$ & $R_1$                                                                                   & $R_2$ & $R_3$ & $R_4$ & $R_5$ & $R_r$ & $R_s$ \\ \hline
		$1/3$                & 0,67                                                                                    & 0,63  & 0,62  & 0,65  & 0,61  & 0,64  & 0,62  \\ \hline
		$1/2$                & 0,64                                                                                    & 0,63  & 0,62  & 0,62  & 0,62  & 0,64  & 0,61  \\ \hline
		$2/3$                & 0,64                                                                                    & 0,62  & 0,61  & 0,61  & 0,62  & 0,66  & 0,61  \\ \hline
	\end{tabular}
	\caption{Wartości napięcia zasilania $U$ podczas pomiarów poszczególnych rezystancji.}
\end{table}

% ---------- OBLICZENIA ----------
\subsection{Wartości badanych rezystancji i ich wartości średnie}

Dla każdego podziału listwy $L$ obliczono stosunek $\frac{l_1}{l_2}$:

\begin{table}[h]
	\centering
	\begin{tabular}{|c|c|}
		\hline
		\textbf{Ustawienie} $L$ & \textbf{Stosunek} $\frac{l_1}{l_2}$ \\ \hline
		$1/3$                   & 0,5                                 \\ \hline
		$1/2$                   & 1,0                                 \\ \hline
		$2/3$                   & 2,0                                 \\ \hline
	\end{tabular}
	\caption{Wartości stosunku podziału listwy dla badanych ustawień $L$.}
\end{table}

Następnie korzystając z wzoru \eqref{eq:mostek_liniowy}, dla każdej nastawy opornicy dekadowej $R_w$ i długości listwy $L$, obliczono wartości nieznanego oporu $R_x$:


\subsection{Ocena niepewności maksymalnych}




% ---------- NIEPEWNOŚCI ----------
% \section{Ocena niepewności pomiaru}

% ---------- WNIOSKI ----------
% \section{Wnioski}

% ---------- WYKRESY ----------
% \section{Wykresy}
% \newpage

\bibliographystyle{apalike}
\bibliography{bibliography}

\end{document}
