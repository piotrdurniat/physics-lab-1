\documentclass[a4paper,12pt]{article}
\usepackage[left=2cm,right=2cm,top=2cm,bottom=2cm]{geometry} 
\usepackage{multicol} 
\usepackage{ragged2e} 
\usepackage{graphicx} 
\usepackage{float}
\usepackage{caption}
\usepackage{amsmath} 
\usepackage{amssymb} 
\usepackage[svgnames]{xcolor}
\usepackage[colorlinks=true, urlcolor=blue, linkcolor=black, citecolor=orange]{hyperref} 
\usepackage{polski} 
\usepackage[utf8]{inputenc} 
\usepackage{enumitem} 
\usepackage{indentfirst}
\usepackage{array}
\usepackage{longtable}
\usepackage{pdflscape}
\usepackage[round]{natbib}
\setlist[itemize]{itemsep=0pt, topsep=0pt}
\usepackage{siunitx}


\begin{document}
%
% \noindent
% \begin{minipage}{0.5\textwidth}
%     \raggedright
%     % full name, index number
%     \textbf{Imię Nazwisko} \\
%     Rok, Kierunek \\
%     Termin zajęć
%     \vspace{0.5cm}
% \end{minipage}
% \begin{minipage}{0.5\textwidth}
%     \raggedleft
%     % date of the experiments
%     Data wykonania \\
%     \vspace{0.5cm}
%     Prowadząca: \\
%     Tytuł Imię Nazwisko
% \end{minipage}
%
% \vspace{2cm}
% \begin{center}
%     % lab number
%     \LARGE \textbf{Ćwiczenie nr 42} \\[0.5cm]
%     % title
%     \Large \textbf{Mostek Wheatstone'a}
% \end{center}
%
% \vspace{1cm}
% \noindent

\vspace{2cm}
\begin{center}
	\LARGE \textbf{Ćwiczenie nr 42} \\[0.5cm]
	\Large \textbf{Mostek Wheatstone'a}
\end{center}
% ---------- WSTĘP TEORETYCZNY ----------
\section{Wstęp teoretyczny}

\subsection{Opór elektryczny i prawa Kirchhoffa}
Opór elektryczny $R$ przewodnika zależy od jego wymiarów geometrycznych oraz rodzaju materiału. Dla jednorodnego przewodu o długości $l$ i polu przekroju $S$, opór wyraża się wzorem \citep{wstep_do_cwiczenia}:
$$ R = \rho \frac{l}{S} $$
gdzie $\rho$ to opór właściwy materiału.
Do analizy obwodów rozgałęzionych, takich jak mostek, stosuje się prawa Kirchhoffa, szczegółowo omówione w literaturze akademickiej \citep{fizyka_tom3}:
\begin{itemize}
	\item \textbf{I Prawo:} Suma algebraiczna prądów wpływających do i wypływających z węzła jest równa zeru ($\sum I = 0$).
	\item \textbf{II Prawo:} W zamkniętym obwodzie (oczku) suma spadków napięć na oporach jest równa sumie sił elektromotorycznych źródeł.
\end{itemize}

\subsection{Zasada działania mostka Wheatstone'a}
Mostek Wheatstone'a to układ służący do precyzyjnego pomiaru oporów, składający się z czterech rezystorów ($R_x, R_2, R_3, R_4$), źródła zasilania oraz galwanometru \citep{Drynski1976}.
Metoda ta opiera się na doprowadzeniu układu do stanu równowagi, w którym przez przekątną mostka (galwanometr) nie płynie prąd ($I_g = 0$). Oznacza to, że potencjały w punktach węzłowych mostka są równe.
Zastosowanie praw Kirchhoffa dla stanu równowagi prowadzi do relacji:
$$ R_x R_4 = R_2 R_3 $$
Pozwala to wyznaczyć nieznany opór $R_x$ na podstawie znanych wartości pozostałych rezystorów:
$$ R_x = R_2 \frac{R_3}{R_4} $$
Zaletą metody mostkowej jest to, że wynik pomiaru nie zależy od wahań napięcia zasilającego ani od oporu wewnętrznego źródła, co czyni ją jedną z najdokładniejszych metod pomiarowych w laboratoriach studenckich \citep{Drynski1976}.

% ---------- OPIS DOŚWIADCZENIA ----------
% \section{Opis doświadczenia}
% ...

% ---------- WYNIKI POMIARÓW ----------
% \section{Wyniki pomiarów}
% ...

% ---------- OBLICZENIA ----------
% \section{Opracowanie wyników}
% ...

% ---------- WNIOSKI ----------
% \section{Wnioski}
% ...

% ---------- WYKRESY ----------
% \section{Wykresy}
% ...

\bibliographystyle{apalike}
\bibliography{bibliography}

\end{document}
