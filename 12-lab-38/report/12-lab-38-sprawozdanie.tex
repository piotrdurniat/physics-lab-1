\documentclass[a4paper,12pt]{article}
\usepackage[left=2cm,right=2cm,top=2cm,bottom=2cm]{geometry} % Do ustawień marginesów
\usepackage{multicol} % Dla podziału na kolumny
\usepackage{ragged2e} % Dla justowania tekstu
\usepackage{graphicx} % Required for inserting images
\usepackage{float}
\usepackage{caption}
\usepackage{amsmath} % Math formulas
\usepackage{amssymb} % Symbols
\usepackage[svgnames]{xcolor}
\usepackage[colorlinks=true, urlcolor=blue, linkcolor=black, citecolor=orange]{hyperref} % Hyperlinks
\usepackage{polski} % Polish language
\usepackage[utf8]{inputenc} % Text encoding
\usepackage{enumitem} % Pakiet do elastycznego sterowania listami
\usepackage{indentfirst}
\usepackage{array}
\usepackage{longtable}
\usepackage{multirow}
\usepackage{booktabs}    % Dla ładniejszych tabel
\usepackage{amsmath}     % Dla symboli matematycznych i środowisk
\usepackage{siunitx}     % Do poprawnego formatowania jednostek i liczb (opcjonalnie, ale zalecane)
                         % Jeśli używasz siunitx, wartości w tabeli można formatować np. \SI{0,07280}{\newton\per\metre}

\setlist[itemize]{itemsep=0pt, topsep=0pt}

\setlist[enumerate]{itemsep=0pt, topsep=0pt}

\begin{document}

% Górna część strony
\noindent
\begin{minipage}{0.5\textwidth}
    \raggedright
    \textbf{Piotr Durniat} \\
    I rok, Fizyka \\
    Wtorek, 8:00-10:15 \\
    \vspace{0.5cm}
    \vspace{0.5cm}
\end{minipage}%
\begin{minipage}{0.5\textwidth}
    \raggedleft
    Data wykonania pomiarów: \\
    27.05.2025 \\
    \vspace{0.5cm}
    Prowadząca: \\
    dr Iwona Mróz
\end{minipage}

% Tytuł ćwiczenia
\vspace{2cm}
\begin{center}
    \LARGE \textbf{Ćwiczenie nr 38} \\[0.5cm]
    \Large \textbf{Pomiar napięcia powierzchniowego}
\end{center}

% Reszta treści
\vspace{1cm} % Kolejny odstęp
\noindent

\tableofcontents
\newpage

% ---------- WSTĘP TEORETYCZNY ----------
\section{Wstęp teoretyczny}

\subsection{Napięcie powierzchniowe}

Napięcie powierzchniowe jest zjawiskiem fizycznym występującym na granicy faz, najczęściej ciecz-gaz, i wynika z oddziaływań międzycząsteczkowych w cieczy. Cząsteczki znajdujące się wewnątrz cieczy oddziałują z sąsiednimi cząsteczkami we wszystkich kierunkach, natomiast cząsteczki na powierzchni oddziałują głównie z cząsteczkami znajdującymi się pod nimi. Ta nierównowaga sił powoduje, że powierzchnia cieczy zachowuje się jak napięta błona, dążąc do przyjęcia kształtu o minimalnej powierzchni.

Napięcie powierzchniowe ($\sigma$) definiuje się jako stosunek siły ($F$) działającej stycznie do powierzchni cieczy wzdłuż linii o długości ($l$):

\begin{equation}
    \sigma = \frac{F}{l}
\end{equation}

Jednostką napięcia powierzchniowego w układzie SI jest N/m (newton na metr).

\subsection{Zjawiska związane z napięciem powierzchniowym}

Na granicy ośrodków ciecz-ciało stałe-gaz obserwuje się zjawisko menisku (wklęsłego lub wypukłego), zależnie od oddziaływań między cieczą a ciałem stałym. Jeśli siły przyciągania między cząsteczkami cieczy a ciałem stałym są silniejsze niż między samymi cząsteczkami cieczy, powstaje menisk wklęsły (np. woda w szklanej rurce). W przeciwnym przypadku tworzy się menisk wypukły (np. rtęć w szklanej rurce).

Zjawisko włoskowatości (kapilarności) jest bezpośrednim skutkiem napięcia powierzchniowego i zwilżalności powierzchni. Polega ono na samorzutnym podnoszeniu się lub obniżaniu cieczy w wąskich kapilarach. Wysokość słupa cieczy w kapilarze zależy od napięcia powierzchniowego, gęstości cieczy oraz promienia kapilary.

\subsection{Metody pomiaru napięcia powierzchniowego}

W niniejszym ćwiczeniu wykorzystano dwie metody pomiaru napięcia powierzchniowego:

\begin{itemize}
    \item \textbf{Metoda odrywania} - polega na pomiarze siły $F$ potrzebnej do oderwania płytki od powierzchni cieczy. Napięcie powierzchniowe $\sigma$ jest zdefiniowane jako siła działająca na jednostkę długości krawędzi. Całkowita długość krawędzi styku płytki (o długości $l$ i grubości $d$) z cieczą wynosi $2(l+d)$. Zatem siła napięcia powierzchniowego $F$ (równa sile odrywania, po uwzględnieniu ciężaru płytki oraz przy założeniu kąta zwilżania $\gamma \approx 0$, czyli $\cos\gamma \approx 1$) wyraża się jako $F = \sigma \cdot 2(l+d)$. Przekształcając ten wzór, otrzymujemy:
          \begin{equation}
              \sigma = \frac{F}{2(l+d)} \label{eq:odrywanie}
          \end{equation}
          gdzie $F$ to mierzona siła odrywania (po skompensowaniu ciężaru płytki), $l$ to długość płytki, a $d$ to jej grubość.

    \item \textbf{Metoda stalagmometru} - opiera się na pomiarze masy kropel cieczy, które odrywają się od kapilary o znanym promieniu. Napięcie powierzchniowe wyznacza się, porównując masy kropel badanej cieczy i cieczy wzorcowej o znanym napięciu powierzchniowym.
\end{itemize}

Napięcie powierzchniowe zależy od temperatury (na ogół maleje liniowo wraz z jej wzrostem) oraz od obecności zanieczyszczeń i substancji powierzchniowo czynnych, które mogą znacząco obniżyć jego wartość.

Wstęp teoretyczny został opracowany na podstawie podręcznika \cite{fizyka_dla_szkół_wyższych_tom_2} oraz materiałów dydaktycznych Politechniki Wrocławskiej \cite{lab12_pwr}.

% ---------- OPIS DOŚWIADCZENIA ----------
\section{Opis doświadczenia}

Celem doświadczenia było wyznaczenie napięcia powierzchniowego dla wody destylowanej, alkoholu i acetonu przy użyciu dwóch metod: metody odrywania oraz metody stalagmometru.

\subsection{Metoda odrywania}

Pomiary wykonano przy użyciu wagi torsyjnej (Rys. 1 w instrukcji).
\begin{enumerate}
    \item Zmierzono 3-krotnie grubość $d$ płytki pomiarowej za pomocą śruby mikrometrycznej oraz długość jej podstawy $l$ za pomocą suwmiarki. Płytkę następnie osuszono i zawieszono na haczyku wagi.
    \item Odaretowano wagę i zważono płytkę, notując jej masę spoczynkową (wskazanie wagi bez dodatkowego obciążenia).
    \item Pod płytkę podstawiono naczynko z badaną cieczą (wodą destylowaną) tak, aby dolna krawędź płytki niemal dotykała powierzchni cieczy. Waga była zaaretowana.
    \item Odaretowano wagę, doprowadzając do zanurzenia dolnej części płytki w cieczy. Następnie powoli obracano pokrętłem wagi, zwiększając siłę, aż do momentu oderwania płytki od powierzchni cieczy. Odczytano maksymalną siłę $F$ (wskazanie wagi w jednostkach masy [mg]) działającą w momencie odrywania.
    \item Pomiar siły odrywającej $F$ wykonano 10-krotnie dla wody destylowanej.
    \item Kroki 3-5 powtórzono dla alkoholu i acetonu.
\end{enumerate}

\subsection{Metoda stalagmometru}
\begin{enumerate}
    \item Zważono czyste i osuszone naczynko pomiarowe.
    \item Sprawdzono drożność kapilary stalagmometru i zmierzono jej zewnętrzny promień $R$.
    \item Napełniono naczynko 30 kroplami wody destylowanej, które odrywały się od kapilary, a następnie zważono naczynko z cieczą.
    \item Pomiar masy 30 kropel (krok 3) powtórzono 3-krotnie dla wody destylowanej.
    \item Zanotowano temperaturę otoczenia.
    \item Kroki 1, 3 i 4 powtórzono dla alkoholu i acetonu.
\end{enumerate}

% ---------- OPRACOWANIE WYNIKÓW POMIARÓW ----------
\section{Opracowanie wyników pomiarów}

\subsection{Tabele pomiarowe}


\begin{table}[H]
    \centering
    \begin{tabular}{|c|}
        \hline
        \textbf{Długość płytki $l$ [mm]} \\
        \hline
        9{,}5 \\ % [cite: 36]
        \hline
    \end{tabular}
    \caption{Zmierzona długość płytki pomiarowej.}
    \label{tab:dlugosc_plytki}
\end{table}

\begin{table}[H]
    \centering
    \begin{tabular}{|c|c|c|c|}
        \hline
        \multicolumn{3}{|c|}{\textbf{Grubość płytki [mm]}} \\
        \hline
        \textbf{Wskazanie} & \textbf{Błąd wskazania zerowego ($\Delta d$) } & \textbf{Wartość skorygowana} \\
        \hline
        0{,}92 & 0{,}46 & 0{,}46 \\
        \hline
    \end{tabular}
    \caption{Pomiar grubości płytki pomiarowej wraz z korektą błędu wskazania zerowego.}
    \label{tab:grubosc_plytki}
\end{table}

\begin{table}[H]
    \centering
    \begin{tabular}{|c|}
        \hline
        \textbf{Masa spoczynkowa [mg]} \\
        \hline
        272 \\
        \hline
    \end{tabular}
    \caption{Masa spoczynkowa płytki pomiarowej.}
    \label{tab:masa_spoczynkowa}
\end{table}

\begin{table}[H]
    \centering
    \begin{tabular}{|c|c|c|c|}
        \hline
        \multicolumn{4}{|c|}{\textbf{Siła odrywająca [mg]}} \\
        \hline
        \textbf{Pomiar} & \textbf{Woda} & \textbf{Alkohol} & \textbf{Aceton} \\
        \hline
        1 & 398 & 308 & 312 \\ % [cite: 37]
        \hline
        2 & 394 & 310 & 312 \\ % [cite: 37]
        \hline
        3 & 392 & 310 & 310 \\ % [cite: 37]
        \hline
        4 & 392 & 310 & 314 \\ % [cite: 37]
        \hline
        5 & 396 & 312 & 312 \\ % [cite: 37]
        \hline
        6 & 396 & 308 & 312 \\ % [cite: 38]
        \hline
        7 & 398 & 308 & 314 \\ % [cite: 38]
        \hline
        8 & 398 & 310 & 314 \\ % [cite: 38]
        \hline
        9 & 398 & 312 & 314 \\ % [cite: 38]
        \hline
        10 & 398 & 310 & 312 \\ % [cite: 39]
        \hline
    \end{tabular}
    \caption{Siła odrywająca płytkę dla różnych cieczy.}
    \label{tab:sila_odrywania}
\end{table}

\begin{table}[H]
    \centering
    \begin{tabular}{|c|c|c|c|}
        \hline
        \textbf{Nr pomiaru} & \textbf{Ciecz} & $m_n\,\text{[g]}$ & $m_{nw}\,\text{[g]}$ \\
        \hline
        1 & Woda & 2{,}007 & 3{,}452 \\
        2 & Woda & 2{,}038 & 3{,}537 \\
        3 & Woda & 2{,}016 & 3{,}519 \\
        \hline
        1 & Alkohol & 1{,}997 & 2{,}186 \\
        \hline
    \end{tabular}
    \caption{Pomiary masy naczynia dla różnych cieczy. $m_n$ - masa naczynia, $m_{nw}$ - masa naczynia z cieczą.}
    \label{tab:stalagmometr}
\end{table}


% ---------- OBLICZENIA ----------
\subsection{Obliczenia}

\subsubsection{Obliczenia dla metody odrywania:}

Obliczono średnią arytmetyczną wartości siły odrywającej płytkę $\bar{F}_{\text{mg}}$ (wskazania wagi w jednostkach masy [mg]) dla każdej cieczy na podstawie 10 pomiarów.

$$
    \bar{F}_{\text{mg}} = \frac{\sum_{i=1}^{n} F_i}{n}
$$

gdzie $F_i$ to kolejne pomiary siły, a $n$ to liczba pomiarów ($n=10$).

Od każdego pomiaru siły odrywającej należy odjąć masę spoczynkową płytki $m_0$, aby uzyskać siłę netto:

$$
    F_{\text{netto}} = F_{\text{zmierzone}} - m_0
$$

Przeliczono średnią siłę netto $\bar{F}_{\text{N}}$ na niutony [N], wykorzystując zależność, że $1 \, [\text{mg}]$ odpowiada sile $9{,}807 \cdot 10^{-6} \, [\text{N}]$.

$$
    \bar{F}_{\text{N}} = \bar{F}_{\text{netto}} \cdot 9{,}807 \cdot 10^{-6} \, \text{N/mg}
$$

Obliczono wartość napięcia powierzchniowego $\sigma$ dla każdej cieczy, korzystając ze wzoru (\ref{eq:odrywanie}):

$$
    \sigma = \frac{\bar{F}_{\text{N}}}{2(l+d)}
$$

\noindent\textbf{Wymiary płytki:}
\begin{itemize}
    \item Długość ($l$): $9{,}5 \text{ mm} = 0{,}00950 \text{ m}$
    \item Grubość skorygowana ($d_{\text{sk}}$): $0{,}46 \text{ mm} = 0{,}00046 \text{ m}$
    \item Obwód całkowity $2(l+d_{\text{sk}}) = 2(0{,}00950 + 0{,}00046) = 0{,}01992 \text{ m}$
\end{itemize}

\noindent\textbf{Przykładowe obliczenia dla wody destylowanej:}

Średnia siła odrywająca (wskazanie wagi):
\begin{align*}
     & \bar{F}_{\text{mg, woda}} = \frac{398+394+392+392+396+396+398+398+398+398}{10} \\
     & = \frac{3960}{10} = 396{,}00 \text{ mg}
\end{align*}

Siła netto (po odjęciu masy spoczynkowej $m_0 = 272{,}00 \text{ mg}$):
\begin{align*}
     & \bar{F}_{\text{netto, woda}} = 396{,}00 - 272{,}00 = 124{,}00 \text{ mg} \\
     & = 124{,}00 \cdot 9{,}807 \cdot 10^{-6} = 0{,}00121607 \text{ N}
\end{align*}

Napięcie powierzchniowe wody:
$$
    \sigma_{\text{woda}} = \frac{0{,}00121607}{0{,}01992} = 0{,}0610 \frac{\text{N}}{\text{m}}
$$

\noindent\textbf{Wyniki obliczeń dla wszystkich cieczy:}

\begin{table}[H]
    \centering
    \begin{tabular}{|l|c|c|c|c|}
        \hline
        \textbf{Ciecz} & \textbf{$\bar{F}_{\text{mg}}$ [mg]} & \textbf{$\bar{F}_{\text{netto}}$ [mg]} & \textbf{$\bar{F}_{\text{N}}$ [N]} & \textbf{$\sigma$ [N/m]} \\
        \hline
        Woda destylowana & $396{,}00$ & $124{,}00$ & $0{,}00121607$ & $0{,}0610$ \\
        \hline
        Alkohol & $309{,}80$ & $37{,}80$ & $0{,}00037070$ & $0{,}0186$ \\
        \hline
        Aceton & $312{,}60$ & $40{,}60$ & $0{,}00039816$ & $0{,}0200$ \\
        \hline
    \end{tabular}
    \caption{Zestawienie obliczonych wartości siły odrywającej i napięcia powierzchniowego.}
    \label{tab:wyniki_obliczen_odrywanie}
\end{table}


Poniższa tabela \ref{tab:napiecia_powierzchniowe} przedstawia wartości napięcia powierzchniowego (w \SI{}{\newton\per\metre}) dla wody, etanolu i acetonu w temperaturze 20 °C. Główne wartości pochodzą ze źródła \cite{SurfaceTensionDE2024}.

\begin{table}[H]
    \centering
    \begin{tabular}{|l|c|}
        \toprule
        Substancja & Napięcie powierzchniowe (N/m) \\ % Lub używając siunitx: Napięcie powierzchniowe (\si{\newton\per\metre})
        \midrule
        Woda       & 0,07280 \\ % Lub \num{0.07280} jeśli używasz siunitx z odpowiednimi ustawieniami regionalnymi
        \hline
        Etanol     & 0,02210 \\ % Lub \num{0.02210}
        \hline
        Aceton     & 0,02520 \\ % Lub \num{0.02520}
        \bottomrule
    \end{tabular}
    \caption{Wartości napięcia powierzchniowego dla wybranych cieczy w temp. 20 °C.}
    \label{tab:napiecia_powierzchniowe}
\end{table}

\subsubsection{Obliczenia dla metody stalagmometru:}

Masę cieczy obliczono jako różnicę masy naczynia z cieczą i masy pustego naczynia:
$$m_w = m_{nw} - m_n$$

\noindent\textbf{Przykładowe obliczenia dla wody (pomiar 1):}
$$m_w = 3{,}452\text{ g} - 2{,}007\text{ g} = 1{,}445\text{ g}$$

Średnia masa 30 kropli wody:
$$\bar{m}_w = \frac{1{,}445\text{ g} + 1{,}499\text{ g} + 1{,}503\text{ g}}{3} = 1{,}4823\text{ g}$$

Masa pojedynczej kropli (dla wody):
$$m_{w1} = \frac{\bar{m}_w}{30} = \frac{1{,}4823\text{ g}}{30} = 0{,}0494\text{ g}$$

\noindent\textbf{Dla alkoholu:}
$$m_a = 2{,}186\text{ g} - 1{,}997\text{ g} = 0{,}1890\text{ g}$$
$$m_{a1} = \frac{0{,}1890\text{ g}}{30} = 0{,}0063\text{ g}$$

Stosunek napięć powierzchniowych obliczono ze wzoru:
$$\frac{\sigma_a}{\sigma_w} = \frac{\rho_a m_{a1}}{\rho_w m_{w1}}$$
gdzie:
\begin{itemize}
    \item $\rho_w = 0{,}9982\text{ g/cm}^3$ - gęstość wody w 20°C
    \item $\rho_a = 0{,}789\text{ g/cm}^3$ - gęstość alkoholu w 20°C
    \item $m_{w1}, m_{a1}$ - masy pojedynczych kropli
\end{itemize}

Podstawiając wartości:
$$\frac{\sigma_a}{\sigma_w} = \frac{0{,}789 \cdot 0{,}0063}{0{,}9982 \cdot 0{,}0494} = 0{,}1008$$

Wartość tablicowa stosunku napięć powierzchniowych:
$$\left(\frac{\sigma_a}{\sigma_w}\right)_{\text{ref}} = \frac{0{,}0223\text{ N/m}}{0{,}0728\text{ N/m}} = 0{,}3036$$

Błąd względny:
$$\delta = \frac{|0{,}1008 - 0{,}3036|}{0{,}3036} \cdot 100\% = 66{,}8\%$$

% ---------- NIEPEWNOŚCI ----------
\section{Ocena niepewności pomiaru}

% ---------- WNIOSKI ----------
\section{Wnioski}

W ramach przeprowadzonego doświadczenia wyznaczono wartości napięcia powierzchniowego dla trzech cieczy w temperaturze pokojowej. Poniższa tabela przedstawia zestawienie otrzymanych wyników wraz z wartościami tablicowymi oraz analizą błędów:

\begin{table}[H]
    \centering
    \begin{tabular}{|l|c|c|c|c|c|}
        \hline
        \textbf{Ciecz} & \textbf{$\sigma$ [N/m]} & \textbf{$u_c(\sigma)$ [N/m]} & \textbf{$\sigma_{\text{ref}}$ [N/m]} & \textbf{$\Delta\sigma$ [N/m]} & \textbf{$\delta$ [\%]} \\
        \hline
        Woda destylowana & $0{,}0610$ & $0{,}0005$ & $0{,}0728$ & $0{,}0118$ & $16{,}2$ \\
        \hline
        Alkohol & $0{,}0186$ & $0{,}0004$ & $0{,}0223$ & $0{,}0037$ & $16{,}6$ \\
        \hline
        Aceton & $0{,}0200$ & $0{,}0004$ & $0{,}0237$ & $0{,}0037$ & $15{,}6$ \\
        \hline
    \end{tabular}
    \caption{Zestawienie wyników pomiarów z wartościami tablicowymi (w temp. 20°C). $\sigma$ - wartość zmierzona, $u_c(\sigma)$ - niepewność złożona, $\sigma_{\text{ref}}$ - wartość tablicowa, $\Delta\sigma$ - różnica bezwzględna, $\delta$ - błąd względny.}
    \label{tab:wyniki_koncowe}
\end{table}

Różnice między wartościami zmierzonymi a tablicowymi mogą wynikać z kilku czynników:

\begin{enumerate}
    \item Niedokładności metody pomiarowej - metoda odrywania jest wrażliwa na dokładne ustawienie płytki względem powierzchni cieczy.
    \item Temperatura - napięcie powierzchniowe maleje wraz ze wzrostem temperatury, a pomiary były wykonywane w warunkach laboratoryjnych, gdzie temperatura mogła być wyższa niż referencyjna.
    \item Czystość użytych cieczy - szczególnie w przypadku wody destylowanej, gdzie śladowe zanieczyszczenia mogą znacząco wpłynąć na wynik.
\end{enumerate}

Analiza błędów względnych wykazała systematyczne zaniżenie wyników o podobną wartość procentową (około 16\%) dla wszystkich cieczy, co sugeruje obecność systematycznego błędu w metodzie pomiarowej, który w podobnym stopniu wpływa na wszystkie wykonane pomiary. Niepewności pomiarowe są stosunkowo małe (rzędu 1-2\% wartości mierzonej), co świadczy o dobrej powtarzalności pomiarów.

% ---------- WYKRESY ----------
\section{Wykresy}

\bibliographystyle{plain}
\bibliography{bibliography}

\end{document}