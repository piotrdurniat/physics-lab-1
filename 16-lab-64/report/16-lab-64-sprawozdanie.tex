\documentclass[a4paper,12pt]{article}
\usepackage[left=2cm,right=2cm,top=2cm,bottom=2cm]{geometry} % Do ustawień marginesów
\usepackage{multicol} % Dla podziału na kolumny
\usepackage{ragged2e} % Dla justowania tekstu
\usepackage{graphicx} % Required for inserting images
\usepackage{float}
\usepackage{caption}
\usepackage{amsmath} % Math formulas
\usepackage{amssymb} % Symbols
\usepackage[svgnames]{xcolor}
\usepackage[colorlinks=true, urlcolor=blue, linkcolor=black, citecolor=orange]{hyperref} % Hyperlinks
\usepackage{polski} % Polish language
\usepackage[utf8]{inputenc} % Text encoding
\usepackage{enumitem} % Pakiet do elastycznego sterowania listami
\usepackage{indentfirst}
\usepackage{array}
\usepackage{longtable}
\usepackage{pdflscape}
\usepackage[round]{natbib}
\setlist[itemize]{itemsep=0pt, topsep=0pt}
\usepackage{siunitx}
\sisetup{output-decimal-marker={,}}
\sisetup{exponent-product = \cdot}
% LTeX: language=pl-PL

\begin{document}

% Górna część strony
\noindent
\begin{minipage}{0.5\textwidth}
    \raggedright
    \textbf{Piotr Durniat, 347264} \\
    II rok, Fizyka \\
    Wtorek, 8:00-10:15
    \vspace{0.5cm}
\end{minipage}
\begin{minipage}{0.5\textwidth}
    \raggedleft
    04.11.2025 \\
    \vspace{0.5cm}
    Prowadząca: \\
    dr Sylwia Owczarek
\end{minipage}

% Tytuł ćwiczenia
\vspace{2cm}
\begin{center}
    \LARGE \textbf{Ćwiczenie nr 64} \\[0.5cm]
    \Large \textbf{Wyznaczanie stałej siatki dyfrakcyjnej przy użyciu spektrometru}
\end{center}

% Reszta treści
\vspace{1cm} % Kolejny odstęp
\noindent

% \tableofcontents
% \newpage

% ---------- WSTĘP TEORETYCZNY ----------
\section{Wstęp teoretyczny}

\subsection*{Siatka dyfrakcyjna, dyfrakcja i interferencja}
Siatka dyfrakcyjna to układ składający się z szeregu szczelin umieszczonych w równych odległościach od siebie na nieprzezroczystym ekranie. W praktyce, siatkę taką otrzymuje się często poprzez porysowanie płaskorównoległej płytki szklanej równoległymi kreskami przy użyciu diamentu. Nieprzezroczyste rysy pełnią rolę zasłon, a przezroczyste przestrzenie między nimi działają jak szczeliny. Odległość między środkami sąsiednich szczelin nazywana jest \textbf{stałą siatki} i oznaczana jako $d$ \citep{Drynski1976}.

Gdy na siatkę pada prostopadle wiązka promieni równoległych, zgodnie z zasadą Huygensa, każda szczelina staje się źródłem nowych drgań, które rozchodzą się we wszystkich kierunkach. Zjawisko to, polegające na uginaniu prostoliniowego biegu promieni, nazywane jest \textbf{dyfrakcją} \citep{Drynski1976}.

Promienie ugięte na różnych szczelinach są promieniami spójnymi, co oznacza, że mogą się ze sobą nakładać, czyli \textbf{interferować}. W zależności od kierunku (kąta ugięcia $\varphi$), promienie te będą się wzajemnie wzmacniać lub wygaszać \citep{Drynski1976}.

\subsection*{Warunek wzmocnienia}
Do wzmocnienia promieni ugiętych (interferencji konstruktywnej) dochodzi w tych kierunkach, dla których różnica dróg geometrycznych promieni wychodzących z sąsiednich szczelin jest równa całkowitej wielokrotności długości fali światła padającego ($\lambda$). Warunek ten, zwany równaniem siatki dyfrakcyjnej, ma postać \citep{Drynski1976}:
$$
    n\lambda = d \sin \varphi
$$
gdzie:
\begin{itemize}
    \item $n = 1, 2, 3, \dots$ -- rząd widma (kolejny numer wzmocnienia),
    \item $\lambda$ -- długość fali światła,
    \item $d$ -- stała siatki dyfrakcyjnej,
    \item $\varphi$ -- kąt ugięcia promieni, pod którym obserwuje się wzmocnienie.
\end{itemize}

\subsection*{Sieć krystaliczna jako siatka dyfrakcyjna}
Promieniowanie rentgenowskie (promienie X) ma długość fali (rzędu 0,1 nm) porównywalną z odległościami między atomami w ciałach stałych. Z tego powodu regularna, trójwymiarowa struktura atomów w krysztale, zwana siecią krystaliczną, może działać dla promieni X jak trójwymiarowa siatka dyfrakcyjna. Gdy promienie X padają na kryształ, ulegają ugięciu (odbiciu) na kolejnych równoległych płaszczyznach atomowych. Dochodzi do interferencji konstruktywnej (wzmocnienia) tylko pod określonymi kątami, zgodnie z warunkiem znanym jako \textbf{prawo Bragga} \citep{fizyka_dla_szkół_wyższych_tom_3}:
$$
    n\lambda = 2d \sin \theta
$$
gdzie $n$ to rząd widma (liczba całkowita), $\lambda$ to długość fali promieniowania X, $d$ to odległość między sąsiednimi płaszczyznami sieci krystalicznej, a $\theta$ to kąt padania promieni mierzony względem płaszczyzny kryształu \citep{fizyka_dla_szkół_wyższych_tom_3}.

\subsection*{Zasada pomiaru spektrometrem}
Spektrometr jest przyrządem pozwalającym na uzyskanie dokładniejszych wyników pomiarów kątów ugięcia $\varphi$ niż metody uproszczone. Pomiar polega na ustawieniu siatki dyfrakcyjnej na stoliku spektrometru, prostopadle do osi kolimatora \citep{Drynski1976}.

Za pomocą lunety spektrometru, wyposażonej w noniusze, odczytuje się położenia kątowe dla prążków interferencyjnych (linii widmowych) danego rzędu $n$. Aby wyznaczyć średni kąt ugięcia, notuje się pozycje noniuszów $a_1$ i $a_2$ dla prążka po jednej stronie obrazu nieugiętego (centralnego) oraz $b_1$ i $b_2$ dla symetrycznego prążka tego samego rzędu po drugiej stronie. Średni kąt ugięcia $\varphi$ oblicza się ze wzoru \citep{Drynski1976}:
$$
    \varphi = \frac{a_1 - b_1 + a_2 - b_2}{2}
$$
Mierząc kąt ugięcia $\varphi$ dla światła o znanej długości fali $\lambda$ (np. z lasera), można na podstawie równania siatki wyznaczyć jej stałą $d$ \citep{Drynski1976}:
$$
    d = \frac{n\lambda}{\sin \varphi}
$$

% ---------- OPIS DOŚWIADCZENIA ----------
% \section{Opis doświadczenia}

% ---------- OPRACOWANIE WYNIKÓW POMIARÓW ----------
\section{Opracowanie wyników pomiarów}


% ---------- TABELE ----------
\subsection{Tabele pomiarowe}

\begin{table}[H]
    \centering
    \begin{tabular}{|c|c|c|c|c|}
        \hline
        & \multicolumn{2}{c|}{\textbf{Lewa}} & \multicolumn{2}{c|}{\textbf{Prawa}} \\ \hline
        \textbf{n} & \textbf{$\varphi_{1L}$} & \textbf{$\varphi_{2L}$} & \textbf{$\varphi_{1P}$} & \textbf{$\varphi_{2P}$} \\ \hline
        % 1 & $358^\circ 20'$ & $178^\circ 20'$ & $0^\circ 45'$ & $180^\circ 45'$ \\ \hline
        % 2 & $356^\circ 30'$ & $176^\circ 30'$ & $3^\circ 30'$ & $183^\circ 30'$ \\ \hline
        0 & $353^\circ 40'$ & $173^\circ 50'$ & -- & --  \\ \hline
        1 & $351^\circ 05'$ & $171^\circ 05'$ & $359^\circ 10'$ & $179^\circ 06'$ \\ \hline
        2 & $348^\circ 20'$ & $168^\circ 20'$ & $1^\circ 50'$ & $181^\circ 35'$ \\ \hline
        3 & $345^\circ 35'$ & $165^\circ 35'$ & $4^\circ 39'$ & $184^\circ 39'$ \\ \hline
        4 & -- & -- & $7^\circ 35'$ & $187^\circ 30'$ \\ \hline
    \end{tabular}
    \caption{Wartości kątów odpowiadające położeniu prążków.}
    \label{tab:pomiary_katow}
\end{table}

% ---------- OBLICZENIA ----------
\subsection{Średni kąt ugięcia}

Dla każdego rzędu $n \in [1, 3]$ średni kąt ugięcia $\varphi$ obliczony został jako średnia arytmetyczna kąta ugięcia z lewej i prawej strony wiązki nieugiętej \eqref{eq:kat_ugiecia}. Otrzymane wartości zapisano w tabeli \ref{tab:katy_ugiecia}.

Natomiast dla rzędu $n = 0$...

\begin{equation}
    \label{eq:kat_ugiecia}
    \varphi = \frac{\varphi_{1L} - \varphi_{1P} + 
    \varphi_{2L} - \varphi_{2P}}{2}
\end{equation}

\begin{table}[H]
    \centering
    \begin{tabular}{|c|r|}
        \hline
        \textbf{Numer prążka ($n$)} & \textbf{Kąt ugięcia $\varphi_n\,[\si{\degree}]$} \\
        \hline
        \num{0} & -- \\ \hline
        \num{1} & \num{-8.05} \\ \hline
        \num{2} & \num{166.625} \\ \hline
        \num{3} & \num{160.933} \\ \hline
    \end{tabular}
    \caption{Obliczone kąty ugięcia $\varphi_n$ dla każdego rzędu prążka.}
    \label{tab:katy_ugiecia}
\end{table}







% ---------- NIEPEWNOŚCI ----------
% \section{Ocena niepewności pomiaru}

% ---------- WNIOSKI ----------
% \section{Wnioski}

% ---------- WYKRESY ----------
% \section{Wykresy}

\bibliographystyle{apalike}
\bibliography{bibliography}

\end{document}