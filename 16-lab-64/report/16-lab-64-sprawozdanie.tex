\documentclass[a4paper,12pt]{article}
\usepackage[left=2cm,right=2cm,top=2cm,bottom=2cm]{geometry} % Do ustawień marginesów
\usepackage{multicol} % Dla podziału na kolumny
\usepackage{ragged2e} % Dla justowania tekstu
\usepackage{graphicx} % Required for inserting images
\usepackage{float}
\usepackage{caption}
\usepackage{amsmath} % Math formulas
\usepackage{amssymb} % Symbols
\usepackage[svgnames]{xcolor}
\usepackage[colorlinks=true, urlcolor=blue, linkcolor=black, citecolor=orange]{hyperref} % Hyperlinks
\usepackage{polski} % Polish language
\usepackage[utf8]{inputenc} % Text encoding
\usepackage{enumitem} % Pakiet do elastycznego sterowania listami
\usepackage{indentfirst}
\usepackage{array}
\usepackage{longtable}
\usepackage{pdflscape}
\usepackage[round]{natbib}
\setlist[itemize]{itemsep=0pt, topsep=0pt}
\usepackage{siunitx}
\sisetup{output-decimal-marker={,}}
\sisetup{exponent-product = \cdot}
% LTeX: language=pl-PL

\begin{document}

% Górna część strony
\noindent
\begin{minipage}{0.5\textwidth}
	\raggedright
	\textbf{Piotr Durniat, 347264} \\
	II rok, Fizyka \\
	Wtorek, 8:00-10:15
	\vspace{0.5cm}
\end{minipage}
\begin{minipage}{0.5\textwidth}
	\raggedleft
	04.11.2025 \\
	\vspace{0.5cm}
	Prowadząca: \\
	dr Sylwia Owczarek
\end{minipage}

% Tytuł ćwiczenia
\vspace{2cm}
\begin{center}
	\LARGE \textbf{Ćwiczenie nr 64} \\[0.5cm]
	\Large \textbf{Wyznaczanie stałej siatki dyfrakcyjnej przy użyciu spektrometru}
\end{center}

% Reszta treści
\vspace{1cm} % Kolejny odstęp
\noindent

% \tableofcontents
% \newpage

% ---------- WSTĘP TEORETYCZNY ----------
\section{Wstęp teoretyczny}

\subsection*{Siatka dyfrakcyjna, dyfrakcja i interferencja}
Siatka dyfrakcyjna to układ składający się z szeregu szczelin umieszczonych w równych odległościach od siebie na nieprzezroczystym ekranie. W praktyce siatkę taką otrzymuje się często poprzez porysowanie płaskorównoległej płytki szklanej równoległymi kreskami przy użyciu diamentu. Nieprzezroczyste rysy pełnią funkcję zasłon, a przezroczyste przestrzenie między nimi działają jak szczeliny. Odległość między środkami sąsiednich szczelin nazywana jest \textbf{stałą siatki} i oznaczana jako $d$ \citep{Drynski1976}.

Gdy na siatkę pada prostopadle wiązka promieni równoległych, zgodnie z zasadą Huygensa, każda szczelina staje się źródłem nowych drgań, które rozchodzą się we wszystkich kierunkach. Zjawisko to, polegające na uginaniu prostoliniowego biegu promieni, nazywane jest \textbf{dyfrakcją} \citep{Drynski1976}.

Promienie ugięte na różnych szczelinach są promieniami spójnymi, co oznacza, że mogą się ze sobą nakładać, czyli \textbf{interferować}. W zależności od kierunku (kąta ugięcia $\varphi$), promienie te będą się wzajemnie wzmacniać lub wygaszać \citep{Drynski1976}.

\subsection*{Warunek wzmocnienia}
Do wzmocnienia promieni ugiętych (interferencji konstruktywnej) dochodzi w tych kierunkach, dla których różnica dróg geometrycznych promieni wychodzących z sąsiednich szczelin jest równa całkowitej wielokrotności długości fali światła padającego ($\lambda$). Warunek ten, zwany równaniem siatki dyfrakcyjnej, ma postać \citep{Drynski1976}:
$$
	n\lambda = d \sin \varphi
$$
gdzie:
\begin{itemize}
	\item $n = 1, 2, 3, \dots$ -- rząd widma (kolejny numer wzmocnienia),
	\item $\lambda$ -- długość fali światła,
	\item $d$ -- stała siatki dyfrakcyjnej,
	\item $\varphi$ -- kąt ugięcia promieni, pod którym obserwuje się wzmocnienie.
\end{itemize}

\subsection*{Sieć krystaliczna jako siatka dyfrakcyjna}
Promieniowanie rentgenowskie (promienie X) ma długość fali (rzędu 0,1 nm) porównywalną z odległościami między atomami w ciałach stałych. Z tego powodu regularna, trójwymiarowa struktura atomów w krysztale, zwana siecią krystaliczną, może działać dla promieni X jak trójwymiarowa siatka dyfrakcyjna. Gdy promienie X padają na kryształ, ulegają ugięciu (odbiciu) na kolejnych równoległych płaszczyznach atomowych. Dochodzi do interferencji konstruktywnej (wzmocnienia) tylko pod określonymi kątami, zgodnie z warunkiem znanym jako \textbf{prawo Bragga} \citep{fizyka_dla_szkół_wyższych_tom_3}:
$$
	n\lambda = 2d \sin \theta
$$
gdzie $n$ to rząd widma (liczba całkowita), $\lambda$ to długość fali promieniowania X, $d$ to odległość między sąsiednimi płaszczyznami sieci krystalicznej, a $\theta$ to kąt padania promieni mierzony względem płaszczyzny kryształu \citep{fizyka_dla_szkół_wyższych_tom_3}.

\subsection*{Zasada pomiaru spektrometrem}
Spektrometr jest przyrządem pozwalającym na uzyskanie dokładniejszych wyników pomiarów kątów ugięcia $\varphi$ niż metody uproszczone. Pomiar polega na ustawieniu siatki dyfrakcyjnej na stoliku spektrometru, prostopadle do osi kolimatora \citep{Drynski1976}.

Za pomocą lunety spektrometru, wyposażonej w noniusze, odczytuje się położenia kątowe dla prążków interferencyjnych (linii widmowych) danego rzędu $n$. Aby wyznaczyć średni kąt ugięcia, notuje się pozycje noniuszów $a_1$ i $a_2$ dla prążka po jednej stronie obrazu nieugiętego (centralnego) oraz $b_1$ i $b_2$ dla symetrycznego prążka tego samego rzędu po drugiej stronie. Średni kąt ugięcia $\varphi$ oblicza się ze wzoru \citep{Drynski1976}:
$$
	\varphi = \frac{a_1 - b_1 + a_2 - b_2}{2}
$$
Mierząc kąt ugięcia $\varphi$ dla światła o znanej długości fali $\lambda$ (np. z lasera), można na podstawie równania siatki wyznaczyć jej stałą $d$ \citep{Drynski1976}:
$$
	d = \frac{n\lambda}{\sin \varphi}
$$

% ---------- OPIS DOŚWIADCZENIA ----------
\section{Opis doświadczenia}

Pomiary przeprowadzono zgodnie z następującą procedurą:
\begin{enumerate}
	\item Włączenie zasilania lasera i oświetlenie szczeliny kolimatora.
	\item Ustawienie siatki dyfrakcyjnej na stoliku spektrometru w taki sposób, aby położenie prążka zerowego było bliskie zeru na skali kątowej (lub zanotowanie jego dokładnego położenia).
	\item Odczytanie wartości kątów odpowiadających prążkom pierwszego rzędu po lewej oraz prawej stronie skali.
	\item Zmierzenie wartości kątów dla kolejnych par prążków wyższych rzędów.
\end{enumerate}

% ---------- OPRACOWANIE WYNIKÓW POMIARÓW ----------
\section{Opracowanie wyników pomiarów}


% ---------- TABELE ----------
\subsection{Tabele pomiarowe}

\begin{table}[H]
	\centering
	\begin{tabular}{|c|c|c|c|c|}
		\hline
		           & \multicolumn{2}{c|}{\textbf{Lewa}} & \multicolumn{2}{c|}{\textbf{Prawa}}                                                     \\ \hline
		\textbf{n} & \textbf{$\varphi_{1L}$}            & \textbf{$\varphi_{2L}$}             & \textbf{$\varphi_{1P}$} & \textbf{$\varphi_{2P}$} \\ \hline
		0          & $353^\circ 40'$                    & --                                  & --                      & $173^\circ 50'$         \\ \hline
		1          & $351^\circ 05'$                    & $171^\circ 05'$                     & $359^\circ 10'$         & $179^\circ 06'$         \\ \hline
		2          & $348^\circ 20'$                    & $168^\circ 20'$                     & $1^\circ 50'$           & $181^\circ 35'$         \\ \hline
		3          & $345^\circ 35'$                    & $165^\circ 35'$                     & $4^\circ 39'$           & $184^\circ 39'$         \\ \hline
		4          & --                                 & --                                  & $7^\circ 35'$           & $187^\circ 30'$         \\ \hline
	\end{tabular}
	\caption{Wartości kątów odpowiadające położeniu prążków.}
	\label{tab:pomiary_katow}
\end{table}

% ---------- OBLICZENIA ----------
\subsection{Średni kąt ugięcia}

Dla każdego rzędu $n \in [1, 3]$ średni kąt ugięcia $\varphi_n$ obliczony został jako średnia arytmetyczna separacji kątowej między prążkami z lewej i prawej strony wiązki nieugiętej \eqref{eq:kat_ugiecia}.

Różnicę kątową $|\dots|_{\text{cykl}}$ wyznaczano jako moduł różnicy odczytów. Ze względu na cykliczność skali goniometru, w przypadku gdy różnica ta przekraczała $180^\circ$ (przejście przez zero), obliczano dopełnienie do kąta pełnego ($360^\circ - |\text{odczyt}_L - \text{odczyt}_P|$). Otrzymane wartości zapisano w tabeli \ref{tab:katy_ugiecia}. Natomiast dla rzędu $n = 0$ kąt ugięcia przyjęto z definicji jako $\varphi_0 = 0^\circ$.

\begin{equation}
	\label{eq:kat_ugiecia}
	\varphi_n = \frac{|\varphi_{1L} - \varphi_{1P}|_{\text{cykl}} + |\varphi_{2L} - \varphi_{2P}|_{\text{cykl}}}{2}
\end{equation}

\subsubsection*{Przykładowe obliczenia (dla $n=1$)}
Podstawiając wartości kątowe zmierzone dla prążków pierwszego rzędu:
$$
	\Delta \varphi_{\text{noniusz 1}} = |351^\circ 05' - 359^\circ 10'| = 8^\circ 05'
$$
$$
	\Delta \varphi_{\text{noniusz 2}} = |171^\circ 05' - 179^\circ 06'| = 8^\circ 01'
$$
Średnia separacja kątowa:
$$
	\varphi_1 = \frac{8^\circ 05' + 8^\circ 01'}{2} = 8^\circ 03' = 8 + \frac{3}{60}^\circ \approx 8,050^\circ
$$

\begin{table}[H]
	\centering
	\begin{tabular}{|c|r|}
		\hline
		\textbf{Numer prążka ($n$)} & \textbf{Separacja kątowa $\varphi_n\,[\si{\degree}]$} \\
		\hline
		\num{1}                     & \num{8.050}                                           \\ \hline
		\num{2}                     & \num{13.375}                                          \\ \hline
		\num{3}                     & \num{19.067}                                          \\ \hline
	\end{tabular}
	\caption{Obliczone średnie separacje kątowe \(\varphi_n\) (wg wzoru \ref{eq:kat_ugiecia}).}
	\label{tab:katy_ugiecia}
\end{table}
\subsection{Wyznaczenie stałej siatki dyfrakcyjnej}

Stała siatki dyfrakcyjnej dla każdego rzędu $n$ obliczona została ze wzoru \eqref{eq:stala_siatki}. Następnie obliczono końcową wartość stałej siatki dyfrakcyjnej $\bar{d}$ jako średnią arytmetyczną tych wszystkich wartości. Wartości dla poszczególnych rzędów oraz wartość średnią zapisano w tabeli \ref{tab:stala_siatki}.

\begin{equation}
	\label{eq:stala_siatki}
	d = \frac{n\lambda}{\sin \varphi}
\end{equation}
gdzie $\lambda = \SI{632.8}{\nano\meter}$ to długość fali światła lasera użytego w doświadczeniu.


\begin{table}[H]
	\centering
	\begin{tabular}{|c|S[table-format=1.3e-1]|}
		\hline
		\textbf{Numer prążka ($n$)}        & {\textbf{Obliczona stała siatki $d_n\,[\si{\meter}]$}} \\
		\hline
		\num{1}                            & \num{4.519e-6}                                         \\ \hline
		\num{2}                            & \num{5.471e-6}                                         \\ \hline
		\num{3}                            & \num{5.811e-6}                                         \\ \hline
		\textbf{Wartość średnia $\bar{d}$} & \num{5.267e-6}                                         \\
		\hline
	\end{tabular}
	\caption{Obliczone wartości stałej siatki dyfrakcyjnej \(d\).}
	\label{tab:stala_siatki}
\end{table}

\subsection{Liczba rys na milimetr}

Znając średnią stałą siatki $\bar{d}$, wyznaczono liczbę rys (szczelin) przypadającą na szerokość \SI{1}{\milli\meter}. Wielkość tę obliczono jako odwrotność stałej siatki wyrażonej w milimetrach:

\begin{equation}
	N = \frac{1}{\bar{d}\,[\si{\milli\meter}]} = \frac{\num{1e-3}\,\si{\meter}}{\bar{d}\,[\si{\meter}]}
\end{equation}

Podstawiając obliczoną wartość średnią $\bar{d} \approx \SI{5.267e-6}{\meter}$:

\begin{equation}
	N = \frac{\num{1e-3}}{\num{5.267e-6}} \approx \SI{189.86}{\per\milli\meter}
\end{equation}


% ---------- NIEPEWNOŚCI ----------
\section{Ocena niepewności pomiaru}

\subsection{Niepewność pomiaru kąta ugięcia}

Niepewność standardową pomiaru kąta ugięcia oszacowano metodą typu B. Przyjęto dokładność noniusza goniometru $\Delta \varphi = 1' = \frac{1}{60} \SI{}{\degree}$. Zakładając rozkład prostokątny, niepewność wynosi:

\begin{equation}
	u(\varphi) = \frac{\Delta \varphi}{\sqrt{3}} = \frac{\frac{1}{60}}{\sqrt{3}} \approx \SI{0.0096}{\degree} \approx \SI{1.7e-4}{rad}
\end{equation}

\subsection{Niepewność stałej siatki dla poszczególnych rzędów \( u(d) \)}

Niepewność wyznaczenia stałej siatki \( d \) dla każdego rzędu obliczono metodą przenoszenia błędu (z prawa propagacji niepewności) dla wzoru na stałą siatki \ref{eq:stala_siatki}. Ponieważ długość fali \( \lambda \) uznano za wielkość dokładną, a \( n \) jest stałą, jedynym źródłem niepewności jest kąt \( \varphi \).

Wyprowadzenie wzoru na niepewność \( u(d) \):
\begin{align*}
	u(d)                                & = \left| \frac{\partial d}{\partial \varphi} \right| u(\varphi) \\
	\frac{\partial d}{\partial \varphi} & = -d \cdot \ctg \varphi
\end{align*}
Zatem:
\begin{equation}
	u(d) = d \cdot |\ctg \varphi| \cdot u(\varphi)
\end{equation}
(gdzie \( u(\varphi) \) wyrażono w radianach).

\subsubsection*{Przykładowe obliczenia (dla $n=1$)}
Podstawiając wartości dla pierwszego rzędu: $d_1 \approx \SI{4.52e-6}{\meter}$, $\varphi_1 \approx \ang{8.05}$ oraz $u(\varphi) \approx \num{1.7e-4}$ rad:

\begin{equation}
	u(d_1) = \num{4.52e-6} \cdot |\ctg(\ang{8.05})| \cdot \num{1.7e-4} \approx \SI{5.4e-9}{\meter}
\end{equation}


\subsection{Niepewność średniej stałej siatki}

Ze względu na zauważalny rozrzut wyników stałej siatki dla różnych rzędów widma, ostateczną niepewność wyniku średniego obliczono metodą typu A, jako odchylenie standardowe średniej arytmetycznej:

$$
	u(\bar{d}) = \sqrt{\frac{1}{k(k-1)} \sum_{i=1}^{k} (d_i - \bar{d})^2}
$$

Podstawiając obliczone wartości dla \( k=3 \) rzędów (gdzie \(\bar{d} \approx \SI{5.27e-6}{\meter}\)):

\begin{align*}
	 & u(\bar{d}) = \sqrt{\frac{1}{3(3-1)} \left[ (\num{4.52e-6} - \num{5.27e-6})^2 + \dots + (\num{5.81e-6} - \num{5.27e-6})^2 \right]} \\
	 & \approx \SI{3.9e-7}{\meter}
\end{align*}

\subsection{Niepewność liczby rys na milimetr}

Niepewność wyznaczenia gęstości rys \( N = 1/\bar{d} \) obliczono z prawa propagacji niepewności:

\begin{equation}
	u(N) = \left| \frac{\partial N}{\partial \bar{d}} \right| u(\bar{d}) = \frac{u(\bar{d})}{\bar{d}^2} = N \cdot \frac{u(\bar{d})}{\bar{d}}
\end{equation}

Podstawiając wartości (\( \bar{d} \approx \SI{5.27e-6}{\meter} \), \( u(\bar{d}) \approx \SI{7.4e-7}{\meter} \), \( N \approx \SI{190}{\per\milli\meter} \)):

\begin{equation}
	u(N) \approx \SI{190}{\per\milli\meter} \cdot \frac{\num{7.4e-7}}{\num{5.27e-6}} \approx \SI{1.4e1}{\per\milli\meter}
\end{equation}

% ---------- WNIOSKI ----------
\section{Wnioski}

Celem ćwiczenia było wyznaczenie stałej siatki dyfrakcyjnej przy użyciu spektrometru goniometrycznego. Na podstawie przeprowadzonych pomiarów kątów ugięcia dla trzech rzędów widma sformułowano następujące wnioski:

\begin{enumerate}
	\item Wyznaczona średnia wartość stałej siatki dyfrakcyjnej oraz jej niepewność standardowa wynoszą odpowiednio:
	      $$
		      \bar{d} = \num{5.27e-6}\,\si{\meter}
	      $$
	      $$
		      u(\bar{d}) = \num{3.9e-7}\,\si{\meter}
	      $$

	\item Na podstawie wyznaczonej stałej obliczono gęstość siatki, która wynosi:
	      $$
		      N \approx 190\,\si{\per\milli\meter}
	      $$
	      z niepewnością $u(N) \approx 14\,\si{\per\milli\meter}$.

	\item Analiza niepewności wskazuje na dominujący udział niepewności typu A (statystycznej) nad niepewnością typu B (wynikającą z dokładności noniusza). Wartość $u(\bar{d})$ rzędu $10^{-7}\,\si{\meter}$ jest znacznie wyższa niż niepewności pojedynczych pomiarów $u(d_n)$ (rzędu $10^{-9}\,\si{\meter}$), co świadczy o dużym rozrzucie wyników między poszczególnymi rzędami widma.

	\item Zauważalna jest systematyczna zależność wyników od rzędu widma (wzrost wartości $d$ wraz ze wzrostem $n$). Sugeruje to występowanie błędu systematycznego, którego źródłem mogło być nieidealnie prostopadłe ustawienie płaszczyzny siatki dyfrakcyjnej względem osi wiązki padającej (osi kolimatora).
\end{enumerate}

% ---------- WYKRESY ----------
% \section{Wykresy}

\bibliographystyle{apalike}
\bibliography{bibliography}

\end{document}
