\documentclass[a4paper,12pt]{article}
\usepackage[left=2cm,right=2cm,top=2cm,bottom=2cm]{geometry} 
\usepackage{multicol} 
\usepackage{ragged2e} 
\usepackage{graphicx} 
\usepackage{float}
\usepackage{caption}
\usepackage{amsmath} 
\usepackage{amssymb} 
\usepackage[svgnames]{xcolor}
\usepackage[colorlinks=true, urlcolor=blue, linkcolor=black, citecolor=orange]{hyperref} 
\usepackage{polski} 
\usepackage[utf8]{inputenc} 
\usepackage{enumitem} 
\usepackage{indentfirst}
\usepackage{array}
\usepackage{longtable}
\usepackage{pdflscape}
\usepackage[round]{natbib}
\setlist[itemize]{itemsep=0pt, topsep=0pt}
\usepackage{siunitx}

% SI setup preferences
\sisetup{output-decimal-marker={,}}
\sisetup{exponent-product = \cdot}
\sisetup{per-mode = symbol}

% LTeX: language=pl-PL

\begin{document}

\noindent
\begin{minipage}{0.5\textwidth}
	\raggedright
	% full name, index number
	\textbf{Piotr Durniat, 347264} \\
	II rok, Fizyka \\
	Wtorek, 8:00-10:15
	\vspace{0.5cm}
\end{minipage}
\begin{minipage}{0.5\textwidth}
	\raggedleft
	% date of the experiments
	02.12.2025 \\
	\vspace{0.5cm}
	Prowadząca: \\
	dr Sylwia Owczarek
\end{minipage}

\vspace{2cm}
\begin{center}
	% lab number
	\LARGE \textbf{Ćwiczenie nr 43} \\[0.5cm]
	% title
	\Large \textbf{Prawo Ohma dla prądu stałego}
\end{center}

\vspace{1cm}
\noindent

% \tableofcontents
% \newpage

% ---------- WSTĘP TEORETYCZNY ----------
\section{Wstęp teoretyczny}

\subsection{Przewodnictwo elektryczne w metalach i półprzewodnikach}
W metalach nośnikami prądu elektrycznego są elektrony swobodne, tworzące tzw. gaz elektronowy. Ich koncentracja jest stała i niezależna od temperatury \citep{fizyka_tom3}. W półprzewodnikach samoistnych nośnikami są elektrony w paśmie przewodnictwa oraz dziury w paśmie walencyjnym, powstające w wyniku generacji termicznej. Koncentracja nośników w półprzewodnikach silnie zależy od temperatury oraz obecności domieszek (półprzewodniki typu n i p) \citep{wstep_do_cwiczenia}.

\subsection{Natężenie prądu i prawo Ohma}
Natężenie prądu elektrycznego $I$ definiuje się jako stosunek ładunku $dQ$ przepływającego przez poprzeczny przekrój przewodnika do czasu $dt$:
$$
	I = \frac{dQ}{dt}
$$
Jednostką natężenia w układzie SI jest amper (\unit{\ampere}). Dla przewodników metalicznych w stałej temperaturze obowiązuje prawo Ohma, stwierdzające proporcjonalność natężenia prądu do przyłożonego napięcia $U$:
$$
	I = \frac{1}{R} U = G U
$$
gdzie $R$ to opór elektryczny, a $G$ to konduktancja \citep{fizyka_tom3}.

\subsection{Opór elektryczny i jego zależność od temperatury}
Opór elektryczny $R$ jest miarą przeciwstawiania się elementu przepływowi prądu. Jednostką oporu jest om (\unit{\ohm}).
Dla metali opór rośnie wraz ze wzrostem temperatury w przybliżeniu liniowo:
$$
	R(T) = R_0 [1 + \alpha(T - T_0)]
$$
Wynika to ze wzrostu amplitudy drgań sieci krystalicznej, co zwiększa prawdopodobieństwo rozpraszania elektronów.
W półprzewodnikach opór maleje wykładniczo ze wzrostem temperatury:
$$
	R(T) = A \exp\left(\frac{E_g}{2kT}\right)
$$
Jest to spowodowane gwałtownym wzrostem koncentracji nośników ładunku, który dominuje nad efektem spadku ich ruchliwości \citep{wstep_do_cwiczenia}.

\subsection{Model pasmowy}
Właściwości elektryczne ciał stałych wyjaśnia teoria pasmowa, zakładająca istnienie dozwolonych pasm energetycznych (walencyjnego i przewodnictwa) rozdzielonych pasmem wzbronionym o szerokości $E_g$.
\begin{itemize}
	\item \textbf{Metale:} Pasmo walencyjne zachodzi na pasmo przewodnictwa lub pasmo przewodnictwa jest tylko częściowo zapełnione. Brak przerwy energetycznej ($E_g = 0$).
	\item \textbf{Półprzewodniki:} Istnieje niewielka przerwa energetyczna ($E_g < \SI{2}{eV}$). W $T=\SI{0}{K}$ pasmo przewodnictwa jest puste.
	\item \textbf{Izolatory:} Szeroka przerwa energetyczna ($E_g > \SI{3}{eV}$) uniemożliwia przeskok elektronów do pasma przewodnictwa w temperaturze pokojowej \citep{fizyka_tom3}.
\end{itemize}

\subsection{Charakterystyki prądowo-napięciowe}
Charakterystyka $I(U)$ obrazuje zależność prądu płynącego przez element od przyłożonego napięcia.
\begin{itemize}
	\item \textbf{Opór drutowy:} Charakterystyka liniowa, zgodna z prawem Ohma. Nachylenie prostej odpowiada konduktancji $1/R$.
	\item \textbf{Żarówka:} Element nieliniowy. Wraz ze wzrostem napięcia rośnie temperatura włókna, co powoduje wzrost oporu i wolniejszy przyrost prądu (charakterystyka o malejącym nachyleniu).
	\item \textbf{Dioda półprzewodnikowa:} Przewodzi prąd głównie w jedną stronę (polaryzacja przewodzenia). Zależność opisuje równanie Shockleya:
	      $$ I = I_s [\exp(\frac{eU}{nkT}) - 1] $$
	      W kierunku zaporowym płynie jedynie pomijalnie mały prąd nasycenia $I_s$ \citep{wstep_do_cwiczenia}.
	\item \textbf{Termistor:} Półprzewodnikowy rezystor czuły na temperaturę. Płynący prąd powoduje wydzielanie ciepła Joule'a, co zmienia jego rezystancję (zazwyczaj maleje dla termistorów NTC), prowadząc do nieliniowej charakterystyki $I(U)$.
\end{itemize}

\subsection{Reżimy pracy zasilacza}
Zasilacz laboratoryjny może pracować w dwóch podstawowych trybach \citep{instrukcja43}:
\begin{itemize}
	\item \textbf{CV (Constant Voltage):} Stabilizacja napięcia. Zasilacz utrzymuje stałe napięcie wyjściowe niezależnie od obciążenia (dopóki pobór prądu nie przekroczy ustawionego limitu).
	\item \textbf{CC (Constant Current):} Stabilizacja prądu. Zasilacz utrzymuje stały prąd wyjściowy, dostosowując napięcie do rezystancji obciążenia. Tryb ten aktywuje się po przekroczeniu ustawionego limitu prądowego.
\end{itemize}
% ---------- OPIS DOŚWIADCZENIA ----------
% \section{Opis doświadczenia}

% ---------- OPRACOWANIE WYNIKÓW POMIARÓW ----------
% \section{Opracowanie wyników pomiarów}

% ---------- TABELE ----------
% \subsection{Tabele pomiarowe}

% ---------- OBLICZENIA ----------
% \subsection{...}

% ---------- NIEPEWNOŚCI ----------
% \section{Ocena niepewności pomiaru}

% ---------- WNIOSKI ----------
% \section{Wnioski}

% ---------- WYKRESY ----------
% \section{Wykresy}
% \newpage

\bibliographystyle{apalike}
\bibliography{bibliography}

\end{document}
