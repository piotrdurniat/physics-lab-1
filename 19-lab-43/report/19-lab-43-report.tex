\documentclass[a4paper,12pt]{article}
\usepackage[left=2cm,right=2cm,top=2cm,bottom=2cm]{geometry} 
\usepackage{multicol} 
\usepackage{ragged2e} 
\usepackage{graphicx} 
\usepackage{float}
\usepackage{caption}
\usepackage{amsmath} 
\usepackage{amssymb} 
\usepackage[svgnames]{xcolor}
\usepackage[colorlinks=true, urlcolor=blue, linkcolor=black, citecolor=orange]{hyperref} 
\usepackage{polski} 
\usepackage[utf8]{inputenc} 
\usepackage{enumitem} 
\usepackage{indentfirst}
\usepackage{array}
\usepackage{longtable}
\usepackage{pdflscape}
\usepackage[round]{natbib}
\setlist[itemize]{itemsep=0pt, topsep=0pt}
\usepackage{siunitx}

% SI setup preferences
\sisetup{output-decimal-marker={,}}
\sisetup{exponent-product = \cdot}
\sisetup{per-mode = symbol}

% LTeX: language=pl-PL

\begin{document}

\noindent
\begin{minipage}{0.5\textwidth}
	\raggedright
	% full name, index number
	\textbf{Piotr Durniat, 347264} \\
	II rok, Fizyka \\
	Wtorek, 8:00-10:15
	\vspace{0.5cm}
\end{minipage}
\begin{minipage}{0.5\textwidth}
	\raggedleft
	% date of the experiments
	02.12.2025 \\
	\vspace{0.5cm}
	Prowadząca: \\
	dr Sylwia Owczarek
\end{minipage}

\vspace{2cm}
\begin{center}
	% lab number
	\LARGE \textbf{Ćwiczenie nr 43} \\[0.5cm]
	% title
	\Large \textbf{Prawo Ohma dla prądu stałego}
\end{center}

\vspace{1cm}
\noindent

% ---------- WSTĘP TEORETYCZNY ----------
\section{Wstęp teoretyczny}

\subsection{Przewodnictwo elektryczne w metalach i półprzewodnikach}
W metalach nośnikami prądu elektrycznego są elektrony swobodne, tworzące tzw. gaz elektronowy. Ich koncentracja jest stała i niezależna od temperatury \citep{fizyka_tom3}. W półprzewodnikach samoistnych nośnikami są elektrony w paśmie przewodnictwa oraz dziury w paśmie walencyjnym, powstające w wyniku generacji termicznej. Koncentracja nośników w półprzewodnikach silnie zależy od temperatury oraz obecności domieszek (półprzewodniki typu n i p) \citep{wstep_do_cwiczenia}.

\subsection{Natężenie prądu i prawo Ohma}
Natężenie prądu elektrycznego $I$ definiuje się jako stosunek ładunku $dQ$ przepływającego przez poprzeczny przekrój przewodnika do czasu $dt$:
$$
	I = \frac{dQ}{dt}
$$
Jednostką natężenia w układzie SI jest amper (\unit{\ampere}). Dla przewodników metalicznych w stałej temperaturze obowiązuje prawo Ohma, stwierdzające proporcjonalność natężenia prądu do przyłożonego napięcia $U$:
$$
	I = \frac{1}{R} U = G U
$$
gdzie $R$ to opór elektryczny, a $G$ to konduktancja \citep{fizyka_tom3}.

\subsection{Opór elektryczny i jego zależność od temperatury}
Opór elektryczny $R$ jest miarą przeciwstawiania się elementu przepływowi prądu. Jednostką oporu jest om (\unit{\ohm}).
Dla metali opór rośnie wraz ze wzrostem temperatury w przybliżeniu liniowo:
$$
	R(T) = R_0 [1 + \alpha(T - T_0)]
$$
Wynika to ze wzrostu amplitudy drgań sieci krystalicznej, co zwiększa prawdopodobieństwo rozpraszania elektronów.
W półprzewodnikach opór maleje wykładniczo ze wzrostem temperatury:
$$
	R(T) = A \exp\left(\frac{E_g}{2kT}\right)
$$
Jest to spowodowane gwałtownym wzrostem koncentracji nośników ładunku, który dominuje nad efektem spadku ich ruchliwości \citep{wstep_do_cwiczenia}.

\subsection{Model pasmowy}
Właściwości elektryczne ciał stałych wyjaśnia teoria pasmowa, zakładająca istnienie dozwolonych pasm energetycznych (walencyjnego i przewodnictwa) rozdzielonych pasmem wzbronionym o szerokości $E_g$.
\begin{itemize}
	\item \textbf{Metale:} Pasmo walencyjne zachodzi na pasmo przewodnictwa lub pasmo przewodnictwa jest tylko częściowo zapełnione. Brak przerwy energetycznej ($E_g = 0$).
	\item \textbf{Półprzewodniki:} Istnieje niewielka przerwa energetyczna ($E_g < \SI{2}{eV}$). W $T=\SI{0}{K}$ pasmo przewodnictwa jest puste.
	\item \textbf{Izolatory:} Szeroka przerwa energetyczna ($E_g > \SI{3}{eV}$) uniemożliwia przeskok elektronów do pasma przewodnictwa w temperaturze pokojowej \citep{fizyka_tom3}.
\end{itemize}

\subsection{Charakterystyki prądowo-napięciowe}
Charakterystyka $I(U)$ obrazuje zależność prądu płynącego przez element od przyłożonego napięcia.
\begin{itemize}
	\item \textbf{Opór drutowy:} Charakterystyka liniowa, zgodna z prawem Ohma. Nachylenie prostej odpowiada konduktancji $1/R$.
	\item \textbf{Żarówka:} Element nieliniowy. Wraz ze wzrostem napięcia rośnie temperatura włókna, co powoduje wzrost oporu i wolniejszy przyrost prądu (charakterystyka o malejącym nachyleniu).
	\item \textbf{Dioda półprzewodnikowa:} Przewodzi prąd głównie w jedną stronę (polaryzacja przewodzenia). Zależność opisuje równanie Shockleya:
	      $$ I = I_s [\exp(\frac{eU}{nkT}) - 1] $$
	      W kierunku zaporowym płynie jedynie pomijalnie mały prąd nasycenia $I_s$ \citep{wstep_do_cwiczenia}.
	\item \textbf{Termistor:} Półprzewodnikowy rezystor czuły na temperaturę. Płynący prąd powoduje wydzielanie ciepła Joule'a, co zmienia jego rezystancję (zazwyczaj maleje dla termistorów NTC), prowadząc do nieliniowej charakterystyki $I(U)$.
\end{itemize}

\subsection{Reżimy pracy zasilacza}
Zasilacz laboratoryjny może pracować w dwóch podstawowych trybach \citep{instrukcja43}:
\begin{itemize}
	\item \textbf{CV (Constant Voltage):} Stabilizacja napięcia. Zasilacz utrzymuje stałe napięcie wyjściowe niezależnie od obciążenia (dopóki pobór prądu nie przekroczy ustawionego limitu).
	\item \textbf{CC (Constant Current):} Stabilizacja prądu. Zasilacz utrzymuje stały prąd wyjściowy, dostosowując napięcie do rezystancji obciążenia. Tryb ten aktywuje się po przekroczeniu ustawionego limitu prądowego.
\end{itemize}

% ---------- OPIS DOŚWIADCZENIA ----------
\section{Opis doświadczenia}

Układ pomiarowy zasilano zasilaczem ZT-980-2M pracującym w trybie stabilizacji napięcia lub prądu.

\begin{enumerate}
	\item \textbf{Drut oporowy:} Zasilacz ustawiono w trybie regulacji napięcia (CV). Zmierzono zależność natężenia prądu od napięcia w zakresie $0 - \SI{1,5}{V}$ z krokiem co ok. \SI{100}{mV}.
	\item \textbf{Dioda prostownicza:} Zasilacz przełączono w tryb regulacji prądu (CC) w celu ochrony elementu przed uszkodzeniem. Wykonano pomiary w kierunku przewodzenia do natężenia \SI{200}{mA}.
	\item \textbf{Żarówka:} Pomiary wykonano w trybie CC, zmieniając natężenie prądu w zakresie $0 - \SI{200}{mA}$.
\end{enumerate}
Termistor został pominięty w badaniach. Niepewności pomiarowe oszacowano na podstawie ostatniej cyfry znaczącej wyświetlanej na miernikach cyfrowych.

% ---------- WYNIKI POMIARÓW ----------
\section{Wyniki pomiarów}
W poniższych tabelach zestawiono wyniki pomiarów dla poszczególnych elementów.

\begin{table}[H]
	\centering
	\caption{Wyniki pomiarów dla drutu oporowego (Zadanie 1).}
	\label{tab:resistor}
	\sisetup{table-format=1.4}
	\begin{tabular}{|S|S|}
		\hline
		{$U$ [\unit{\volt}]} & {$I$ [\unit{\ampere}]} \\
		\hline
		0,0001               & 0,01                   \\
		0,1092               & 0,02                   \\
		0,1985               & 0,03                   \\
		0,2902               & 0,04                   \\
		0,4029               & 0,06                   \\
		0,5119               & 0,07                   \\
		0,6001               & 0,08                   \\
		0,6920               & 0,09                   \\
		0,8000               & 0,10                   \\
		0,9000               & 0,11                   \\
		1,0090               & 0,12                   \\
		1,1000               & 0,13                   \\
		1,2000               & 0,14                   \\
		1,3000               & 0,15                   \\
		1,3990               & 0,16                   \\
		1,5000               & 0,17                   \\
		\hline
	\end{tabular}
\end{table}

\begin{table}[H]
	\centering
	\caption{Wyniki pomiarów dla diody prostowniczej (Zadanie 2).}
	\label{tab:diode}
	\sisetup{table-format=1.4}
	\begin{tabular}{|S|S|}
		\hline
		{$U$ [\unit{\volt}]} & {$I$ [\unit{\ampere}]} \\
		\hline
		0,5699               & 0,0001                 \\
		0,6744               & 0,0050                 \\
		0,7296               & 0,0101                 \\
		0,8083               & 0,0200                 \\
		0,8788               & 0,0300                 \\
		0,8797               & 0,0400                 \\
		0,8199               & 0,0501                 \\
		0,8406               & 0,0600                 \\
		0,8600               & 0,0702                 \\
		0,8705               & 0,0800                 \\
		0,8973               & 0,0900                 \\
		\hline
	\end{tabular}
	\quad
	\begin{tabular}{|S|S|}
		\hline
		{$U$ [\unit{\volt}]} & {$I$ [\unit{\ampere}]} \\
		\hline
		0,9139               & 0,1002                 \\
		0,9314               & 0,1100                 \\
		0,9501               & 0,1200                 \\
		0,9699               & 0,1299                 \\
		0,9904               & 0,1398                 \\
		0,9150               & 0,1503                 \\
		0,9270               & 0,1603                 \\
		0,9370               & 0,1699                 \\
		0,9490               & 0,1803                 \\
		0,9610               & 0,1902                 \\
		0,9720               & 0,1999                 \\
		\hline
	\end{tabular}
\end{table}

\begin{table}[H]
	\centering
	\caption{Wyniki pomiarów dla żarówki (Zadanie 3).}
	\label{tab:bulb}
	\sisetup{table-format=1.4}
	\begin{tabular}{|S|S|}
		\hline
		{$U$ [\unit{\volt}]} & {$I$ [\unit{\ampere}]} \\
		\hline
		0,0150               & 0,0002                 \\
		0,0345               & 0,0101                 \\
		0,0713               & 0,0200                 \\
		0,1125               & 0,0303                 \\
		0,1580               & 0,0401                 \\
		0,2149               & 0,0500                 \\
		0,3020               & 0,0600                 \\
		0,4179               & 0,0701                 \\
		0,5811               & 0,0803                 \\
		0,7472               & 0,0900                 \\
		0,9242               & 0,1001                 \\
		\hline
	\end{tabular}
	\quad
	\begin{tabular}{|S|S|}
		\hline
		{$U$ [\unit{\volt}]} & {$I$ [\unit{\ampere}]} \\
		\hline
		0,9960               & 0,1104                 \\
		1,1670               & 0,1198                 \\
		1,3690               & 0,1303                 \\
		1,5740               & 0,1404                 \\
		1,7830               & 0,1501                 \\
		2,0100               & 0,1602                 \\
		2,2350               & 0,1698                 \\
		2,4880               & 0,1801                 \\
		2,7430               & 0,1901                 \\
		3,0030               & 0,2000                 \\
		                     &                        \\
		\hline
	\end{tabular}
\end{table}

% ---------- OBLICZENIA ----------
\section{Opracowanie wyników}

\subsection{Wyznaczenie oporu drutu oporowego}

Znaleziono równanie prostej dla oporu drutowego metodą regresji liniowej. Przyjęto zależność $I(U) = G U + b$, gdzie $G$ to konduktancja. Na podstawie danych z Tabeli \ref{tab:resistor}:
\[
	I(U) = \num{0,1072} \cdot U + \num{0,0120}
\]
Opór elektryczny $R$ wyznaczono jako odwrotność współczynnika kierunkowego $G = \SI{0,1072}{\siemens}$:
\[
	R = \frac{1}{G} = \frac{1}{0,1072} \approx \SI{9,33}{\ohm}
\]
Niepewność oporu $u(R)$ obliczono w oparciu o błąd standardowy dopasowania prostej:
\[
	u(R) = \SI{0,14}{\ohm}
\]

\subsection{Wyznaczenie oporu różniczkowego żarówki}

Wyznaczono opór różniczkowy $R_d$ żarówki dla napięcia bliskiego \SI{600}{mV}. Wykorzystano dwa punkty pomiarowe: $(U_1, I_1) = (\SI{0,5811}{\volt}, \SI{0,0803}{\ampere})$ oraz $(U_2, I_2) = (\SI{0,7472}{\volt}, \SI{0,0900}{\ampere})$.
\[
	R_d = \frac{\mathrm{d}U}{\mathrm{d}I} \approx \frac{U_2 - U_1}{I_2 - I_1} = \frac{0,7472 - 0,5811}{0,0900 - 0,0803} = \frac{0,1661}{0,0097} \approx \SI{17,1}{\ohm}
\]
Niepewność $u(R_d)$ oszacowano metodą różniczki zupełnej, uwzględniając niepewności odczytu napięcia ($u(U)=\SI{0,0001}{\volt}$) i prądu ($u(I)=\SI{0,0001}{\ampere}$):
\[
	u(R_d) = R_d \sqrt{\left(\frac{\sqrt{2} \cdot u(U)}{\Delta U}\right)^2 + \left(\frac{\sqrt{2} \cdot u(I)}{\Delta I}\right)^2} = 17,1 \cdot \sqrt{\left(\frac{0,00014}{0,1661}\right)^2 + \left(\frac{0,00014}{0,0097}\right)^2} \approx \SI{2,5}{\ohm}
\]

% ---------- WNIOSKI ----------
\section{Wnioski}
\begin{itemize}
	\item \textbf{Drut oporowy:} Charakterystyka jest liniowa, co potwierdza zgodność z prawem Ohma dla metali w stałej temperaturze. Wyznaczona wartość rezystancji wynosi:
	      \[
		      R = \SI{9,33}{\ohm}, \quad u(R)=\SI{0,14}{\ohm}
	      \]
	      Dominującym składnikiem niepewności jest błąd dopasowania regresji liniowej.

	\item \textbf{Żarówka:} Charakterystyka jest nieliniowa (nachylenie maleje ze wzrostem napięcia), co jest przykładem elementu niespełniającego prawa Ohma (opór zależny od temperatury). Opór różniczkowy przy napięciu ok. \SI{0,6}{\volt} wynosi:
	      \[
		      R_d = \SI{17,1}{\ohm}, \quad u(R_d)=\SI{2,5}{\ohm}
	      \]

	\item \textbf{Dioda prostownicza:} Element ten również nie spełnia prawa Ohma, wykazując silną asymetrię przewodzenia i wykładniczy wzrost prądu w kierunku przewodzenia.
\end{itemize}

% ---------- WYKRESY ----------
% \newpage
\section{Wykresy}

\begin{figure}[p]
	\centering
	\includegraphics[width=0.9\textheight, angle=90]{img/plot_resistor.png}
	\caption{Charakterystyka prądowo-napięciowa drutu oporowego.}
	\label{fig:resistor}
\end{figure}
\clearpage

\begin{figure}[p]
	\centering
	\includegraphics[width=0.9\textheight, angle=90]{img/plot_bulb.png}
	\caption{Charakterystyka prądowo-napięciowa żarówki.}
	\label{fig:bulb}
\end{figure}
\clearpage

\begin{figure}[p]
	\centering
	\includegraphics[width=0.9\textheight, angle=90]{img/plot_diode.png}
	\caption{Charakterystyka prądowo-napięciowa diody w kierunku przewodzenia.}
	\label{fig:diode}
\end{figure}
\clearpage
\bibliographystyle{apalike}
\bibliography{bibliography}

\end{document}
